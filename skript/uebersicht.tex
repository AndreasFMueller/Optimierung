\chapter{"Ubersicht}
\lhead{"Ubersicht}
\rhead{}
Steven Wozniak war daf"ur ber"uhmt, den ersten Apple-Computer mit
einer m"oglichst kleinen Zahl von ICs konstruiert zu haben.
Ausserdem soll der Firmware Code besonders schlank gewesen sein, so
dass im vorhandenen Speicher mehr Funktionen untergebracht werden 
konnten. Hier wurden offenbar zwei Optimierungsprobleme gel"ost.
Beide d"urften zwar nicht ausschlaggebend f"ur den Erfolg von
Apple gewesen sein, aber einen Beitrag dazu geleistet haben.

Vor einer "ahnlichen Herausforderung steht man h"aufig: mit m"oglichst
geringem Aufwand soll ein m"oglichst gutes Resultat erreicht werden.
Typischerweise sind auf der Aufwand-Seite immer mehrere Gr"ossen varierbar.
Bei Embedded Hardware kann man in gewissen Grenzen
fehlende Hardware-Funktionalit"at durch Software kompensieren.
Billigere Rohmaterialen brauchen gr"osseren Bearbeitungsaufwand.
Aber auch auf der Resultat-Seite sind oft verschiedene Gr"ossen zu
betrachten.
Will man mit dem Produkt vor allem Geld verdienen, oder vor allem
das Image f"ordern?
Was ist dem Kunden mehr wert: ein kleines und leichtes
Ger"at, oder ein robustes, nicht kaputt zu kriegendes? 
In vielen F"allen l"asst sich das Ziel auf eine einfache
Zahl reduzieren: Gewinnmaximierung.

Optimierungsmethoden sollen helfen, mathematisch formulierbare
Optimierungsprobleme zu l"osen. Dazu ist zun"achst zu kl"aren,
wie ein mathematisches Optimierungsproblem aussieht.
Das oben erw"ahnte Problem von Steven Wozniak ist wohl eher nicht
von dieser Art. Beispiele von mathematischen Optimierungsproblemen
sollen im Abschnitt \label{section-einfuehrungsbeispiele} einen
ersten Eindruck davon geben, was in diesem Skript behandelt werden soll.
Abschnitt \label{section-begriffe} stellt die im Weiteren verwendeten
Begriffe zusammen. Abschnitt \label{section-klassifizierung} versucht
eine grobe Klassifizierung der Optimierungsprobleme, und in Abschnitt
\label{section-loesungsuebersicht} werden die im Folgenden zu
besprechenden L"osungsmethoden kurz skizziert.

\section{Einf"uhrungsbeispiele\label{section-einfuehrungsbeispiele}}
\rhead{Beispiel}
Wir stellen ein paar Probleme zusammen, die den allgemeinen Charakter
von Optimierungsaufgaben verdeutlichen sollen.

\subsection{Solarzellen\label{uebersicht:solarzellen}}
Die Leistung, die eine photovoltaische Zelle abgeben kann, ist proportional
zum einfallenden Lichtstrom. Maximale Leistung erreicht man also,
wenn der einfallende Strom maximal ist. Die Stromdichte (Leistung 
pro Quadratmeter) ist konstant, aber der Winkel $\alpha$ zwischen der
Zellennormalen und der Richtung des einfallenden Sonnenlichts schwankt.
Wie muss man die Solarzellen ausrichten, um maximale Leistung zu erhalten?

Die produzierte Leistung ist proportional zu $\cos\alpha$, erreicht also
das Maximum dort, wo die Funktion $\alpha\mapsto\cos\alpha$ ihr
Maximum erreicht.

Etwas interessanter wird das Problem, wenn nicht die maximale momentane
Leistung interessiert, sondern die "uber den Tag gewonnene Gesamtenergie.
Die Leistung schwankt wegen des sich "andernden Sonnenstandes, es w"are
daher interessant zu wissen, bei welcher Elevation sich im insgesamt
die beste Energieausbeute ergeben w"urde.

Unter der Annahme, dass die Strahlung "uber den ganzen Tag gleich bleibt,
brauchen wir ein Modell f"ur die Elevation der Sonne "uber dem Horizont.
Bei Tag- und Nachtgleiche geht die Sonne um 0600 auf, um 1800 unter,
und erreicht als maximale Elevation "uber dem Horizont $1-\beta$,
wobei $\beta$ die geographische Breite des Standortes ist.
Zu anderen Zeiten des Jahres kommt noch die Elevation $\delta$
(f"ur Deklination) der Sonne "uber dem Himmels"aquator hinzu.
Wir k"onnen die zeitabh"angige Elevation der Sonne also mit einem
einfachen Modell
\[
t\mapsto \delta + (1-\beta)\cos t,\quad t_A\le t\le t_U
\]
wobei $t_A$ die Zeit des Sonnenaufgangs vor dem h"ochsten Sonnenstand
und $t_U$ die Zeit des Untergangs nach dem h"ochsten Stand ist.
Die Gesamtenergie in Abh"angigkeit von der Neigung der
Solarzellen ist dann
\[
f(\alpha)=
\int_{t_A}^{t_U} \cos (\delta+(1-\beta)\cos t - \alpha)\,dt
\]
und es ist der Wert von $\alpha$ zu finden, der $f(\alpha)$
maximiert.

Nat"urlich weiss man, wie man ein solches Problem l"ost: man leitet
$f(\alpha)$ nach $\alpha$ ab und sicht nach Nullstellen:
\[
f'(\alpha)=\int_{t_A}^{t_U}
\sin(\delta + (1-\beta)\cos t-\alpha)
\,dt
\]
Mit Hilfe der Additionstheoreme erh"alt man
\begin{align*}
0=
f'(\alpha)
&=
\int_{t_A}^{t_U}
\sin(\delta + (1-\beta)\cos t)\cos\alpha
-
\cos(\delta + (1-\beta)\cos t)\sin\alpha
\,dt
\\
&=
\cos\alpha
\int_{t_A}^{t_U}
\sin(\delta + (1-\beta)\cos t)\,dt
-
\sin\alpha
\int_{t_A}^{t_U}
\cos(\delta + (1-\beta)\cos t)
\,dt
\\
\tan\alpha
&=
\frac{
\int_{t_A}^{t_U}
\sin(\delta + (1-\beta)\cos t)\,dt
}{
\int_{t_A}^{t_U}
\cos(\delta + (1-\beta)\cos t)\,dt.
}
\end{align*}
Dieses Optimierungsproblem kann also mit Hilfe der Analysis direkt
gel"ost werden.

Noch etwas spannender wird das Problem allerdings, wenn nicht nur
die Energieausbeute eines einzelnen Tages, sondern die Ausbeute
"uber ein ganzes Jahr optimiert werden soll, w"ahrend das Problem allerdings, wenn nicht nur
die Energieausbeute eines einzelnen Tages, sondern die Ausbeute
"uber ein ganzes Jahr optimiert werden soll. Die Deklination $\delta$
der Sonnen schwankg im Laufe des Jahres um "uber $46^\circ$, und die
Sonnenaufgangs- und Untergangszeiten $t_A$ und $t_U$ sind auch nicht mehr 
so einfach zu bestimmen. Dieses erweiterte Problem ist nur noch
numerische zu l"osen. Noch komplizierter wird es, wenn nicht der Energieertrag,
sondern der finanzielle Ertrag "uber das ganze Jahr optimiert werden soll,
der Ertrag pro Einergieeinheit h"angt ja
auch noch von Tageszeit und Jahreszeit ab.

\subsection{Hackfleischskandal\label{uebersicht:hackfleisch}}
Ein Pizza-Hersteller braucht f"ur seine Pizza Hackfleisch, wobei er
zur Gewinnoptimierung Rindfleisch mit billigerem Pferdefleisch mischen
will. Die kg-Kosten f"ur Rinds-Hack und f"ur Pferde-Hack sind bekannt.
Seine Zulieferer k"onnen jedoch nur eine bestimmte maximale Menge
an Pferdefleisch liefern, da der Markt nicht mehr hergibt.
Da die Verarbeitung des Pferdefleisches Resourcen beim Zulieferer
bindet, kann der Zulieferer umso weniger Rindfleisch liefern, je mehr
Pferdefleisch der Pizza-Hersteller bestellt. In welchem Verh"altnis
muss er Pferde und Rindfleisch bestellen, um einen m"oglichst hohen
Gewinn zu erzielen?

Offenbar muss der Pizza-Hersteller eine Menge $x$ von Pferdefleisch
und eine Menge $y$ von Rindfleisch bestellen. Sein Zulieferer hat
nur die Lieferkapazit"at $c$, es muss also gelten $x+y\le c$.
Der Markt f"ur Pferdefleisch ist eher klein, es muss also gelten
$x\le m$ sein. Und nat"urlich m"ussen $x\ge 0$ und $y\ge 0$ sein.
Die "ubrigen Kosten und der Umsatz sind proportional zur Menge
der gesamthaft produzierten Pizzas die immer gleich viel Fleisch
enthalten (hier best"unde f"ur so einen diebischen Pizza-Hersteller
durchaus auch noch ``Optimierungspotential''), also zu $x+y$. Die
Kosten f"ur die Fleischsorten sind bekannt, nennen wir sie $c_x$ und $c_y$.
Der Pizzahersteller versucht also den Gewinn
\[
G=
c(x+y)-c_xx-c_yy
=(c-c_x)x+(c-c_y)y
\]
zu maximieren unter der Einschr"ankungen
\begin{align*}
x&\ge 0,\\y&\ge 0,\\
x+y&\le c,\\
x&\le m.
\end{align*}
Der Gewinn wie auch die Einschr"ankungen sind lineare Gleichungen
bzw.~Ungleichungen.

\subsection{Tankwanne\label{uebersicht:wanne}}
Es soll eine quaderf"ormige Wanne aus Stahlblech zusammengeschweisst
werden, die ein Fassungsverm"ogen $V$ haben muss, sie soll n"amlich
auslaufendes "Ol eines ebenfalls quaderf"ormigen Tankes auffangen k"onnen,
dessen Grundfl"ache L"ange $l$ und Breite $b$ hat, und der in die
Wanne gestellt werden soll.
Das Blech hat die Fl"achendichte $\varrho$. Der Preis der Wann setzt sich
zusammen aus den Masseabh"angigen Materialkosten $c_m$ und den allein
von der L"ange der Schweissnaht abh"angigen Kosten f"ur den Schweisser
$c_s$. Wie muss die Wanne konstruiert werden, damit sie m"oglichst 
billig wird?

Die Wanne ist quaderf"ormig, nennen wir $(x,y,z)$ die L"ange, Breite
und H"ohe der Wanne. Damit die Wanne das verlangte
Fassungsverm"ogen hat, muss $xyz\ge V$ sein.
Damit der Tank in die Wanne passt, muss $x\ge l$ und $y\ge b$ sein.
Die Masse der Wanne setzt sich zusammen aus der Masse von Boden
und Seitenfl"achen, also
$\varrho(xy+2xz+2yz)$. Der Schweisser muss vier vertikale N"ahte
der L"ange $z$ und je zwei horizontale N"ahte der L"ange $x$ und $y$
schweissen. Die Gesamtkosten f"ur die Wanne sind also
\[
f(x,y,z)=c_m\varrho(xy+2z(x+y))+c_s(4z+2x+2y).
\]
Die Aufgabe besteht also darin, die Variablen $x$, $y$ und $z$ so
zu bestimmen, dass $f(x,y,z)$ minimal wird unter den Einschr"ankgungen
\begin{align*}
x&\ge l\\
y&\ge b\\
xyz&\ge V.
\end{align*}

\subsection{Pyramidenbau\label{uebersicht:pyramidenbau}}
Variatos, Vorarbeiter auf der Baustelle der Pyramide des Pharaos
Mathesis\footnote{Name von der Redaktion ge"andert}~V.,hat den Auftrag, mit
einer Equipe von Arbeitern einen Sandstein-Block von $A$ nach $B$
zu bringen. Die Equipe steht ihm ab dem Zeitpunkt $a$ zur Verf"ugung,
und ab Zeitpunkt $b$ soll sie an einer anderen Stelle auf der Baustelle
wieder eingesetzt werden, sie soll dann noch m"oglichst fit f"ur
den Folgeauftrag sein. Die Wahl des Weges von $A$ nach $B$ ist
dem Vorarbeiter freigestellt, er kann eine direkte Route w"ahlen, die
jedoch "uber eine leichte Erhebung f"uhrt, oder er kann die Erhebung
umgehen, was jedoch einen deutlich l"angeren Weg bedeutet. Wie soll
er den Weg planen?

Der Vorarbeiter muss also eine
Kurve $x(t)$ w"ahlen, die den Weg des Blocks durch die Baustelle
beschreibt.
Wenn die Equipe m"oglichst fit sein soll, muss die aufgewendete
Gesamtenergie minimiert werden. 
Letztere kann aus der Momentanen Leistung bestimmt werden, wir 
nehmen an, dass am Punkt $x(t)$ die Leistung $P(x(t), \dot x(t))$
erbracht werden muss, um den Block mit Geschwindigkeit $\dot x(t)$
zu bewegen. Die zu minimierende Gr"osse ist also
\[
\int_a^bF(x(t), \dot x(t))\,dt.
\]
Dieses Integral betrachten wir als eine Gr"osse, die von der Wahl
der Funktion $x(t)$ abh"angt.

\section{Begriffe\label{section-begriffe}}
\rhead{Begriffe}
\subsection{Zul"assiges Gebiet}
Allen Beispielen von Optimierungsproblemen war gemeinsam, dass eine
Anzahl von Variablen $x=(x_1,\dots,x_n)$ variert werden k"onnen.
Im Ingenieur-Zusammenhang spricht man beim Vektor 
\[
\begin{pmatrix}
x_1\\\vdots\\x_n
\end{pmatrix}
\]
oft auch vom {\em Designvektor}, die einzelnen Variablen $x_i$ werden
dann oft {\em Designvariablen} genannt, um sie von anderen im Problem
auftretenden Variablen zu unterscheiden, die zum Beispiel zur Bestimmung
eines Optimums nicht frei ver"andert werden d"urfen.

Dabei sind Einschr"ankungen zu beachten.
Die {\em Einschr"ankungen} k"onnen in der Form von Gleichungen
\begin{equation}
g(x)=0,
\label{begriff-nebenbedingungen}
\end{equation}
auch {\em Nebenbedingungen} genannt, oder Ungleichungen
\begin{equation}
h(x)\le 0
\label{begriff-einschraenkungen}
\end{equation}
vorliegen.
\index{Nebenbedingung}
$g(x)$ und $h(x)$ k"onnen auch vektorwertig sein,
das heisst ist eine Menge von Nebenbedingungen
\begin{align*}
g_i(x)&=0&x&\in\mathbb R, 1\le i\le m
\end{align*}
oder von Einschr"ankungen durch Ungleichungen
\begin{align*}
h_j(x)&\le 0&x&\in\mathbb R, 1\le j\le l
\end{align*}
gegeben.

Dass hier Gleichungen in der Form $g_i(x)=0$ betrachtet werden, also
\marginpar{\raggedright\tiny Rechte Seite $0$ ist keine Einschr"ankung}
nur mit $0$ auf der rechten Seite, ist keine wesentliche Einschr"ankung.
Denn eine Gleichung der Form $g_i(x)=c$ kann in die Standardform gebracht
werden, indem man $c$ auf beiden Seiten der Gleichung subtrahiert,
und die Funktion $g_i(x)$ durch $g_i(x)-c$ ersetzt.

Auch ist es keine Einschr"ankung, dass nur Ungleichungen der Form 
\marginpar{\raggedright\tiny Andere Formen von Ungleichungen sind "aquivalent}
$h_j(x)\le 0$ betrachtet werden. St"unde auf der rechten Seite eine
Konstante $c$, k"onnte man diese wieder loswerden, indem man $h_j(x)$
durch $h_j(x)-c$ ersetzt. Eine Ungleichung der Form $h_j(x)\ge 0$
kann durch multiplizieren mit $-1$ auf die Standardform gebracht werden,
man ersetzt also $h_j(x)$ durch $-h_j(x)$.

Eine Gleichung $g(x)=0$ definiert eine Teilmenge, typischerweise eine
Fl"ache oder Kurve (allgemeine eine Untermannigfaltigkeit) im 
Parameterraum.
Eine Ungleichung andererseits definiert eine berandete Teilmenge,
zum Beispiel ist
\[
\{(x,y,z)\,|\,h(x,y,z)=x^2+y^2+z^2\le 1\}
\]
eine Vollkugel, deren Rand die Kugeloberfl"ache mit der Gleichung
\[
g(x,y,z)=x^2+y^2+z^2=1
\]
ist. Im Allgemeinen wird die Zahl der 

\begin{definition}
Die Menge der Punkte $x\in\mathbb R^n$, die alle Einschr"ankungen
erf"ullen, heisst {\em zul"assiges Gebiet}.
\index{zul\"assiges Gebiet}
\end{definition}

\subsection{Zielfunktion}
Unter den gegebenen Einschr"ankungen sollen jetzt Werte der
Variablen $x_1,\dots,x_n$ gefunden werden, f"ur die eine weitere
Funktion $f\colon \mathbb R^n \to \mathbb R$ m"oglichst gross
oder m"oglichst klein wird.
Die Funktion $f(x_1,\dots,x_n)$ heisst {\em Zielfunktion}.
\index{Zielfunktion}
Offenbar spielt es keine grosse Rolle, ob nach einem Maximum oder einem
\marginpar{\raggedright\tiny Kein mathematischer Unterschied zwischen
Maximum- und Minimum-Problem}
Minimum gefragt wird: Die Maxima von $f(x)$ sind die Minima von $-f(x)$.
Man kann also jedes Optimierungsproblem als Maximum- oder Minimum-Problem
formulieren.

Ein Optimerungsproblem besteht also aus der Vorgabe einer Zielfunktion
$f(x)$ und eines zul"assigen Bereichs $G\subset \mathbb R^n$, welcher
durch Gleichungen $g(x)=0$ oder Ungleichungen $h(x)\le 0$ 
definiert sein kann.
\index{Optimierungsproblem!L\"osung}
Eine L"osung des Optimierungsproblems ist ein Punkt $x^*\in G$ im zul"assigen
Gebiet, so dass $f(x^*)$ maximal oder minimal ist unter allen Werten
der Funktion $f(x)$ mit $x\in G$. Eine solche L"osung heisst
{\it globales} Optimum.
\index{Optimum!global}

Nicht immer kann man ein globales Optimum finden, manchmal gelingt es
nur, ein {\it lokales} Optimum zu finden, also ein Punkt $x^*$ so,
dass $f(x^*)$ maximal oder minimal ist unter allen Werten $f(x)$,
wenn $x$ aus einer Umgebung von $x^*$ in $G$ stammt.
\index{Optimum!lokal}

\subsection{Variationsproblem}
Das Beispiel \ref{uebersicht:pyramidenbau} passt nicht in das bisher
entworfene allgemeine Schema: f"ur jeden
Parameterwert $t$ ist die aktuelle Position des Blocks zu bestimmen,
es sind also unendlich viele Positionen $x(t)$ zu bestimmen, so 
dass die Gesamtenergie, also das Integral der Leistung
\begin{equation}
F(x)=
\int_{a}^{b}P(x(t),\dot x(t))\,dt
\label{uebersicht:funktional}
\end{equation}
minimal wird.
Der Ausdruck (\ref{uebersicht:funktional}) h"angt von der Wahl der
{\em Funktion} $x(t)$ ab (das $x$ in $F(x)$ ist nicht nur eine
Variable), man nennt ihn ein {\em Funktional}.
Die Aufgabe, eine Funktion $x(t)$ zu finden, die das Funktional
(\ref{uebersicht:funktional}) minimiert oder maximiert heisst
ein Variationsproblem.

Man kann sich aber vorstellen, das Interval
\marginpar{\raggedright\tiny Approximation eines Variationsproblems
durch ein Optimierungsproblem}
$[a,b]$ in $n$ Teilintervalle
$[t_0,t_1]$, $[t_1,t_2]$, $\dots$, $[t_{n-1},t_n]$ zu
zerlegen mit $a=t_0$ und $b=t_n$. Statt der Kurve
$x(t)$ ist dann eine endliche Menge von Punktion $x_k=x(t_k)$ 
zu bestimmen. Die Zielfunktion kann approximiert werden
\begin{equation}
\int_{a}^{b}P(x(t),\dot x(t))\,dt
\simeq
\sum_{k=1}^n P\biggl(x_k,\frac{x_k-x_{k-1}}{t_k-t_{k-1}}\biggr)\cdot (t_k-t_{k-1})
=f(x_1,\dots,x_n).
\label{uebersicht:funktional:diskret}
\end{equation}
Das Problem, die unendlich vielen Punkte der optimalen Kurve
zu finden kann also approximiert werden durch eine endliche Menge
von diskreten Punkten.
Es ist dann die Funktion $f$ zu minimieren
unter gewissen Einschr"ankungen an die Punkte $x_k$ (der Stein kann
zum Beispiel nicht mit "Uberlichtgeschwindigkeit bewegt werden).


Umgekehrt k"onnen wir die bisher betrachteten Optimierungsproblem auch
\marginpar{\raggedright\tiny Optimierungsproblem als ``diskretes''
Variationsproblem}
als Vartiationsproblem betrachten. Ein Vektor
\[
\begin{pmatrix}
x_1\\\vdots\\ x_n
\end{pmatrix}
\]
ist ja nichts anderes als eine Funktion
\[
x\colon \{1,\dots n\}\to \mathbb R: k\mapsto x_k,
\]
die ganzzahligen Werten zwischen 1 und $n$ einen Zahlenwert
zuordnen.
Die Funktion $k\mapsto x_k$ ist eine ``diskrete Variante'' einer
Kurve $t\mapsto x(t)$, das Funktional f"ur eine ``diskrete Kurve''
ist einfach nur die Summe (\ref{uebersicht:funktional:diskret}).



\section{Klassifizierung \label{section-klassifizierung}}
Wir gehen in diesem wieder von dem Problem aus, dass die Funktion
$f\colon\mathbb R^n\to\mathbb R$ maximiert werden soll.
Wir setzen im allgemeinen nicht voraus, dass die Funktion differenzierbar
sein muss, doch viele L"osungsverfahren verwenden Ableitungen und 
funktionieren daher nur f"ur differenzierbare Funktionen.

Das Problem, das Extremum der Funktion $f$ zu finden f"ur beliebige
Werte von $x\in\mathbb R^n$ heisst ein {\em nicht eingeschr"anktes
Optimierungsproblem}.


Ein {\em eingeschr"anktes Optimierungsproblem} liegt vor, wenn die
Funktion $f$ unter
Nebenbedingungen 
\begin{align*}
g_i(x)&=0&&x\in\mathbb R^n, 1\le i\le m
\end{align*}
und Einschr"ankungen
\begin{align*}
h_j(x)&=0&&x\in\mathbb R^n, 1\le j\le l
\end{align*}
maximiert werden soll. Beide Beispiele \ref{uebersicht:hackfleisch}
und \ref{uebersicht:wanne} sind eingeschr"ankte Optimierungsprobleme.

Ein {\em lineares Optimierungsproblem} liegt vor, wenn die
Zielfunktion $f$ wie auch die Nebenbedingungen $g_i$ und
Einschr"ankungen $h_j$ lineare Funktionen sind.
F"ur lineare Optimierungsprobleme wird im
Kapitel~\ref{chapter-lineare-optimierung} mit dem Simplexalgorithmus
ein leistungsf"ahiges L"osungsverfahren vorgestellt.
Das Hackfleisch-Beispiel \ref{uebersicht:hackfleisch} ist ein lineares
Optimierungsproblem.

Ist die Zielfunktion oder eine der Einschr"ankungen nicht linear
spricht man von einem {\em nichtlinearen Optimierungsproblem}.
Das Beispiel \ref{uebersicht:wanne} der Tankwanne ist ein nichtlineares
Optimierungsproblem.

Ist die Zielfunktion ein Polynom in mehreren Variablen, dann spricht
man von einem {\em geometrischen Optimierungsproblem}, vor allem
aus praktischen Gr"unden: f"ur diese besonderen F"alle gibt es 
spezielle L"osungsverfahren, die bei beliebigen Zielfunktionen
nicht mehr funktionieren.

In einigen Anwendungsproblemen k"onnen die Designvariablen nicht beliebige
relle Werte annehmen. Zum Beispiel kann ein Zahnrad nur eine ganze Anzahl
von Z"ahnen haben, oder Widerstandswerte sind nur in bestimmten Werten
verf"ugbar, deren Logarithmus ein ganzzahliges Vielfaches einer
Grundeinheit ist. Sind nur ganzzahlige L"osungen f"ur ein Optimierungsproblem
erlaubt, spricht man von einem {\em ganzzahligen Optimierungsproblem}.

H"aufig kommen auch Kombination vor: einige Designvariablen m"ussen
ganzzahlig sein, die anderen k"onnen beliebige reelle Werte annehmen.
Solche Probleme heissen {\em gemischte Optimierungsprobleme}.

Problem, in denen nicht nur endlich viele diskrete Werte so zu bestimmen
sind, dass eine Zielgr"osse optimal wird, sondern eine ganz Funktion,
wie im Pyramidenbau-Beispiel \ref{uebersicht:pyramidenbau} heissen
{\em Variationsprobleme}. Der folgende Speziallfall ist besonders wichtig.
Ein System kann mit Hilfe einer "ausseren Kraft $x(t)$ gesteuert werden.
Das System soll jetzt von Zustand $A$ in Zustand $B$ gebracht werden,
wobei der Energieverbrauch (oder eine andere Gr"osse, zum Beispiel die
daf"ur ben"otigte Zeit) minimiert werden soll. Ein solches Problem heisst
ein {\em Kontroll-Problem}.
