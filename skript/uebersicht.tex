\chapter{"Ubersicht}
Steven Wozniak war daf"ur ber"uhmt, den ersten Apple-Computer mit
einer m"oglichst kleinen Zahl von ICs konstruiert zu haben.
Ausserdem soll der Firmware Code besonders schlank gewesen sein, so
dass im vorhandenen Speicher mehr Funktionen untergebracht werden 
konnten. Hier wurden offenbar zwei Optimierungsprobleme gel"ost.
Beide d"urften zwar nicht ausschlaggebend f"ur den Erfolg von
Apple gewesen sein, aber einen Beitrag dazu geleistet haben.

Vor einer "ahnlichen Herausforderung steht man h"aufig: mit m"oglichst
geringem Aufwand soll ein m"oglichst gutes Resultat erreicht werden.
Typischerweise sind auf der Aufwand-Seite immer mehrere Gr"ossen varierbar.
Bei Embedded Hardware kann man in gewissen Grenzen
fehlende Hardware-Funktionalit"at durch Software kompensieren.
Billigere Rohmaterialen brauchen gr"osseren Bearbeitungsaufwand.
Aber auch auf der Resultat-Seite sind oft verschiedene Gr"ossen zu
betrachten.
Will man mit dem Produkt vor allem Geld verdienen, oder vor allem
das Image f"ordern?
Was ist dem Kunden mehr wert: ein kleines und leichtes
Ger"at, oder ein robustes, nicht kaputt zu kriegendes? 
In vielen F"allen l"asst sich das Ziel auf eine einfache
Zahl reduzieren: Gewinnmaximierung.

Optimierungsmethoden sollen helfen, mathematisch formulierbare
Optimierungsprobleme zu l"osen. Dazu ist zun"achst zu kl"aren,
wie ein mathematisches Optimierungsproblem aussieht.
Das oben erw"ahnte Problem von Steven Wozniak ist wohl eher nicht
von dieser Art. Beispiele von mathematischen Optimierungsproblemen
sollen im Abschnitt \label{section-einfuehrungsbeispiele} einen
ersten Eindruck davon geben, was in diesem Skript behandelt werden soll.
Abschnitt \label{section-begriffe} stellt die im Weiteren verwendeten
Begriffe zusammen. Abschnitt \label{section-klassifizierung} versucht
eine grobe Klassifizierung der Optimierungsprobleme, und in Abschnitt
\label{section-loesungsuebersicht} werden die im Folgenden zu
besprechenden L"osungsmethoden kurz skizziert.

\section{Einf"uhrungsbeispiele\label{section-einfuehrungsbeispiele}}

\section{Begriffe\label{section-begriffe}}
Allen Beispielen von Optimierungsproblemen war gemeinsam, dass eine
Anzahl von Variablen $x=(x_1,\dots,x_n)$ variert werden k"onnen.
Dabei sind Einschr"ankungen zu beachten.
Die Einschr"ankungen k"onnen in der Form von Gleichungen
\begin{equation}
g(x)=0
\label{begriff-nebenbedingungen}
\end{equation}
oder Ungleichungen
\begin{equation}
h(x)\le 0
\label{begriff-einschraenkungen}
\end{equation}
vorliegen, auch Nebenbedingungen genannt.
\index{Nebenbedingung}
$g(x)$ und $h(x)$ k"onnen auch vektorwertig sein.
Eine Gleichung $g(x)$ definiert eine Teilmenge, typischerweise eine
Fl"ache oder Kurve (allgemeine eine Untermannigfaltigkeit) im 
Parameterraum.
Eine Ungleichung andererseits definiert eine berandete Teilmenge,
zum Beispiel ist
\[
\{(x,y,z)\,|\,h(x,y,z)=x^2+y^2+z^2\le 1\}
\]
eine Vollkugel, deren Rand die Kugeloberfl"ache mit der Gleichung
\[
g(x,y,z)=x^2+y^2+z^2=1
\]
ist.
Die Menge der Punkte $x\in\mathbb R^n$, die alle Nebenbedingungen
erf"ullen, heisst {\it zul"assiges Gebiet}.
\index{zul\"assiges Gebiet}
Die zu optimierende Gr"osse ist eine Funktion $f$ der Variablen $x$,
sie heisst die {\it Zielfunktion}.
\index{Zielfunktion}

Ein Optimerungsproblem besteht also aus der Vorgabe einer Zielfunktion
$f(x)$ und eines zul"assigen Bereichs $G\subset \mathbb R^n$, welcher
durch Gleichungen $g(x)=0$ oder Ungleichungen $h(x)\le 0$ 
definiert sein kann.
\index{Optimierungsproblem!L\"osung}
Eine L"osung des Optimierungsproblems ist ein Punkt $x^*\in G$ im zul"assigen
Gebiet, so dass $f(x^*)$ maximal oder minimal ist unter allen Werten
der Funktion $f(x)$ mit $x\in G$. Eine solche L"osung heisst
{\it globales} Optimum.
\index{Optimum!global}

Nicht immer kann man ein globales Optimum finden, manchmal gelingt es
nur, ein {\it lokales} Optimum zu finden, also ein Punkt $x^*$ so,
dass $f(x^*)$ maximal oder minimal ist unter allen Werten $f(x)$,
wenn $x$ aus einer Umgebung von $x^*$ in $G$ stammt.
\index{Optimum!lokal}



\section{Klassifizierung von Optimierungsproblemen\label{section-klassifizierung}}
\subsection{Extremwerte}
Die einfachste Art von Optimierungsproblem ist das Finden der Extremwerte
einer Funktion $f\colon\mathbb R^n\to\mathbb R$.
Im eindimensionalen Fall einer differentzierbaren Funktion lernt man dazu
in der Analysis, dass die Extrema entweder auf dem Rand des Definitionsbereichs
zu finden sind, oder aber innere Punkte sind, in denen die Funktion
verschwindende Ableitung $f'(x)=0$ hat.

Die Zielfunktion ist in diesem Optimierungsproblem $f$, das zul"assige
Gebiet ist der Definitionsbereicht von $f$.

