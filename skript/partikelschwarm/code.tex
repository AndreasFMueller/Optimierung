\section{Code}
Der komplette C++ Code befindet sich im Anhang. In diesem Teil wird der
Code kurz beschrieben.

\subsection{Position-Klasse}
Die Klasse \textit{Position} stellt einen mehrdimensionalen Datentyp
zur Verf"ugung.
Damit ist es m"oglich, Probleme jeder beliebigen Dimension zu l"osen.

\subsection{Particle-Klasse}
In der \textit{Particle}-Klasse werden die Positionen und
Geschwindigkeiten jedes Partikels
als Attribute gespeichert. Weiter werden Funktionen f"ur Geschwindigkeits-
und Positionsupdates
zur Verf"ugung gestellt.

\subsection{Swarm-Klasse}
Die Klasse \textit{Swarm} dient dazu, einen Schwarm von Partikeln zu
erzeugen und zu initialisieren.
Zus"atzlich wird eine Funktion zur Ausf"uhrung der Optimierung bereitgestellt.

\subsection{Random}
Es wurde eine Funktion \textit{getRand} erzeugt, welche Zufallswerte in
einem bestimmten Bereich zur"uckgibt. Diese werden in der Initialisierung
und bei jeder Iteration des Schwarms ben"otigt.
