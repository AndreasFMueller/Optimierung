\section{Einf"uhrung}
\index{Partikelschwarm-Optimierung}
Die Partikelschwarm-Optimierung wurde von James Kennedy und
\index{Kennedy, James}
Russel Eberhart
\index{Eberhart, Russel}
im Jahre 1995 entwickelt. Es handelt sich um einen numerischen
Optimierungsalgorithmus. Er simuliert das Verhalten von nat"urlichen
Schw"armen, um mathematische Problemstellungen zu optimieren.

Es bewegt sich ein Schwarm von Partikeln durch den Suchraum. Jedes
Partikel hat eine Richtung und Geschwindigkeit sowie eine ``Fitness''. Die
\index{Fitness}
Fitness sagt aus, wie gut die momentane Position ist. Partikel mit einer
guten Fitness ziehen andere Partikel an, solche mit schlechter Fitness
stossen andere Partikel ab. So beeinflussen sich die Partikel gegenseitig
und man kann sehr viel parallel berechnen, was vor allem bei Multicore
Computersystemen von Vorteil ist.
Diese Art von Optimierung wurde erst durch die Entwicklung von
leistungsstarken Computersystemen m"oglich. Die st"andig steigende
Rechenleistung erm"oglicht immer umfassendere Simulationen und damit
immer umfassendere Optimierungen.

\subsection{Schwarm}
\index{Schwarm}
In der Natur gibt es viele verschiedene Strukturen und Formen in denen
viele einzelne Bestandteile zusammen wirken. Eine davon ist ein Schwarm.
Wenn man von Schw"armen spricht, denkt man an Fisch- oder an
Vogelschw"arme, die sich in der freien Wildbahn  bilden um z.B. als
Zugvogelschwarm nach S"uden oder Norden zu ziehen, als Fischschwarm
gemeinsam auf Nahrungssuche zu gehen oder sich gegen nat"urlich Feinde zu
verteidigen. Solche Schw"arme k"onnen das Wissen jedes Einzelnen nutzen
und so beispielsweise die besten Futterpl"atze finden.

\subsection{Schwarmintelligenz}
\index{Schwarmintelligenz}
Ein Schwarm kann als Ganzes eine Eigenschaft haben, die aus den
einzelnen Mitgliedern des Schwarms nicht voraussagbar ist. So kann
eine einzelne Nervenzelle keine Informationen speichern. Wenn nun aber
Milliarden von Nervenzellen richtig zusammen zu einem ``Hirn'' verkn"upft sind, kann das Ganze denken wie auch
Informationen speichern. Dies sind offenbar emergente Eigenschaften,
welche nicht durch einzelne Nervenzellen sondern durch eine Vielzahl
von Nervenzellen zustande kommt
\cite{partice-swarm-optimization}.

\subsection{Emergenz}
\index{Emergenz}
Emergente Eigenschaften sind Eigenschaften, die durch das Zusammenspiel
vom vielen einzelnen Teilen, zum Beispiel in einem Schwarm,
entstehen. Diese Eigenschaften lassen sich nicht immer nur aus den
Bestandteilen des Schwarms vorhersagen.
Dass ein Zusammenspiel von Bestandteilen ein Verhalten begr"undet, sagt
noch nicht aus, dass dieses Verhalten automatisch intelligenter ist,
als das Verhalten der Bestandteile selbst. In der Natur hat sich aber
gezeigt, dass dieses emergente Verhalten oft wesentlich intelligenter ist,
als das Verhalten von den einzelnen Bestandteilen.

Die Emergenz ist Grundbestandteil von Algorithmen, welche auf das
Verhalten von Schw"ar\-men zur"uckgreifen \cite{melchior-bsc}.

\subsection{Schwarmalgorithmus}

In den 1980er Jahren begann man das Verhalten von Schw"armen anhand
von Computermodellen zu erforschen. Man wollte etwas "uber Evolution
und deren Mechanismen lernen. Aus diesen Simulationen entstand der
Schwarmalgorithmus.
In diesem Algorithmus wurden die Naturgesetze mit eingebaut, da man das
Verhalten von Tieren in der freien Wildbahn erforschen wollte und diese
auch diesen Naturgesetzen unterliegen.

\begin{itemize}
\item Vermeiden von Kollisionen 
\item Angleichen der Geschwindigkeit
\item Angleichen der Flugrichtung
\end{itemize}

Man stellte fest, dass es in solchen Schw"armen nicht m"oglich war, immer
alle anderen Teilnehmer zu beachten, da die Schw"arme sonst anfangen zu
kreisen. Man fand heraus, das solche Schw"arme in der Natur nicht zentral
gesteuert werden, sondern es vielmehr ein Zusammenspiel vieler Teile ist,
die sich gegenseitig beeinflussen.

Mit der Zeit entdeckte man, dass solch ein Algorithmus auch zur L"osung
von Optimierungsproblemen verwendet werden kann.
Man erkannte auch, dass beim L"osen von mathematischen
Optimierungsproblemen die oben erw"ahnten Naturgesetze keine Rolle mehr
spielen. Denn nun geht es nicht mehr um das Verhalten von Tierschw"armen
in der freien Natur. Bei mathematischen Problemen ist es z.B. nicht
so entscheidend, ob nun 2 Partikel zur selben Zeit am selben Ort sind
(was in der Natur zu einer Kollision mit verletzten Tieren f"uhrt). So
wurde der Schwarmalgorithmus zum Partikelschwarm-Algorithmus entwickelt,
welcher nicht zum Erforschen von Schw"armen in der Natur gedacht ist,
sondern um Optimierungen zu berechnen.

		
\subsubsection{Algorithmen}
Es gibt verschiedene Alogrithmen, welche die Schwarmintelligenz nutzen. Im
Anschluss werden einige kurz vorgestellt. Am weitesten verbreitet
ist der Partikel\-schwarm-Algorithmus sowie der Ameisenalgorithmus,
jedoch gibt es noch weitere Algorithmen, welche ebenfalls erforscht und
weiterentwickelt werden. Auf diese Algorithmen wird im Weitern nicht
eingegangen \cite{uulm}.

\begin{description}
\item[Ant Colony Algorithm:]
\index{Ant colony Algorithm}
Auch Ameisen nutzen die Intelligenz des Ganzen wenn sie Futterpl"atze
suchen. Hat eine Ameise einen Futterplatz gefunden, hinterl"asst sie
auf dem R"uckweg Pheromone, an denen sich andere Ameisen orientieren
und so den Futterplatz und auch den R"uckweg finden k"onnen. Findet
nun eine Ameise einen schnelleren Weg zur"uck, so verfl"uchtigen
sich diese Pheromone langsamer, da es weniger Zeit braucht f"ur einen
\index{Pheromone}
Weg. So werden die andere Ameisen aufmerksam auf diese st"arkere Spur und
schlussendlich nutzen alle diesen schnelleren Weg. Dieses Prinzip wird im
Ameisenalgorithmus (ant colony optimization algorithm) ausgen"utzt.Dieser
Algorithmus wird typischerweise f"ur eine Pfadoptimierung verwendet.

\item [Invasive Weed:]
\index{Invasive Weed}
Der Invasive Weed Algorithmus orientiert sich an der Fortpflanzung
von Pflanzen. Es werden Pflanzen frei "uber das Suchgebiet verteilt
und dann die Fitness bewertet. Je fitter eine Pflanze ist, desto mehr
Pflanzensamen verteilt sie. Auch bei diesen Pflanzensamen wird die
Fitness bewertet. Das Ganze wird solange fortgef"uhrt, bis eine maximale
Anzahl Pflanzen, die zuvor definiert wurde, erreicht ist. Dann wird
die Fitness aller Pflanzen bewertet und nur die Besten werden in einen
neuen Durchgang mitgenommen. Diese Schritte werden solange wiederholt,
bis das Abbruchkriterium erf"ullt ist.
		

\item [Bees Algorithm:]
\index{Bees Algorithm}
Beim Bienenalgorithmus hat man, wie der Name schon sagt, bei den Bienen
abgeschaut.
In einem ersten Schritt werden ``Scouts''
frei "uber das Suchgebiet verteilt und im Anschluss deren Fitness
betrachtet. Bei dem ``Scout'' mit der besten
Fitness werden in der Umgebung erneut ``Bienen''
verteilt und deren Fitness beurteilt. Dieser Schritt wiederholt sich, bis
man ein Ergebnis erreicht hat, das das Abbruchkriterium erf"ullt.Dieser
Algorithmus wird typischerweise f"ur kombinatorische oder funktionale
Optimierung verwendet.

\item[Firefly Algorithm:]
\index{Firefly Algorithm}
Der Firefly Algorithm nutzt das Paarungsverhalten von
Gl"uh\-w"urm\-chen. Gl"uh\-w"urm\-chen werden unisexuel implementiert, damit
\index{Gluwurmchen@Gl\"uhw\"urmchen}
jedes Tier f"ur das andere als Partner in Frage kommt.
Zuerst werden Gl"uh\-w"urm\-chen frei auf dem Suchgebiet verteilt. Danach
wird die Fitness beurteilt. Je fitter ein Gl"uhw"urmchen ist, desto
heller leuchtet es, je heller es leuchtet, desto interessanter ist
ein Gl"uhw"urmchen f"ur die Artgenossen. Das heisst, jedes einzelne
Gl"uhw"urmchen bewegt sich auf das Tier zu, das aus seiner Sicht am
hellsten ist. Je n"aher man sich auf ein Gl"uhw"urmchen zu bewegt, desto
heller leuchtet es. So werden bald alle Gl"uhw"urmchen Richtung Optimum
wandern. W"urden nun alle gleich hell leuchten, sprich dieselbe Fitness
haben, w"urden sich die Tiere zuf"allig im Suchraum bewegen.

\end{description}
