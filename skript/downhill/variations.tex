\section{Variationen}
Es gibt verschiedenste Variationen des Downhill-Simplex Algorithmus.
Einige verwenden zus"atzliche Freiheitsgrade, wie z.B. einen Parameter
$\sigma$, welcher $\beta$ bei Komprimierung ersetzt.

Auch gibt es Erweiterungen welche statistische Auswertung bei verrauschten
Zielfunktionen (siehe \figref{fig:downhillRauschen1}) durchf"uhren.
Verrauschte Zielfunktionen treten unter anderem auf wenn man Parameter
f"ur ein System optimieren will, welche sich nur durch Messung vergleichen
lassen. Obwohl auch der ''normale`` Downhill-Simplex mehr oder weniger
funktioniert (siehe \figref{fig:downhillRauschen2}), hat er M"uhe zu
konvergieren, da er immer auch die Chance hat sich f"alschlicherweise
auszudehnen. Wie man sieht ist viel Potential vorhanden indem der
Einfluss des Rauschens verringert wird.
Informationen dazu unter \footnote{\url{http://www.informs-sim.org/wsc91papers/1991_0126.pdf}}
\begin{figure}[h]
\centering
\includegraphics[width=0.8\textwidth]{downhill/himmelblauoverview.png}
\caption{Verrauschte Zielfunktion}
\label{fig:downhillRauschen1}
\end{figure}

\begin{figure}[h]
\centering
\includegraphics[width=0.8\textwidth]{downhill/himmelblauall.png}
\caption{Beispiel Verlauf Simplex mit verrauschter Zielfunktion}
\label{fig:downhillRauschen2}
\end{figure}
