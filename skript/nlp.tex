\chapter{Nichtlineare Optimierung\label{chapter-nichtlineare-optimierung}}
\lhead{Nichtlineare Optimierung}
Praktische Optimierungsprobleme sind oft nicht linear. 
Nicht nur die Zielfunktion kann nichtlinear sein, auch die
Beschreibung des zul"assigen Gebietes kann nichtlineare Ungleichungen
verwenden. 

Wir gehen in diesem Kapitel in drei Schritten vor:
\begin{enumerate}
\item Im ersten Abschnitt repetieren wir die Theorie der Bestimmung
von Extremwerten von Funktionen.
\item Im zweiten Abschnitt untersuchen wir Extremwerte von Funktionen
von mehreren Variablen in einem zul"assigen Gebiet, welches durch
Gleichungen gegeben wird.
\item Im dritten Abschnitt verallgemeinern wir die Theorie auf
ein Gebiet, welches durch nichtlineare Ungleichungen gegeben ist.
\end{enumerate}
In jedem Schritt werden wir ein L"osungsverfahren kennenlernen,
welches mindestens im Prinzip in der Lage ist, die Optima zu
finden.
Allerdings m"ussen dazu jeweils nichtlineare Gleichungen
gel"ost werden, was oft nur numerisch m"oglich ist.
Daher gibt es auch eine Reihe von numerischen Verfahren, welche die ganze
Theorie umgehen, und direkt optimale L"osungen zu finden versuchen.
Eine Auswahl solcher Verfahren wird im vierten Abschnitt vorgestellt.

\section{Extremwerte von Funktionen\label{nlp:extremwerte}}
\rhead{Extremwerte}
Das einfachste nichtlineare Optimierungsproblem verlangt nur die
Optimierung einer Zielfunktion, zul"assig ist ganz $\mathbb R^n$.
Es geht also darum, Maximum oder Minimum einer Funktion
\[
f\colon \mathbb R^n\to\mathbb R
\]
zu finden. Da der Definitionsbereich unberandet ist, ist ein
Extremum notwendigerweise ein innerer Punkt des Definitionsbereiches.

Beschr"ankungen des Definitionsgebietes f"uhren
zu zus"atzlichen Komplikationen, es muss jetzt unterschieden
werden, ob Extrema auf dem Rand oder doch im Inneren des Definitionsgebietes
liegen.
Methoden, die auf Ableitungen basieren, die mit Grenzwerten
arbeiten, zum Beispiel Ableitungen, d"urfen nur noch einseitige
Grenzwerte verwenden.

\subsection{Funktionen einer Variablen\label{nlp:extremwerte:einevariable}}
Das einfachste nichtlineare Optimierungsproblem ist, das Maximum
oder Minimum einer Funktion $f\colon\mathbb R\to\mathbb R$ zu finden.
A priori ist nicht einmal klar, dass $f$ die verlangten Extrema
besitzt, nicht einmal wenn $f$ eine glatte Funktion ist. So hat
die Funktion
\[
x\mapsto e^{-x^2}
\]
ein Maximum im Punkt $x=0$, aber kein Minimum. Zwar kann jeder beliebig
kleine Werte $\varepsilon>0$ erreicht werden, zum Beispiel mit
$x=\sqrt{-\log \varepsilon}$, der Wert $0$ wird aber nicht angenommen.

Auch lassen sich f"ur das Extremum keine Kriterien angeben, die 
als Basis eines L"osungsverfahrens dienen k"onnten, wenn die Funktion
$f$ nicht minimale Glattheitseigenschaften hat.
Ist $f$ differenzierbar, dann muss in einem Extremum $x_0$
$f'(x_0)=0$ gelten. Nur wenn $f$ zweimal differenzierbar ist,
kann man auch noch entscheiden, ob $x_0$ eine lokale Extremalstelle
ist: $f''(x_0)>0$ zeigt ein lokales Minimum an, $f''(x_0) < 0$
ein lokales Maximum.

Ein nichtlineares Maximumproblem (Minimumproblem) f"ur eine differenzierbare
Funktion $f\colon\mathbb R\to\mathbb R$ wird also im Allgemeinen in
folgenden Schritten gel"ost:
\begin{enumerate}
\item Finde alle Nullstellen $\{x_1,\dots,x_n\}$ der Ableitung $f'(x)$.
\item Falls $f''(x_i)$ existiert, eliminiere alle $x_i$, f"ur die
$f''(x_i) > 0$ ($f''(x_i) < 0$).
gesucht wird.
\item Berechne alle Funktionswerte $y_i=f(x_i)$, und w"ahle den
gr"ossten (den kleinsten) aus.
\end{enumerate}

Ist die Funktion $f$ nicht "uberall differenzierbar, k"onnte sie Extrema
auch an Stellen annehmen, wo $f$ nicht differenzierbar ist. Daher
sind in Schritt~1 des obigen Algorithmus auch noch all jene Punkte
zu ber"ucksichtigen, in denen $f'(x)$ nicht definiert ist.

\begin{beispiel} Man finde das Maximum der Funktion $x\mapsto f(x)=\sin x$.
\bigskip

Die Ableitung der Funktion ist $f'(x)=\cos x$.
Die Zahlen $\frac{\pi}2+k\pi$, $k\in\mathbb N$, 
sind Nullstellen, dies sind die Kandidaten f"ur die Extremstellen.
Um zu entscheiden, ob $\frac{\pi}2+k\pi$ ein Maximum oder Minimum ist, muss
man diezweite Ableitung untersuchen: $f''(x)=-\sin x$. Es gilt
\[
f''\biggl(\frac{\pi}2+k\pi\biggr)
=
-\sin\biggl(\frac{\pi}2+ k\pi\biggr)
=\begin{cases}
1&\qquad \text{$k$ gerade}\\
-1&\qquad\text{$k$ ungerade}
\end{cases}
\]
Folglich sind die Stellen $\frac{\pi}2+2k\pi$ Maxima, und
$\frac{\pi}2+(2k+1)\pi$ Minima, $k\in\mathbb Z$.
\end{beispiel}


\subsection{Funktionen mehrerer Variablen\label{nlp:extremwerte:mehrerevariable}}
\subsubsection{Notwendige Bedingung}
Hat eine Funktion mehrere Variablen
\[
f\colon\mathbb R^n\to\mathbb R:(x_1,\dots,x_n)\mapsto f(x_1,\dots,x_n)
\]
in einem Punkt $x_*=(x_{1*},\dots,x_{n*})$ ein Extremum,
dann hat auch die partielle Funktion
\[
x_i\mapsto f(x_{1*},\dots,x_i,\dots,x_{n*})
\]
ein Extremum, also muss ihre Ableitung in diesem Punkt
verschwinden. Die Ableitung der partiellen Funktion ist aber
die partielle Ableitung nach $x_i$, mithin m"ussen
alle partiellen Ableitungen von $f$ im Punkt $x_*$ verschwinden
\begin{equation}
\frac{\partial f}{\partial x_i}(x_{1*},\dots,x_{n*})=0\;\forall i,
\qquad \Leftrightarrow\qquad
\operatorname{grad}f=\nabla f=0.
\end{equation}

\subsubsection{Zweite Ableitungen}
Ob in einem Punkt $x_*$ mit $\operatorname{grad}f(x_*)=0$ ein
Minimum oder Maximum vorliegt, l"asst sich aus der Taylor-Entwicklung
entscheiden. In einer gen"ugend kleinen Umgebung von $x_*$ gilt
\begin{align*}
f(x)&=f(x_*)+\operatorname{grad}f(x_*)\cdot (x-x_*)+
\sum_{i,k=1}^n\frac{\partial^2 f(x_*)}{\partial x_i\,\partial x_k}(x_i-x_{i*})(x_k-x_{k*})+o((x-x_*)^2)
\\
&=f(x_*)+
\sum_{i,k=1}^n\frac{\partial^2 f(x_*)}{\partial x_i\,\partial x_k}(x_i-x_{i*})(x_k-x_{k*})
+o((x-x_*)^2).
\end{align*}
Ein lokales Minimum liegt vor, wenn die Summe in einer Umgebung
von $x_*$ nur positive Werte annimmt, ein lokales Maximum liegt vor,
wenn sie nur negative Werte annimmt.
Die Summe kann in Matrixform etwas "ubersichtlicher geschrieben werden.
Die Matrix $H$
\begin{equation}
H=D^2f(x_*) \quad\text{mit Matrixelementen}\quad
h_{ik}=
\frac{\partial^2 f(x_*)}{\partial x_i\,\partial x_k},
\end{equation}
heisst {\it Hessesche Matrix} von $f$.
\index{Hessesche Matrix}
Schreiben wir den Vektor $\xi = (x-x_*)$, dann ist die Summe
\[
\sum_{i,k=1}^n\frac{\partial^2 f(x_*)}{\partial x_i\,\partial x_k}(x_i-x_{i*})(x_k-x_{k*})
=\xi^tH\xi.
\]
Ein Minimum liegt also vor,
wenn $\xi^t H\xi > 0$ gilt f"ur jeden Vektor $\xi\ne 0$.
\index{positiv definit}
Eine Matrix mit dieser Eigenschaft heisst {\it positiv definit}.

Ein lokales Maximum liegt vor, wenn $\xi^tH\xi<0$ f"ur $\xi\ne 0$,
in diesem Fall heisst $H$ {\it negativ definit}.
\index{negativ definit}

\begin{beispiel}
Die Funktion $f(x,y)=\frac12x^2+\frac12y^2$ hat an der Stelle $(0,0)$
die hessesche Matrix
\[
A=\begin{pmatrix}
1&0\\0&1
\end{pmatrix},
\]
Sie ist positiv definit, denn
\[
\xi^t A\xi=\xi_1^2+\xi_2^2>0 \; \forall \xi\ne 0.
\]
$f$ hat an der Stelle $(0,0)$ ein Minimum.
\end{beispiel}

\begin{beispiel}
Die Funktion $g(x)=-\frac12x^2-y^2$ hat an der Stelle $(0,0)$
die Hessesche Matrix
\[
B=\begin{pmatrix}
-1&0\\0&-2
\end{pmatrix}.
\]
Sie ist negativ definit, denn
\[
\xi^tB\xi=-\xi_1^2-2\xi_2^2<0\;\forall \xi\ne 0.
\]
$g$ hat an der Stelle $(0,0)$ ein Maximum.
\end{beispiel}

\begin{beispiel}
Die Funktion $h(x,y)=xy$ hat den Gradienten
\[
\operatorname{grad}h=\begin{pmatrix}
y\\x
\end{pmatrix}
\]
hat eine einzige Nullstelle: $(0,0)$. Die hessesche Matrix an dieser
Stelle ist
\[
H=\begin{pmatrix}0&1\\1&0\end{pmatrix}.
\]
Sie ist indefinit, denn
\[
\xi^tH\xi=2\xi_1\xi_2\;
\begin{cases}
>0&\qquad \text{$\xi_1$ und $\xi_2$ haben gleiches Vorzeichen}\\
<0&\qquad \text{$\xi_1$ und $\xi_2$ haben verschiedenes Vorzeichen}
\end{cases}
\]
\end{beispiel}

Das letzte Beispiel zeigt, dass eine Matrix weder positiv noch negativ
definit zu sein braucht.

\subsubsection{Kriterium f"ur positiv definite Matrizen}
Wie entscheidet man, ob die hessesche Matrix positiv definit,
negativ definit oder indefinit ist? Der folgende Satz gibt eine 
Anwort.

\begin{satz}Eine symmetrische Matrix $A$ ist genau dann positiv
definit, wenn alle Eigenwerte von $A$ positiv sind. Sie ist negativ
definit, wenn alle Eigenwerte negativ sind. Sie ist indefinit,
wenn es Eigenwerte mit verschiedenen Vorzeichen gibt.
\end{satz}

\begin{proof}[Beweis]
Symmetrische Matrizen sind diagonalisierbar.
Man w"ahlt also ein Koordinatensystem, indem die Matrix diagonal wird:
\[
A'=\begin{pmatrix}
\lambda_1&         &      &         \\
         &\lambda_2&      &         \\
         &         &\ddots&         \\
         &         &      &\lambda_n
\end{pmatrix}
\]
Dann ist
\[
\xi^tA'\xi =\lambda_1\xi_1^2+\lambda_2\xi_2^2+\dots+\lambda_n\xi_n^2.
\]
Dieser Ausdruck ist genau dann positiv definit, wenn alle $\lambda_i$ positiv
sind, und negativ definit genau dann, wenn alle $\lambda_i$
negativ sind.
\end{proof}

F"ur $2\times 2$-Matrizen muss man die Eigenwerte nicht ausrechnen
um entscheiden zu k"onnen, ob sie positiv oder negativ definit sind.
Es reicht, Determinante und Spur zu bestimmen
\[
\det A = \lambda_1\lambda_2,\qquad \operatorname{tr} A=\lambda_1+\lambda_2.
\]
Ist $\det A >0$ ist bereits klar, dass $A$ definit ist. Ist das Vorzeichen
der Spur positiv, ist $A$ positiv definit.

F"ur stetig differenzierbare Funktion ist die Hessesche Matrix
symmetrisch, und ist daher diagonalisierbar.
Als einfaches Kriterium kann daher
verwendet werden, dass ein lokales Minimum genau dann vorliegt,
wenn alle Eigenwerte der Hesseschen Matrix positiv sind.

\subsubsection{Algorithmus f"ur Extremwerte}
Wir haben also folgenden Algorithmus zur Bestimmung der Extrema:
\begin{enumerate}
\item
Finde alle Nullstellen des Gradienten
$\operatorname{grad}f(x)$.
Dazu ist eine System von $n$ nichtlinearen Ungleichungen zu l"osen.
\item
Berechne die Hessesche Matrix
\[
\frac{\partial^2 f(x)}{\partial x_i\,\partial x_j}
\]
f"ur jede Nullstelle $x$ des Gradienten.
Eliminiere alle Nullstellen
des Gradienten, f"ur die Hessesche Matrix indefinit ist.
Falls Maxima (Minima) gesucht sind, elimiere auch alle Nullstellen
des Gradienten, f"ur die die Hessesche positiv (negativ) definit sind.
Damit sind die lokalen Maxima (Minima) gefunden.
\item 
Zur Bestimmung der absoluten Maxima (Minima), berechne die Funktionswerte
$f(x)$ f"ur jede verbleibende Nullstelle des Gradienten und w"ahle jene
mit dem gr"ossen (kleinsten) Funktionswert.
\end{enumerate}

\begin{beispiel}
Man finde die Extrema der Funktion $(x,y)\mapsto \cos x\cos y$.
\bigskip

{\parindent 0pt
Kandidaten f"ur die Extrema sind die Nullstellen des Gradienten, also
des Vektors}
\[
\operatorname{grad}f = \begin{pmatrix}
-\sin x\cos y\\
-\cos x\sin y
\end{pmatrix}.
\]
Die $x$-Komponente verschwindet, wenn 
\begin{align*}
x&=k\pi,\; k\in\mathbb Z & &\text{oder} & y&=\frac{\pi}2+k\pi,\;k\in\mathbb Z.
\end{align*}
Die $y$-Komponente verschwindet, wenn
\begin{align*}
x&=\frac{\pi}2+k\pi,\;k\in\mathbb Z
& &\text{oder} &
y&=k\pi,\; k\in\mathbb Z.
\end{align*}
Wenn beide Komponenten verschwinden sollen, dann kann das entweder
dadurch geschehen, dass in der ersten Komponente der $\sin x$ 
verschwindet, dann kann aber $\cos x$ nicht verschwinden, es muss
also in der zweiten Komponente $\sin y$ verschwinden.
Oder es kann dadurch geschehen, dass in der erste Komponente
der $\cos y$ verschwindet, dann kann aber $\sin y$ nicht verschwinden,
und es muss in der zweiten Komponente $\cos x$ verschwinden.
Es gibt also zwei Familien von Nullstellen von $f$:
\begin{align}
x&=k\pi,\;k\in\mathbb Z
& &\text{und} &
y&=l\pi,\;l\in\mathbb Z
\label{nlp:eierkarton:1}
\\
x&=\frac{\pi}2+k\pi,\;k\in\mathbb Z
&&\text{und}&
y&=\frac{\pi}2+l\pi,\;l\in\mathbb Z.
\label{nlp:eierkarton:2}
\end{align}
Um zu entscheiden, ob ein Maximum, ein Minimum oder ein Sattelpunkt vorliegt,
muss die Matrix der zweiten Ableitungen ausgerechnet werden:
\begin{equation}
h_{ij}=
\biggl(
\frac{\partial^2 f}{\partial x_i\,\partial x_j}
\biggr)
=
\begin{pmatrix}
-\cos x\cos y &  \sin x \sin y\\
 \sin x\sin y & -\cos x \cos y
\end{pmatrix}
\end{equation}
Setzt man alle vier M"oglichkeiten
(\ref{nlp:eierkarton:1})--(\ref{nlp:eierkarton:2}) ein,
findet man folgende Hesseschen Matrizen:
\begin{align*}
(\ref{nlp:eierkarton:1})
&\Rightarrow
&
H
&=
\begin{pmatrix}
-(-1)^{k+l}&0\\
0&-(-1)^{k+l}
\end{pmatrix},
&
\det H&=1,&
\operatorname{tr} H&= -2\cdot (-1)^{k+l},
\\
(\ref{nlp:eierkarton:2})
&\Rightarrow
&
H&=\begin{pmatrix}
0&(-1)^{k+l}\\
(-1)^{k+l}&0
\end{pmatrix},
&
\det H&=-1,
&
\operatorname{tr}H&=0
\end{align*}
Im Fall (\ref{nlp:eierkarton:1}) hat man eine definite Matrix,
postiiv definit falls $k+l$ ungerade ist. An diesen Stellen hat man
also Extrema, und zwar Maxima f"ur gerades $k+l$ und Minima f"ur ungerades
$k+l$.

Im Fall (\ref{nlp:eierkarton:2}) hat man eine indefinite Matrix,
an diesen Stellen findet man also Sattelpunkte.

\begin{figure}
\begin{center}
\includegraphics{images/nlp-2.pdf}
\end{center}
\caption{Extrema (Maxima rot und Minima blau) und Sattelpunkte (scharz) der
``Eierkarton''-Funktion $f(x,y)=\cos x\cos y$.
\label{nlp:eggcrate}}
\end{figure}
In Abbildung~\ref{nlp:eggcrate} sind die Maxima rot, die Minima blau
und die Sattelpunkte schwarz eingezeichnet.
\end{beispiel}

\section{Extremwerte mit Nebenbedingungen}
\rhead{Extremwerte mit Nebenbedingungen}
In diesem Abschnitt sollen Extremstellen einer Funktion $f(x)$ 
unter zus"atzlichen Bedingungen $g_1(x)=0$, $g_2(x)=0$, $\dots$
gefunden werden.
\subsection{Funktionen einer Variablen\label{nlp:nebenbedingungen:einevariable}}
Nebenbedingungen schr"anken den Definitionsbereich weiter ein.
Bei einer Funktion von nur einer Variablen wird der Definitionsbereich
also auf einzelne Punkte eingeschr"ankt. Exrema m"ussend durch ausrechnen
der Funktionswerte und Vergleichen bestimmt werden. 
Der Fall einer Variablen ist also nicht interessant.

\subsection{Funktionen mehrerer Variablen\label{nlp:nebenbedingungen:mehrereariable}}
In diesem Abschnitt
sollen Gebiete betrachtet werden, die durch
eine Menge von m"oglicherweise nichtlinearen Gleichungen
\begin{equation}
g_i(x) = g_i(x_1,\dots,x_n) = 0, \quad 1\le i\le m,
\end{equation}
auch Nebenbedingungen genannt,
beschrieben sind. Man kann die $g_i$ auch als Komponenten einer
vektorwertigen Funktion
\begin{equation}
g\colon \mathbb R^n\to \mathbb R^m:(x_1,\dots,x_n)\mapsto (g_1(x),\dots, g_m(x)).
\end{equation}
auffassen.

Ein Minimum $x_*$ von $f$ muss nicht mehr die Bedingung
$\operatorname{grad}f(x_*)=0$ erf"ullen. Diese Bedingung gibt ja
wieder, dass jede Ver"anderung von $x_*$ einen gr"osseren Wert
liefern wird. Dies muss jetzt nur noch erf"ullt sein f"ur
Ver"anderungen, die tangential an die durch die Gleichungen
$g_i(x)=0$ definierten Fl"achen erfolgen (Abbildung~\ref{nlp:lagrange}). 

\begin{figure}
\begin{center}
\includegraphics{images/nlp-3.pdf}
\end{center}
\caption{Notwendige Bedingung f"ur ein Extremum der Funktion $f(x)$ unter 
der Nebenbedingung $g(x)=0$. Der Gradient $\operatorname{grad}f(x_*)$
muss im Raum aufgespannt von den Gradienten $\operatorname{grad}g_i(x_*)$
liegen.\label{nlp:lagrange}}
\end{figure}
Ein Vektor $\xi$ ist genau dann tangential an alle Fl"achen $g_i(x)=0$
im Punkt $x_*$,
wenn er auf dem Normalenvektor $\operatorname{grad}g_i(x_*)$
senkrecht steht, wenn also
\begin{equation}
\operatorname{grad}g_i(x_*)\cdot \xi=0.
\end{equation}
Die Richtungsableitungen von $f$ in Richtung $\xi$ muss f"ur jedes solche
$\xi$ verschwinden, also
\begin{equation}
\operatorname{grad}f(x_*)\cdot\xi = 0.
\end{equation}
Der Vektor $\operatorname{grad}f(x_*)$ muss also im Vektorraum
aufgespannt von den Vektoren $\operatorname{grad}g_i(x_*)$ liegen,
es muss also Zahlen $\lambda_i\in\mathbb R$ geben mit
\begin{equation}
\operatorname{grad}f(x_*)=\sum_{i=1}^m\lambda_i \operatorname{grad}g_i(x_*).
\end{equation}
Das Optimierungsproblem 
\begin{align*}
\text{Maximiere}\quad f(x)&=0\\
\text{unter Nebenbedingungen}\quad g_i(x)&=0\;\forall 1\le i\le m
\end{align*}
kann daher mit folgendem Algorithmus gel"ost werden:
\begin{enumerate}
\item Finde L"osungen $x$, $\lambda_i$ des Gleichungssystems
\begin{align*}
g_i(x)&=0\quad\forall 1\le i\le m\\
\operatorname{grad}f(x)-\sum_{i=1}^m\lambda_i\operatorname{grad}g_i(x)&=0
\end{align*}
Dies ist ein nichtlineares Gleichungssystem mit $m+n$ Gleichungen 
f"ur die $m+n$ Unbekannten $x_1,\dots,x_n,\lambda_1,\dots,\lambda_m$.
\item 
Test durch Auswerten von $f(x)$ f"ur alle im ersten Schritt gefundenen
Kandidaten, welche zu einem Maximum f"uhren.
\end{enumerate}
Dieser Algorithmus ist bekannt als die
{\em Methode der Lagrange-Multiplikatoren}.
\index{Lagrange-Multiplikator}

\begin{beispiel}
Man finde das Maximum von $f(x,y)=x^2 + xy + y^2$ unter der Nebenbedingung
$g(x,y)=x^2+2y^2-2=0$.

Die Methode der Lagrange-Multiplikatoren besagt, dass man das Gleichungssystem
\begin{align*}
\operatorname{grad}f(x,y)&=\lambda\operatorname{grad}g(x,y)\\
g(x,y)&=0
\end{align*}
l"osen muss. Berechnen wir den Gradienten, erhalten wir die
Gleichungen
\begin{align*}
2x+y &=2\lambda x& y&=2(\lambda-1)x\\
x+2y&=4\lambda y& x&=2(2\lambda -1)y\\
x^2+2y^2&=2
\end{align*}
Die zwei Gleichungen auf der rechten Seite ergeben
\begin{align*}
y=2(\lambda-1)x&=4(\lambda-1)(2\lambda-1)y.
\\
4(\lambda-1)(2\lambda-1)&=1
\\
8\lambda^2-12\lambda+3&=0
\end{align*}
Damit lassen sich die m"oglichen $\lambda$ bestimmen:
\[
\lambda_{\pm}=\frac{12\pm\sqrt{144-4\cdot 8\cdot 3}}{16}
=\frac{12\pm\sqrt{48}}{16}
=\frac{3\pm\sqrt{3}}{4}.
\]
und daraus dann auch $x$
\begin{align*}
x^2(1+8(\lambda-1)^2)&=2
\\
x&=\pm\sqrt{
\frac{2}{1+8(\lambda-1)^2}}
\end{align*}
und $y$:
\begin{align*}
(4(2\lambda-1)^2+2)y^2&=2\\
y&=\pm\sqrt{\frac{2}{4(2\lambda-1)^2+2}}.
\end{align*}
\end{beispiel}

\subsection{Maximum-Kriterium\label{nlp:nebenbedingungen:maximumkriterium}}
\index{Maximum-Kriterium}
Die Matrix der zweiten Ableitungen erlaubt festzustellen, ob eine
Nullstelle des Gradienten ein lokales Maximum oder Minimum ist.
Dazu musste festgestellt werden, ob $\xi^tH\xi$ f"ur alle Vektoren
$\xi\ne 0$ negativ oder positiv ist. 
Unter Nebenbedingungen sind nat"urlich nur noch Vektoren $\xi\ne 0$
zu testen, welche Tangential an die durch die Nebenbedingungen
definierte Menge sind.

Seien die Vektoren $b_1,\dots,b_m$ eine Basis des Raumes der Vektoren,
die auf allen Gradienten $\operatorname{grad}g_i(x)$ senkrecht stehen.
Zusammen bilden sie eine Matrix 
\[
B=\begin{pmatrix}
b_{11}&b_{12}&\dots &b_{1m}\\
b_{21}&b_{22}&\dots &b_{2m}\\
\vdots&\vdots&\ddots&\vdots\\
b_{n1}&b_{n2}&\dots &b_{nm}
\end{pmatrix}.
\]
Weil die Spalten von $B$ eine Basis bilden,
kann man Multiplikation mit einem $m$-dimensionalen Vektor $\eta$
jeden beliebigen Tangentialvektor $\xi=B\eta$ erzeugen. Statt also
das Vorzeichen von $\xi^tH\xi$ zu testen, kann man
\[
\xi^tH\xi=
(B\eta)^tHB\eta=\eta^tB^tHB\eta
\]
testen. Man muss also nur herausfinden, ob die Matrix $B^tHB$
negativ oder positiv definit ist.

\begin{satz}
\label{nlp:satz:nebenbedingungen-maximumkriterium}
Ist $B$ eine Matrix, deren Spalten eine Basis des Vektorraumes
\[
\{
\operatorname{grad}g_i(x)\,|\,1\le i\le m
\}^{\perp}
=
\{\xi\in\mathbb R^n\,|\, \xi\cdot\operatorname{grad}g_i(x)=0\,\forall 1\le i\le m
\}
\]
bilden, dann hat $f$ im Punkt $x$ ein Maximum (Minimum) wenn die Matrix
\[
\tilde H=B^tHB\quad\text{mit}\quad h_{ij}=\frac{\partial^2f(x)}{\partial x_i\,\partial x_j}
\]
negativ (positiv) definit ist.
\end{satz}

\begin{beispiel}
Unter der Nebenbedingung $g(x)=x^2+y^2+z^2-1=0$ hat die Funktion
$f(x,y,z)=xyz$ im Punkt $p=\frac1{\sqrt{3}}(1,1,1)$ einen Kandidaten
f"ur eine Extremstelle, denn dort ist
\begin{align*}
\operatorname{grad}f(p)
&=\begin{pmatrix}yz\\xz\\xy\end{pmatrix}
=\frac13\begin{pmatrix}1\\1\\1\end{pmatrix}
\\
\operatorname{grad}g(p)
&=\begin{pmatrix}2x\\2y\\2z\end{pmatrix}
=\frac2{\sqrt{3}}\begin{pmatrix}1\\1\\1\end{pmatrix}
\\
\Rightarrow\qquad
\operatorname{grad}f(p)
&=\lambda\operatorname{grad}g(p),\text{ mit }\lambda=\frac1{2\sqrt{3}}.
\end{align*}
F"ur die Entscheidung, ob dies ein Maximum oder Minimum ist, ist gem"ass
Satz~\ref{nlp:satz:nebenbedingungen-maximumkriterium}
eine Basis der Vektoren zu finden, die auf dem Gradienten
$\operatorname{grad}g(p)$ senkrecht stehen. Die Vektoren
\[
b_1=\begin{pmatrix} 1\\-1\\0\end{pmatrix},
\quad
b_2=\begin{pmatrix}0\\1\\-1\end{pmatrix}
\]
stehen beide senkrecht auf dem Gradienten. F"ur die Matrix $B$ kann daher
verwendet werden:
\[
B=
\begin{pmatrix}
 1& 0\\
-1& 1\\
 0&-1
\end{pmatrix}.
\]
Die Matrix der zweiten Ableitungen von $f$ ist
\[
H
=
\biggl(\frac{\partial^2f}{\partial x_i\,\partial x_j}\biggr)_{ij}
=
\begin{pmatrix}
0&z&y\\
z&0&x\\
y&x&0
\end{pmatrix}.
\]
Damit kann man f"ur die Stelle $p$ jetzt die Matrix $\tilde H$ berechnen:
\begin{align*}
\tilde H&=B^tHB
=
\begin{pmatrix}
1&-1& 0\\
0& 1&-1
\end{pmatrix}
\frac1{\sqrt{3}}
\begin{pmatrix}
0&1&1\\
1&0&1\\
1&1&0
\end{pmatrix}
\begin{pmatrix}
 1& 0\\
-1& 1\\
 0&-1
\end{pmatrix}
\\
&
=
\frac1{\sqrt{3}}
\begin{pmatrix}
-1& 1& 0\\
 0&-1& 1
\end{pmatrix}
\begin{pmatrix}
 1& 0\\
-1& 1\\
 0&-1
\end{pmatrix}
=
\frac1{\sqrt{3}}
\begin{pmatrix}
-2& 1\\
 1&-2
\end{pmatrix}.
\end{align*}
Da dies eine $2\times 2$-Matrix ist, kann die Entscheidung, ob die
Matrix positiv oder negativ definit ist, mit Spur und Determinante
erfolgen:
\begin{align*}
\det\tilde H&=\frac13((-2)^2-1)=1>0&&\Rightarrow&&\text{$\tilde H$ definit}\\
\operatorname{tr}\tilde H&=\frac{-4}{\sqrt{3}}<0&&\Rightarrow&&\text{$\tilde H$ negativ definit.}
\end{align*}
Wir schliessen, dass der Punkt $p$ ein Maximum ist.
\end{beispiel}

\section{Nichtlineare Ungleichungen}
\rhead{Nichtlineare Ungleichungen}
Dass zul"assige Gebiet eines
nichtlinearen Optimierungsproblems kann ausserdem durch 
nichtlineare Ungleichungen der Form
\begin{equation}
h_k(x) = h_k(x_1,\dots,x_n) \le 0,\quad 1 \le k\le l
\end{equation}
eingeschr"ankt werden.
Extrema von $f$ k"onnen sowohl im Inneren des zul"assigen Gebietes
wie auch auf dem Rand liegen.
Dabei k"onnen einige der Ungleichungen auch als Gleichungen erf"ullt
sein. Der Rand des zul"assigen Gebietes
wird also in Kurven, Fl"achenst"ucke und Punkte aufgeteilt, je nachdem
wie viele der Ungleichungen als Gleichungen erf"ullt sind.

\subsection{Funktionen einer Variablen}
Der zul"assige Bereich eines eindimensionalen Optimierungsproblems
ist immer ein Interval, schreiben wir $f\colon[a,b]\to\mathbb R$.
Extrema
liegen entweder im Inneren eines Intervals, und sind dort mit
Hilfe der Ableitung zu lokalisieren, oder sie sind in den
Endpunkten des Intervals zu finden.
\begin{figure}
\begin{center}
\includegraphics{images/nlp-4.pdf}
\end{center}
\caption{Funktion einer Variablen $f(x)$ auf dem Interval $[a,b]$ mit
lokalen Maxima in  $P_0$, $P_2$ und $P_4$, sowie lokalen Minima in
$P_1$ und $P_3$. Am Rand des Intervals entscheidet das Vorzeichen der
erste Ableitung dar"uber, ob ein lokales Maximum oder Minimum 
vorliegt.\label{nlp:funktion1var}}
\end{figure}

W"ahrend bei einem inneren Punkt $x_0$ mit $f'(x_0)=0$
das Vorzeichen der zweiten Ableitung dar"uber entscheidet,
ob ein Maximum oder Minimum vorliegt, reicht dazu f"ur
Randpunkte die erste Ableitung (Abbildung~\ref{nlp:funktion1var}).
Der linke Randpunkt $a$ ist nur dann ein lokales Minimum, wenn $f'(a)\ge 0$
ist.
Ebenso ist der rechte Randpunkt nur dann ein lokales Minimum wenn
$f'(a)\le 0$ ist.

Als Vorbereitung und Motivation der Karush-Kuhn-Tucker-Bedingungen
(\ref{nlp:karush-kuhn-tucker-gleichung}) des $n$-dimensionalen Falles, wollen
wir diese Beobachtung noch auf eine andere Weise formulieren.
Das Interval $[a,b]$ wird durch die zwei linearen Ungleichungen
\begin{equation}
\begin{aligned}
h_1(x)&=a-x\le 0\\
h_2(x)&=x-b\le 0
\end{aligned}
\label{nebenbedingungen-eindimensional}
\end{equation}
beschrieben.
F"ur ein lokales Minimum in $x$ haben wir folgende Bedingungen
abgeleitet:
\begin{equation}
f'(x)\quad
\begin{cases}
\le 0&\qquad\text{f"ur $x=a$}\\
=0&\qquad a<x<b\\
\ge 0&\qquad\text{f"ur $x=b$}
\end{cases}
\label{bedingungen-fuer-minimum}
\end{equation}
In den Randpunkten haben die Nebenbedingungen die Ableitung $h_1'(a)=-1$
und $h_2'(b)=1$.
Die Ableitung $f'(x)$ hat also in einem Ranbdpunkt entgegengesetzte
Vorzeichen wie die Ableitung der Funktion, die den Randpunkt definiert.
Es gibt also in jedem Fall Zahlen $\mu_1\le 0$, $\mu_2\le $,
die die Gleichung
\begin{equation}
f'(x)-\mu_1h_1'(x)-\mu_2h_2'(x)=0
\label{kuhn-tucker-eindimensional}
\end{equation}
erf"ullen.
Je nach Fall in (\ref{bedingungen-fuer-minimum}) k"onnen einzelne
der $\mu_i$ von $0$ verschieden sein, gem"ass folgender Tabelle
\begin{center}
\begin{tabular}{|l|>{$}c<{$}|>{$}c<{$}|}
\hline
Fall&\mu_1&\mu_2\\
\hline
Minimum am linken Rand&\ge 0&=0\\
Extremum im Inneren&=0&=0\\
Minimum am rechten Rand&=0&\ge 0\\
\hline
\end{tabular}
\end{center}
F"ur jedes $i$ muss also mindestens eine der Gr"ossen $\mu_i$
und $h_i(x)$ verschwinden, was man auch durch die
Bedingungen
\begin{equation}
\begin{aligned}
\mu_1h_1(x)&=0\\
\mu_2h_2(x)&=0
\end{aligned}
\label{kuhn-tucker-eindimensional-slack}
\end{equation}
ausdr\"ucken kann.

Mit den Bedingungen (\ref{kuhn-tucker-eindimensional}) und
(\ref{kuhn-tucker-eindimensional-slack})
kann man also das Minimalproblem
durch Einf"uhrung zweier neuer Variablen $\mu_1$ und $\mu_2$
darauf zur"uckf"uhren, ein nichtlineares Gleichungssystem mit
drei Unbekannten $x$, $\mu_1\ge 0$ und $\mu_2\ge0$ und den
Gleichungen
\begin{equation}
\begin{aligned}
f'(x)-\mu_1h_1'(x)-\mu_2h_2'(x)&=0\\
\mu_1h_1(x)&= 0\\
\mu_2h_2(x)&= 0
\end{aligned}
\label{karush-kuhn-tucker-eindimensional}
\end{equation}
zu l"osen.
Die Gleichungen (\ref{karush-kuhn-tucker-eindimensional}) heissen
die {\em Karush-Kuhn-Tucker-Bedingungen} oder {\em Kuhn-Tucker-Bedingungen}.
Weiterhin erf"ullt sein m"ussen nat"urlich auch die
Nebenbedingungen (\ref{nebenbedingungen-eindimensional}).

So wie die Ableitung das Problem, ein Minimum zu finden, auf das
Problem reduziert, eine Gleichung zu l"osen, so reduziert die
eindimensionale Kuhn-Tucker-Bedingung (\ref{kuhn-tucker-eindimensional})
das Problem, eine Minimum
mit Nebenbedinungen zu finden, auf das Problem, eine L"osung der
Gleichungen (\ref{karush-kuhn-tucker-eindimensional}) zu finden.

\begin{beispiel}
Man finde ein Minimum der Funktion $f(x)=x^3-3x$ im Interval $[-2,2]$.

\medskip
{\parindent 0pt
Die Randbedingungen sind wieder (\ref{nebenbedingungen-eindimensional}).}
Das Gleichungssystem (\ref{karush-kuhn-tucker-eindimensional}) wird
zu
\begin{align*}
3x^2-3+\mu_1-\mu_2&=0\\
\mu_1(-2-x)&= 0\\
\mu_2(x-2)&= 0
\end{align*}
Aus der ersten Gleichung kann man ableiten:
\[
x=\pm
\sqrt{1-\frac{\mu_1}3+\frac{\mu_2}3}.
\]
F"ur $\mu_1=\mu_2=0$ findet man daraus die L"osungskandidaten
$x=\pm 1$. Da die zweite Ableitung $f''(x)=6x$ ist, gilt $f''(1)=6\ge 0$
und $f''(-1)=-6<0$, nur $x=1$ ist also ein lokales Minimum. Dies sind die
einzigen inneren lokalen Minima.

Ist $\mu_1\ne 0$, muss zus"atzlich aus der zweiten
Gleichung $x=-2$ gelten, also
\begin{align*}
-2&=-\sqrt{1+\frac{\mu_2}3},\\
4&=1+\frac{\mu_2}3,\\
\mu_2&=9.
\end{align*}
F"ur $\mu_2=0$  folgt analog $x=2$ und
\begin{align*}
2&=\sqrt{1-\frac{\mu_1}3},\\
4&=1-\frac{\mu_1}3,\\
\mu_1&=-9.
\end{align*}
Da aber $\mu_1\le 0$ sein muss, ist $x=2$ kein lokales Minimum.
Es bleibt also nur noch, die beiden Kandidaten $x=-2$
mit $f(-2)=(-2)^3-3\cdot(-2)=-8+6=-2$
und $x=1$ mit $f(1)=-2$
zu untersuchen, wir haben also zwei Minima im Interval $[-2,2]$.
Man beachte, das das Minimum $x=-2$ von der Ableitung alleine nicht
gefunden wird.
\end{beispiel}

\subsection{Karush-Kuhn-Tucker-Bedingung f"ur mehrere Variable}
\begin{figure}
\begin{center}
\includegraphics{images/nlp-5.pdf}
\end{center}
\caption{M"ogliche Lagen lokaler Extrema auf dem Rand eines durch
nichtlineare
Ungleichungen beschriebenen Gebietes.\label{nlp:kkt-moegliche-extrema}}
\end{figure}
Im Gegensatz zum linearen Optimierungsproblem muss bei einem durch
lineare Ungleichungen beschriebenen Gebiet ein
Optimum auf dem Rand nicht unbedingt in einer Ecke liegen,
sondern kann auch isoliert auf einem Kurven- oder Fl"achenst"uck
des Randes auftreten.
In Abbildung~\ref{nlp:kkt-moegliche-extrema} werden insgesamt
vier m"ogliche Kandidaten von Extrema (rote Puntke) jeweils zusammen mit
der Niveaulinie der Zielfunktion $f(x)$ dargestellt.

\begin{figure}
\begin{center}
\includegraphics{images/nlp-6.pdf}
\end{center}
\caption{Herleitung der Karush-Kuhn-Tucker Bedingungen f"ur ein Extremum
in einer Ecke des zul"assigen Gebietes.
\label{nlp:kkt-ableitung}}
\end{figure}
Nehmen wir also an, ein Minimum werde im Punkt $x_*$ angenommen,
und sei $h_k(x)\le 0$ eine der Ungleichungen, die im Punkt $x_*$ 
als Gleichung erf"ullt sind, also $h_k(x_*)=0$.
In Abbildung~\ref{nlp:kkt-ableitung} ist die Situation f"ur den
Fall dargestellt, dass $h_1(x)\le 0$ und $h_2(x)\le0$ die zwei
Ungleichungen sind, die im Punkt $x_*$ als Gleichungen erf"ullt
sind.
Richtungen $\xi\in\mathbb R^n$, f"ur die
$\operatorname{grad}h_j(x_*)\cdot\xi > 0$ ist,
f"uhren aus dem zul"assigen Gebiet hinaus. Es d"urfen also nur Richtungen
$\xi$ betrachtet werden, f"ur die $\operatorname{grad}h_j(x_*)\cdot\xi <0$
gilt.

Wie bei der Untersuchung des dualen Problems eines linearen Programms
muss in einem Maximum die Normale der Niveaulinien der Zielfunktion 
in dem Sektor aufgespannt von den Normalen der Randkurven liegen,
die durch die als Gleichungen erf"ullten Ungleichungen definiert
werden. Es muss also Zahlen $\mu_j\ge 0$ geben so, dass
\[
\operatorname{grad}f(x_*)=\sum_{j=0}^l\mu_j\operatorname{grad}h_j(x_*).
\]
Die Zahl $\mu_j$ wird $0$, wenn $x_*$ im Inneren des durch die Ungleichung
$h_j(x)\le 0$ definierten Gebietes liegt, wenn also die Normale
$\operatorname{grad}h_j(x_*)$ gar nicht ``gebraucht'' wird.
Die Zahl $\mu_j$ heisst {\em Lagrange-Multiplikator} der
Ungleichung $h_j(x)\le 0$.
\index{Lagrange-Multiplikator}

Zusammen mit dem Verfahren der Lagrange-Multiplikatoren bekommen wir 
folgenden Algorithmus zur Bestimmung der Extrema einer nichtlinearen
Funktion $f(x)$ unter nichtlinearen Nebenbedingungen $g_i(x)=0$
und nichtlinearen Ungleichunge $h_j(x)\le 0$:

\begin{satz}
Seien $f, g_i, h_j\colon\mathbb R^n\to \mathbb R$ differenzierbare Funktionen
mit $1\le i\le m$ und $1\le j\le l$.
Ist ausserdem $x_*$ ein Maximum von $f$ unter
den Nebenbedingungen $g_i(x_*)=0$ f"ur alle $1\le i\le m$ und unter den
Ungleichungen $h_j(x_*)\le 0$ f"ur alle $1\le j\le l$, dann gibt es
Zahlen $\lambda_i\in\mathbb R$ und $\mu_j\ge 0$ so, dass
\begin{align}
\operatorname{grad}f(x_*)
&=
\sum_{i=1}^m\lambda_i \operatorname{grad}g_i(x_*)
+
\sum_{j=1}^l\mu_j \operatorname{grad}h_j(x_*)
\label{nlp:karush-kuhn-tucker-gleichung}
\\
\mu_jh_j(x_*)&=0\quad \forall 1\le j\le l
\label{nlp:komplementaritaet}
\end{align}
Die Gleichung (\ref{nlp:karush-kuhn-tucker-gleichung}) heisst
{\em Karush-Kuhn-Tucker Bedingung}.
Eine Ungleichung $h_j(x)\le 0$ ist genau dann als Gleichung erf"ullt,
wenn das zugeh"orige $\mu_j=0$ ist.
Die Bedingung (\ref{nlp:komplementaritaet})
heisst {\em Komplementarit"atsbedingung}.
\index{Komplementarit\"atsbedingung}
\index{Karush-Kuhn-Tucker Bedingung}
\end{satz}

\begin{beispiel}
Man finde das Maximum der Funtion
\[
f(x,y,z)=2x^2 + y^2
\]
unter den Bedingungen
\begin{align*}
g(x,y,z)&=xyz-1=0
\\
h(x,y,z)&=x^2+y^2+z^2-9 \le 0
\end{align*}
Die Ungleichung $h(x,y,z)\le 0$ verlangt, dass das Maximum
im Inneren einer Kugel mit Radius $3$ gefunden wird.

\medskip
F"ur die Karush-Kuhn-Tucker Bedingung brauchen wir die
Gradienten aller drei Funktionen:
\begin{align*}
\operatorname{grad}f(x)
&=
\begin{pmatrix}4x\\2y\\0\end{pmatrix},
&
\operatorname{grad}g(x)
&=
\begin{pmatrix}yz\\xz\\xy\end{pmatrix},
&
\operatorname{grad}h(x)
&=
\begin{pmatrix}2x\\2y\\2z\end{pmatrix}.
\end{align*}
Die Karush-Kuhn-Tucker Bedingung wird dann
\begin{align*}
4x&=\lambda yz+2\mu x
\\
2y&=\lambda xz+2\mu y
\\
0&=\lambda xy + 2 \mu z
\end{align*}
Ausserdem m"ussen gelten
\begin{align*}
xyz-1&=0\\
\mu(x^2+y^2+z^2-9)&=0\\
\mu&\ge 0
\end{align*}

Wir untersuchen zun"achst, ob es ein Maximum im Inneren des Gebietes
geben k"onnte. Dass w"are dann der Fall, wenn $\mu=0$ w"are. Dann
werden die Gleichungen zu
\begin{align*}
4x&=\lambda yz
\\
2y&=\lambda xz
\\
0&=\lambda xy
\end{align*}
Aus der Bedingung $g(x,y,z)=xyz-1=0$ folgt, dass keine der Variablen
verschwinden darf. Die letzte Gleichung kann also nur erf"ullt werden,
wenn $\lambda = 0$ ist. Dann folgt aber aus den ersten zwei Gleichungen,
dass $x=0$ und $y=0$, ein Widerspruch.
Es kann also nur Extrema mit $\mu >0$ geben.

Wir bringen jetzt in Karush-Kuhn-Tucker Gleichungen die $\mu$ auf die
linke Seite:
\begin{equation}
\begin{aligned}
(4-2\mu)x&=\lambda yz\\
(2-2\mu)y&=\lambda xz\\
-2\mu z&=\lambda xy
\end{aligned}
\label{nlp:kkt-beispiel:gleichungen}
\end{equation}
Die rechte Seite kann man vereinfachen, indem man $g(x,y,z)=0$ verwendet:
\[
g(x,y,z)=xyz-1=0\qquad\Rightarrow\qquad
xy=\frac1z,\quad
xz=\frac1y,\quad
yz=\frac1x
\]
Eingsetzt  in (\ref{nlp:kkt-beispiel:gleichungen}) bekommt man
\begin{equation}
\begin{aligned}
x^2&=\frac{\lambda}{4-2\mu},&
y^2&=\frac{\lambda}{2-2\mu},&
z^2&=\frac{\lambda}{-2\mu}.
\end{aligned}
\label{nlp:kkt-beispiel:x2}
\end{equation}
Setzt man dies in $h$ ein, bekommt man
\begin{equation}
\frac{\lambda}2
\biggl(
\frac1{2-\mu}+\frac1{1-\mu}-\frac1{\mu}
\biggr) =9
\quad\Rightarrow\quad
\lambda=18\biggl(
\frac1{2-\mu}+\frac1{1-\mu}-\frac1{\mu}
\biggr)^{-1}
\label{nlp:kkt-beispiel:h}
\end{equation}
Setzt man (\ref{nlp:kkt-beispiel:x2}) in die Gleichung $x^2y^2z^2=1$ ein,
bekommt man
\begin{equation}
\lambda^3=-8(2-\mu)(1-\mu)\mu.
\label{nlp:kkt-beispiel:g}
\end{equation}
Durch Einsetzen von (\ref{nlp:kkt-beispiel:h}) in (\ref{nlp:kkt-beispiel:g})
kann man ein Polynom-Gleichung f"ur $\mu$ finden, die man numerisch
l"osen kann. Oder man kann das Gleichungssystem aus
(\ref{nlp:kkt-beispiel:h}) und (\ref{nlp:kkt-beispiel:g}) direkt
mit dem Computer l"osen, und dann daraus die Koordinaten $x$, $y$ und
$z$ mit den Formeln (\ref{nlp:kkt-beispiel:x2}) ausrechnen.
\end{beispiel}

\subsection{Karush-Kuhn-Tucker Bedingung f"ur lineare Programme}
Ein lineares Programm
\[
\max\{ c^tx\,|\,Ax\le b\}
\]
kann nat"urlich auch mit der eben entwickelten Theorie behandelt
werden.
Dazu m"ussen die Gradienten der Zielfunktion und der Ungleichungen
berechnet werden. Die Zielfunktion ist
\[
f(x)=c^tx=c_1x_1+c_2x_2+\dots c_nx_n
\]
und hat den Gradienten
\[
\operatorname{grad}f(x)
=
\begin{pmatrix}c_1\\c_2\\\vdots\\c_n\end{pmatrix}.
\]
Die Ungleichungen haben die Form
\[
h_j(x)=a_{j1}x_1+a_{j2}x_2+\dots+a_{jn}x_n,
\]
ihr Gradient ist
\[
\operatorname{grad}h_j(x)
=
\begin{pmatrix}a_{j1}\\a_{j2}\\\vdots\\a_{jn}\end{pmatrix}.
\]
Die Karush-Kuhn-Tucker Bedingung sagt dann, dass es Zahlen $\mu_j\ge 0$
gibt mit 
\[
\begin{pmatrix}c_1\\c_2\\\vdots\\c_n\end{pmatrix}
=
\mu_1\begin{pmatrix}a_{11}\\a_{12}\\\vdots\\a_{1n}\end{pmatrix}
+
\mu_2\begin{pmatrix}a_{21}\\a_{22}\\\vdots\\a_{2n}\end{pmatrix}
+\dots+
\mu_m\begin{pmatrix}a_{m1}\\a_{m2}\\\vdots\\a_{mn}\end{pmatrix}.
\]
Oder in Matrixform:
\[
A^t\mu=c,
\]
also genau die Gleichungen des dualen Programms.
Die Variablen des dualen linearen Programms sind also nichts anderes
als die Lagrange-Multiplikatoren der Ungleichungen des linearen
Programmes.
\section{Numerische Verfahren}
\rhead{Numerische Verfahren}
Die vorangegangenen Abschnitte dieses Kapitels zeigen, dass
sich ein Extremalproblem auch unter Nebenbedingungen oder Ungleichungen
immer darauf zur"uckf"uhren l"asst, die L"osung eines Systems von
nichtlinearen Ungleichungen zu finden.
Damit Extremalprobleme effizient gel"ost werden k"onnen, sind also
numerische Verfahren erforderlich, die nichtlineare Ungleichungen
effizient mit dem Computer zu l"osen gestatten.

\subsection{Newton-Verfahren}
\begin{figure}
\begin{center}
\includegraphics{images/nlp-7.pdf}
\end{center}
\caption{Newton-Verfahren f"ur eine Funktion einer Variablen.
\label{nlp:newton}}
\end{figure}
Zu l"osen ist also ein System von nichtlinearen Gleichungen
\begin{align*}
f_1(x_1,\dots,x_n)&=0\\
\vdots\qquad&=0\\
f_n(x_1,\dots,x_n)&=0
\end{align*}
Wir d"urfen annehmen, dass die Anzahl der Gleichungen gleich
gross ist wie die Anzahl der Unbekannten, weil andernfalls ohnehin
nicht mit einer eindeutigen L"osung zu rechnen ist.
Die Gleichungen k"onnen auch als vektorwertige Funktion
geschrieben werden:
\[
f(x_1,\dots,x_n)=\begin{pmatrix}
f_1(x_1,\dots,x_n)\\
\vdots\\
f_n(x_1,\dots,x_n)
\end{pmatrix}.
\]
Ist $x_k$ eine approximative Nullstelle, dann kann man mit Hilfe
der Ableitung eine bessere Approximation finden.
In einer Dimension kann man die notwendige Korrektur aus der
Abbildung~\ref{nlp:newton} ablesen:
\begin{equation}
x_{k+1}=x_k-\frac{f(x_k)}{f'(x_k)}.
\label{nlp:newton-iteration}
\end{equation}
Wiederholte Anwendung von (\ref{nlp:newton-iteration}) liefert
schrittweise immer besser Approximationen der Nullstelle.

\begin{beispiel}
Man finde eine Nullstelle der Funktion $f(x)=x^2-2$.

\medskip
{\parindent 0pt Nat"urlich erwartet man als L"osung $x=\sqrt{2}$.}
Als erste ungef"ahre Nullstelle verwenden wir $x_0=1$.
Die Iterationsformel (\ref{nlp:newton-iteration}) wird zu
\[
x_{k+1}=x_k-\frac{x_k^2-2}{2x_k}=\frac{2x_k^2-x_k^2+2}{2x_k}=
\frac12\biggl(x_k+\frac{2}{x_k}\biggr).
\]
Numerische Rechnung ergibt:
\begin{center}
\begin{tabular}{|>{$}c<{$}|>{$}c<{$}|}
\hline
k&x_k\\
\hline
0&1.0000000000\\
1&1.5000000000\\
2&1.4166666667\\
3&1.4142156863\\
4&1.4142135624\\
5&1.4142135624\\
\hline
\end{tabular}
\end{center}
Die Konvergenz des Verfahrens ist ausserordentlich schnell: in jedem
Schritt wird die Zahl der korrekten Stellen ungef"ahr verdoppelt.
\end{beispiel}

Das Newton-Verfahren kann auch auf ein System nichtlinearer
Gleichungen verallgemeinert werden. Wir m"ochten die Korrektur
$\xi=x_{k+1}-x_k$ berechnen, die jetzt ein Vektor ist.
Die partiellen Ableitungen der Funktionen $f_i$ liefern eine
lineare Approximation:
\[
f(x_k+\xi)\simeq f(x_k) + \sum_{j=1}^n\frac{\partial f_i}{\partial x_j}\xi_j
\]
Wir glauben nat"urlich, dass $f(x_k+\xi)=0$ sein wird, dass wir also
im n"achsten Schritt eine viel bessere Approximation der L"osung haben werden.
Die Approximation wird dann zu
\[
\begin{pmatrix}
-f_1(x_k)\\
-f_2(x_k)\\
\vdots\\
-f_n(x_k)
\end{pmatrix}
=
\begin{pmatrix}
\frac{\partial f_1}{\partial x_1}
	&\frac{\partial f_1}{\partial x_2}
		&\dots
			&\frac{\partial f_1}{\partial x_n}\\
\frac{\partial f_2}{\partial x_1}
	&\frac{\partial f_2}{\partial x_2}
		&\dots
			&\frac{\partial f_2}{\partial x_n}\\
\vdots
	&\vdots
		&\ddots
			&\vdots\\
\frac{\partial f_n}{\partial x_1}
	&\frac{\partial f_n}{\partial x_2}
		&\dots
			&\frac{\partial f_n}{\partial x_n}
\end{pmatrix}\xi
\]
Dieses lineare Gleichungssystem kann man mit der inversen
Matrix nach $\xi$ aufl"ossen. Schreiben wir $Df$ f"ur die Matrix
der ersten Ableitungen, dann ist
\[
\xi = -(Df)^{-1}\, f(x_k).
\]
oder die Iterationsformel
\begin{equation}
x_{k+1} = x_k + (x_{k+1}-x_k)=x_k+\xi=x_k-(Df)^{-1}\,f(x_k).
\label{nlp:newton-iteration-ndim}
\end{equation}

\subsection{Nullstellen der ersten Ableitung}
Das Newton-Verfahren kann nat"urlich auch auf die erste Ableitung
angewendet werden, um Nullstellen derselben zu finden.
Es kann aus einer angen"aherten L"osung $x_0$
mit Hilfe der Iteration
\begin{equation}
x_{k+1}=x_k-\frac{f'(x_k)}{f''(x_k)}
\label{newton-1dim}
\end{equation}
eine verbesserte L"osung finden. Die Folge $x_k$ konvergiert mit
quadratischer Konvergenz gegen eine L"osung, wenn der Anfangswert
$x_0$ gen"ugend nahe bei einer L"osung liegt.

Das Newton-Verfahren konvergiert schlecht, wenn $f''(x_*)=0$.
Der kleine Nenner in (\ref{newton-1dim}) kann bewirken, dass
$x_{k+1}$ weit weg von $x_*$ zu liegen kommt, so dass die
Veraussetzung, dass $x_{k+1}$ bereits eine eingermassen gute 
Approximation sein muss, nicht mehr erf"ullt ist.
Diese Einschr"ankung ist allerdings nicht weiter schlimm, denn
es interessieren ja ohnehin vor allem die Maxima ($f''(x_*)<0$)
und die Minima ($f''(x)>0$), in beiden F"allen ist also
$f''(x_*)\ne 0$.

Da das Newton-Verfahren immer nur eine lokale Suche nach einer
Nullstelle von $f'(x)$ unternimmt, ist es grunds"atzlich nicht
in der Lage, alle lokalen Extrema zu finden.
Es kann nur bereits approximativ bekannte Nullstellen verbessern,
und h"ochstens in Ausnahmef"allen auch aus einer sehr schlechten
Anfangssch"atzung eine gute L"osung finden.

\subsection{Nullstellen des Gradienten}
Auch f"ur die Nullstellen des Gradienten kann man das Newton-Verfahren
verwenden. Die Funktion, deren Nullstellen man jetzt finden will,
ist
\[
(x_1,\dots,x_n)\mapsto\begin{pmatrix}
\frac{\partial f}{\partial x_1}\\
\frac{\partial f}{\partial x_2}\\
\vdots\\
\frac{\partial f}{\partial x_n}
\end{pmatrix}.
\]
F"ur die Newton-Iteration (\label{nlp:newton-iteration-ndim}) braucht
man die Ableitungsmatrix, in diesem Fall ist dies
die Matrix mit den Matrix-Elementen
\[
Df=\biggl(
\frac{\partial^2f}{\partial x_i\,\partial x_j}
\biggr),
\]
also die Hessesche Matrix, die man f"ur die Entscheidung, ob ein
Maximum oder Minimum vorliegt, ohnehin berechnen muss.

\begin{beispiel}
Man finde Maxima der Funktion $f(x,y)=\sin x\sin y$.

\medskip
{\parindent 0pt Der Gradient dieser Funktion ist}
\[
\operatorname{grad}f(x,y)=
\begin{pmatrix}
\cos x\sin y\\
\sin x\cos y
\end{pmatrix}.
\]
Die Ableitung des Gradienten ist die Hessesche Matrix:
\[
H=\begin{pmatrix}
-\sin x\sin y&\cos x\cos y\\
\cos x\cos y&-\sin x\sin y
\end{pmatrix}
\]
F"ur das Newton-Verfahren brauchen wir die Inverse, die kann aber
f"ur eine $2\times 2$-Matrix leicht mit Hilfe der Determinanten
berechnet werden:
\[
H^{-1}=\frac{1}{\sin^2x\sin^2y-\cos^2x\cos^2y}\begin{pmatrix}
-\sin x\sin y&-\cos x\cos y\\
-\cos x\cos y&-\sin x\sin y
\end{pmatrix}
\]
Damit kann man jetzt die Newton-Iteration ausgehend vom Punkt $(1,1)$
durchf"uhren:
\begin{center}
\begin{tabular}{|>{$}c<{$}|>{$}c<{$}>{$}c<{$}|}
\hline
k&x_k&y_k\\
\hline
0 & 1.0000000000 & 0.0000000000\\
1 & 2.0925199316 & 2.0925199316\\
2 & 1.2339468373 & 1.2339468373\\
3 & 1.6330931388 & 1.6330931388\\
4 & 1.5704719562 & 1.5704719562\\
5 & 1.5707963268 & 1.5707963268\\
6 & 1.5707963268 & 1.5707963268\\
\hline
\end{tabular}
\end{center}
Ein anderer Startpunkt kann nat"urlich ein anderes Resultat geben,
von $(1,0.8)$ aus erh"alt man:
\begin{center}
\begin{tabular}{|>{$}c<{$}|>{$}c<{$}>{$}c<{$}|}
\hline
k&x_k&y_k\\
\hline
0 & 1.0000000000 & 0.8000000000\\
1 & 3.0417758196 & 3.0444858551\\
2 & 3.1428854616 & 3.1428854549\\
3 & 3.1415926507 & 3.1415926507\\
4 & 3.1415926536 & 3.1415926536\\
\hline
\end{tabular}
\end{center}
Wieder beobachtet man sehr schnelle Konvergenz.
\end{beispiel}

\section{Zusammenfassung: Das Wichtigste in K"urze}
\begin{itemize}
\item Kandidaten f"ur die Extremstellen von differenzierbaren
Funktionen sind die Nullstellen der Ableitung (eine Variable,
Abschnitt~\ref{nlp:extremwerte:einevariable})
oder des Gradienten (mehrere Variable,
Abschnitt~\ref{nlp:extremwerte:mehrerevariable}).
\item Wenn die Matrix der zweiten Ableitungen
\[
h_{ij}=
\frac{\partial^2f(x_*)}{\partial x_i\,\partial x_j}
\]
an einer Stelle $x_*$ mit $\operatorname{grad}f(x_*)=0$
positiv definit ist, liegt ein lokales Minimum vor. Ist die Matrix $H$
negativ definit ist $x_*$ ein lokales Maximum.
\item Ein symmetrische $n\times n$-Matrix ist definit, wenn alle
Eigenwerte das gleiche Vorzeichen haben.
\item Ob eine zweidimensionale symmetrische Matrizen $H$ positiv oder negativ
definit ist, kann man aus Determinante und Spur ablesen.
Ist $\det H >0$, ist sie definit, andernfalls ist sie indefinit.
Ist $\operatorname{tr}H>0$, ist sie positiv definit, sonst negativ
definit.
\item Kandidaten f"ur Extremstellen einer Funktion $f(x)$ unter
Nebenbedingungen $g_i(x)=0$, $1\le i\le m$ erf"ullen die 
Gleichung der Lagrange-Multiplikatoren $\lambda_i$
\[
\operatorname{grad}f(x_*)=\sum_{i=1}^m\lambda_i\operatorname{grad}g_i(x_*).
\]
\item Kandidaten f"ur eine Extremstelle einer Funktion $f(x)$ unter
Nebenbedingungen $g_(x)$, $1\le i\le m$ und Ungleichungen $h_j(x)\le 0$,
$1\le j\le l$ erf"ullen die Karush-Kuhn-Tucker Bedingung
\[
\operatorname{grad}f(x_*)=\sum_{i=1}^m\lambda_i\operatorname{grad}g_i(x_*).
+\sum_{j=1}^l\mu_j\operatorname{grad}h_j(x_*)
\]
und die Komplementarit"atsbedingung
\[
\mu_jh_j(x_*)=0,\quad\forall 1\le j\le l.
\]
Ausserdem m"ussen die $\mu_j\ge0$ sein.
\item Ist $\mu_j>0$, dann ist die zugeh"orige Ungleichung $h_j(x_*)=0$
als Gleichung erf"ullt.
\item Mit dem Newton-Verfahren kann man ausgehend von Sch"atzungen
der L"osungen eines nichtlinearen Gleichungssystems sehr schnell
eine hochgenaue L"osung bekommen.
\end{itemize}
