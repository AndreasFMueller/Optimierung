\chapter{Lineare Optimierung\label{chapter-lineare-optimierung}}
\lhead{Lineare Optimierung}

\section{Problemstellung\label{lp:section:problem}}
\rhead{Problemstellung}
\index{Optimierungsproblem!lineares}
In einem linearen Optimierungsproblem ist die Kostenfunktion linear
und die Ungleichungen, die das zul"assige Gebiet definieren, sind
ebenfalls linear. 
Die Kostenfunktion ist also von der Form
\[
Z=c^tx,\quad c,x\in\mathbb R^n.
\]
Da eine lineare Funktion unbeschr"ankt ist, kann ein lineares
Optimierungsproblem nur dann eine L"osung haben, wenn das zul"assige
Gebiet beschr"ankt ist, mindestens in Richtung gr"osster Zunahme der
Kostenfunktion. Es ist daher kaum eine Einschr"ankung, anzunehmen, dass 
die Variablen $x\in\mathbb R^n$ alle positiv sind. Notfalls kann man 
das durch addieren einer Konstante oder Vorzeichenwechsel erzwingen.

Die Ungleichungen sind von der Form
\[
Ax\le b,
\]
wobei $A$ eine $m\times n$-Matrix und $b\in\mathbb R^m$ ist.
Auch die h"aufig auftretenden Bedingungen $x\ge 0$ k"onnen in
diese Form gebracht werden: $x_i\ge 0\Leftrightarrow -x_i \le 0$.

Eine lineare Gleichung beschreibt eine Ebene, eine
lineare Ungleichung demzufolge einen Halbraum. Das zul"assige Gebiet
eines linearen Optimierungsproblem ist also immer die Schnittmenge
mehrere Halbr"aume.
Ein solches Gebiet hat die Eigenschaft, das jeder Punkt der
Verbindungsstrecke zweier Punkte des Gebietes ebenfalls im Gebiet liegt,
man nennt ein solches Gebiet konvex.
In $\mathbb R^2$  ist ein solches Gebiet ein konvexes Polygon,
in h"oheren Dimensionen ein konvexes Polyeder.

\begin{definition}
\index{Programm!lineares}
Ein {\em lineares Programm} ist die Vorgabe einer linearen Zielfunktion
$c^tx$ und einer Menge von Ungleichungen $Ax\le b$.
Wir schreiben
\[
\max\{c^tx\,|\,Ax\le b\}.
\]
f"ur das lineare Programm. Eine L"osung $x^*$ des linearen Programms
ist ein Vektor so, dass $Ax^*\le b$ und $c^tx^*\ge c^tx$ f"ur jeden
anderen Vektor $x$ mit $Ax\le b$.
\end{definition}

\section{Graphische L"osung\label{lp:section:graphisch}}
\rhead{Graphische L"osung}
\index{Programm!lineares, graphische L\"osung}
F"ur nur zwei Variablen l"asst sich ein lineares Optimierungsproblem
sehr gut visualisieren. Die Ungleichungen
\[
Ax\le b
\]
definieren ein polygonales Gebiet in der Ebene, das zul"assige Gebiet.
Wir illustrieren das an den Beispielungleichungen
\begin{equation}
\begin{linsys}{2}
-x&+&3y&\le&12\\
 x&+&3x&\le&18\\
4x&+& y&\le&28\\
-x& &  &\le&0\\
  & &-y&\le&0
\end{linsys}
\label{lp-beispiel-ungleichungen}
\end{equation}
Sie definieren das Gebiet $G$ in Abbildung \ref{lp-beispiel}.
Die letzten zwei Gleichungen dr"ucken aus, dass $x\ge 0$ und $y\ge 0$.
\begin{figure}
\begin{center}
\includegraphics[width=\hsize]{images/lp-1}
\end{center}
\caption{Zul"assiges Gebiet f"ur die Ungleichungen
(\ref{lp-beispiel-ungleichungen})\label{lp-beispiel}}
\end{figure}

Die Zielfunktion ist ebenfalls linear, f"ur das Beispielgebiet
(\ref{lp-beispiel-ungleichungen}) verwenden wir die Zielfunktion
\begin{equation}
Z=2x+3y
\label{lp-beispiel-zielfunktion}
\end{equation}
\begin{figure}
\begin{center}
\includegraphics[width=\hsize]{images/lp-2}
\end{center}
\caption{Zul"assiges Gebiet mit Graph der Zielfunktion f"ur verschiedene
Zielwerte. Zielwert $Z=24$ ist optimal, er wird angenommen im
Punkt $(6,4)$.
\label{lp-beispiel-zielfunktion}}
\end{figure}%
In Abbildung \ref{lp-beispiel-zielfunktion} sind die Geraden $Z=\operatorname{const}$
f"ur einige Werte der Zielfunktion eingezeichnet.
Man kann ablesen, dass das Optimum im Schnittpunkt der Geraden
beschrieben durch die beiden Gleichungen
\begin{equation}
\begin{linsys}{2}
x&+&3x&=&18\\
4x&+&y&=&28
\end{linsys}
\notag
\end{equation}
angenommen wird. Sie beschreiben den Rand der Halbebenen, die die
zweite und dritte Ungleichung von (\ref{lp-beispiel-ungleichungen}) 
beschreiben.
Die L"osung des Gleichungssystems liefert die optimalen Koordinaten
$(6,3)$.

\section{Das duale Problem\label{lp:section:dual}}
\rhead{Das duale Problem}
\index{Programm!duales lineares}
Aus der graphischen L"osung l"asst sich noch mehr ableiten.
Die L"osung hat man gefunden, wenn sich die Gerade der Zielfunktion
nicht mehr weiter zu gr"osseren Werten von $Z$ verschieben l"asst, ohne
dass sie das zul"assige Gebiet $G$ gar nicht mehr schneidet.
Es darf also in Richtung der Normalen $\vec n_Z$ der Zielfunktion keine Punkte
des Gebietes $G$ mehr geben. Oder anders ausgedr"uckt: f"ur jede
Halbebene, deren Rand durch die L"osungsecke geht, muss die Normale der
Zielfunktion ausserhalb der Halbebene liegen.
Die Normale der Zielfunktion muss sich also linear aus den Normalen
der Halbebenen kombinieren lassen, aber {\em mit positiven} Koeffizienten.
In der Abbildung~\ref{lp:normalen} sind in der N"ahe des Optimums beim
Schnittpunkt der durch die Ungleichungen 2 und 3 definierten Geraden
die zugeh"origen Normalen mit $\vec n_2$ und $\vec n_3$ bezeichnet.
\begin{figure}
\begin{center}
\includegraphics{images/lp-3.pdf}
\end{center}
\caption{Normalen von Zielfunktion und Rand in der Umgebung eines Optimums.
Die Normale $\vec n_Z$ der Zielfunktion muss im grau eingezeichneten
Sektor liegen, also eine positive Linearekombination von $\vec n_2$ und
$\vec n_3$ sein.
\label{lp:normalen}}
\end{figure}

Die Halbebenen, deren Rand eine Ecke nicht trifft, wird nicht verwendet,
in einer Linearkombination von normalen gehen diese also mit dem
Koeffizienten $0$ ein.

Die Normale der Zielfunktion ist der Vektor aus den Koeffizienten, also $c$,
die Normale der Halbr"aume sind die einzelnen Zeilen von $A$.
Gesucht sind also Zahlen $\mu_1,\dots,\mu_m\ge 0$ so, dass
\begin{center}
\begin{tabular}{>{$}c<{$}>{$}c<{$}|>{$}c<{$}>{$}c<{$}>{$}c<{$}}
&\mu_1\cdot&a_{11}&\dots&a_{1n}\\
+&\vdots&\vdots&\ddots&\vdots\\
+&\mu_m\cdot&a_{m1}&\dots&a_{mn}\\
\hline
=&&c_1&\dots&c_n
\end{tabular}
\end{center}
als Gleichungssystem geschrieben
\begin{equation}
\begin{linsys}{3}
a_{11}\mu_1&+&\dots&+&a_{m1}\mu_m&=&c_1\\
\vdots&&\ddots&&\vdots&&\vdots\\
a_{1n}\mu_1&+&\dots&+&a_{mn}\mu_m&=&c_n\\
\end{linsys}
\end{equation}
Schreiben wir $\mu$ f"ur den Spaltenvektor der $\mu_1,\dots,\mu_m$,
k"onnen wir die Bedingung in Matrixform bringen:
\[
A^t\mu=c
\qquad
\text{oder}
\qquad
\mu^tA=c^t.
\]
Es verschwinden genau diejenigen $\mu_i$ f"ur die die zugeh"origen
Ungleichungen $a_{i1}x_1+\dots+a_{in}x_n\le b_i$ nicht exakt
erf"ullt sind.
W"ahlt man also einen anderen Punkt $x$ als das Optimum muss
gelten
\[
\mu_i (b_i - a_{i1}x_1-\dots-a_{in}x_n)\ge 0
\qquad
\Rightarrow
\qquad
\mu_i b_i \ge \mu_i(a_{i1}x_1+\dots+a_{in}x_n),
\]
oder in Matrixform
\[
\mu^t b \ge \mu^tAx=c^tx,
\]
mit Gleichheit genau dann, wenn wir f"ur $x$ des Optimum nehmen.
Der kleinstm"ogliche Wert von $\mu^tb$ ist also das Maximum von $c^tx$.
\begin{definition}
Ist $\max\{c^tx\,|\, Ax\le b\}$ ein lineares Programm, dann heisst
\[
\min\{\mu^t b\,|\, \mu^tA= c^t,\mu\ge 0\}
\]
das dazu duale lineare Programm.
\end{definition}
Das duale Problem hat $m$ Ungleichungen und $n$ Gleichungen.
Der "Ubergang zum dualen Problem macht also aus einem Problem mit $m$ 
Ungleichungen und $n$ Unbekannten ein Problem mit $n$ Gleichungen,
$m$ (einfachen) Ungleichungen, und $m$ Unbekannten.
Aus einem Problem mit vielen Variablen und wenigen Ungleichungen wird
ein Problem mit wenigen Variablen und Ungleichungen und vielen Gleichungen.
Da wir f"ur Gleichungen im Gauss-Algorithmus einen effizienten Helfer haben,
kann das eine bedeutende Vereinfachung sein.

Das duale Problem liefert den gleichen Extremwert.
Sofern man also nur am Maximum interessiert ist, und nicht an den $x$-Werten,
f"ur die es zustande kommt, dann liefert das duale Problem bereits die
L"osung.

Das duale Problem liefert aber nicht nur den Extremwert, sondern
auch die Information, welche Gleichungen exakt erf"ullt sind.
Es sind ja genau diejenigen $\mu_i$ von $0$ verschieden, deren
zugeh"orige Ungleichungen exakt erf"ullt sind. Wenn man also
das duale Problem gel"ost hat, findet man die L"osung des urspr"unglichen
Problems durch l"osen der Gleichungen, die zu den von $0$ verschiedenen
$\mu_i$ geh"oren.

Etwas genauer gilt folgender Satz:
\begin{satz}
Ist $\max\{c^tx\,|\, Ax\le b\}$ ein lineares Programm und 
$\min\{\mu^tb\,|\, \mu^t A=c, \mu \ge 0\}$ das zugeh"orige duale Problem,
dann sind "aquivalent
\begin{compactenum}
\item $x$ und $\mu$ sind L"osungen
\item $c^tx=\mu^tb$
\item $\mu^t(b-Ax)=0.$
\end{compactenum}
\end{satz}

\begin{beispiel}
Wir formulieren das zum linearen Programm
(\ref{lp-beispiel-ungleichungen})
und
(\ref{lp-beispiel-zielfunktion})
duale Programm. Zu minimieren ist die Funktion
\begin{equation}
12\mu_1+18\mu_2+28\mu_3
\label{dual-lp-zielfunktion}
\end{equation}
unter den Bedingungen
\begin{equation}
\begin{linsys}{5}
- \mu_1&+& \mu_2&+&4\mu_3&-&\mu_4& &      &=2\\
 3\mu_1&+&3\mu_2&+& \mu_3& &     &-& \mu_5&=3
\end{linsys}
\label{lp-dual-gleichungen}
\end{equation}
und
\begin{equation}
\mu_i\ge 0,\quad 1\le i\le 5.
\label{lp-dual-ungleichungen}
\end{equation}
Verwenden wir, dass wir aus der graphischen L"osung bereits wissen,
dass nur die zweite und dritte urspr"ungliche Ungleichung von Bedeutung
ist, dann folgt $\mu_1=\mu_4=\mu_5=0$ und es bleibt das Gleichungssystem
\[
\begin{linsys}{2}
 \mu_2&+&4\mu_3&=&2\\
3\mu_2&+& \mu_3&=&3
\end{linsys}
\]
mit L"osung $\mu_2=\frac{10}{11}$ und $\mu_3=\frac{3}{11}$.
Das Minimum ist also
\[
\mu^tb=\begin{pmatrix}0&0&\frac{10}{11}&\frac{3}{11}&0\end{pmatrix}
\begin{pmatrix}12\\18\\28\\0\\0\end{pmatrix}
=
\frac{18\cdot 10}{11}+\frac{28\cdot 3}{11}=\frac{180+84}{11}=\frac{264}{11}=24,
\]
wie schon in der graphischen L"osung gefunden.
\end{beispiel}

\section{Von Ungleichungen zu Gleichungen\label{lp:section:ungleichungen}}
\rhead{Von Ungleichungen zu Gleichungen}
Die graphische L"osung hat gezeigt, dass der L"osungsvektor eines
linearen Optimierungsproblems immer einige der Ungleichungen exakt
erf"ullt. Die L"osung kann also in zwei Schritten gefunden werden:
\begin{compactenum}
\item Herausfinden, welche Ungleichungen exakt erf"ullt sein m"ussen.
\item Mit diesen Gleichungen ein lineares Gleichungssystem l"osen.
\end{compactenum}
Die L"osung des Gleichungssystems im zweiten Schritt l"asst sich mit
dem Gauss-Algorithmus sehr effizient finden. Das Hauptproblem ist also
der erste Schritt.

DIe Hauptschwierigkeit ist dabei, dass wir mit Ungleichungen zu tun
haben.
F"ur die Bestimmung von L"osungsmengen von Ungleichungen gibt
es kaum gut bekannte Verfahren.
Ein Zwischenschritt kann sein, das System von Ungleichungen in
ein System von Gleichungen umzuwandeln.
Die Idee: verwende die Abweichung von der Gleichheit als zus"atzliche
Unbekannte.

Die ersten drei der Ungleichungen (\ref{lp-beispiel-ungleichungen})
k"onnen durch
Einf"uhrung zus"atzlicher Variablen $s_1,\dots,s_3$ in die Gleichungen
\begin{equation}
\begin{linsys}{5}
-x&+&3y&+&s_1& &   & &   &=&12\\
 x&+&3y& &   &+&s_2& &   &=&18\\
4x&+& y& &   & &   &+&s_3&=&28
\end{linsys}
\label{lp-ungleichungen-schlupf}
\end{equation}
umgewandelt werden.
Die Variablen $s_i$ heissen {\em Schlupfvariablen}
\index{Schlupfvariablen}
Die Variable $s_i$ gibt wieder, wie weit entfernt von der Gleichheit
die Ungleichung Nummer $i$ ist.
Nat"urlich muss $s_i\ge 0$ sein.
Ist die Variable $s_i=0$, dann ist die Ungleichung exakt erf"ullt.
Schritt~1 l"auft also darauf hinaus, dass festgestellt werden muss,
welche der Variablen $s_i$ verschwinden.

\begin{figure}
\begin{center}
\includegraphics{images/lp-8.pdf}
\end{center}
\caption{Ecken des zul"assigen Gebietes werden charakterisiert durch
die Variablen die verschwinden.
\label{lp:corners}}
\end{figure}
F"ur die Bestimmung des Optimums m"ussen die Ecken des Gebietes 
untersucht werden. Die Variablen $s_i$ geben an, ob ein Punkt auf
der Randgeraden der durch die zugeh"orige Ungleichung definierten
Halbebene liegt.
Ist $s_i=0$, dann liegt der Punkt auf der zugeh"origen Randgeraden.
Analog geben die Variablen $x$ und $y$ an, ob der Punkt auf der
$x$- oder $y$-Achsen liegt. Die Ecken des zul"assigen Gebietes
werden also durch die Variablen charakterisiert, die verschwinden,
wie in Abbildung~\ref{lp:corners} dargestellt.

Damit haben wir das lineare Programm auf ein neues Problem reduziert.
Schreiben wir $y$ f"ur die Vektoren bestehend aus den $x_i$ und den $s_i$,
und
\begin{align*}
A'&=\begin{pmatrix}
a_{11}&\dots &a_{1n}&1&\dots&0\\
\vdots&\ddots&\vdots&\dots&\ddots&\vdots\\
a_{m1}&\dots &a_{mn}&0&\dots&1
\end{pmatrix}
=\begin{pmatrix}A&E\end{pmatrix}
\\
c'&=\begin{pmatrix}x_1\\\vdots\\x_n\\0\\\vdots\\0\end{pmatrix},
\end{align*}
dann ist das urspr"ungliche lineare Programm
\[
\max\{ c^tx\,|\, Ax\le b\}
\]
"aquivalent zu einem neuen linearen Programm
\[
\max\{ c'^ty\,|\, A'y=b, y\ge 0\},
\]
welches nur noch einfache Ungleichungen hat.
Das neue lineare Programm hat $n+m$ Unbekannte, $m$ Gleichungen
und $n+m$ Ungleichungen.

Man beachte, dass das duale lineare Programm
ebenfalls ein lineares Programm "ahnlicher Struktur ist mit zus"atzlichen
Gleichungen, aber daf"ur nur mit einfachen Ungleichungen der Form
$\mu_i\ge 0$.

Die L"osung des Optimierungsproblems ist auf dem Rand des
zul"assigen Gebietes zu finden.
Die Ecken des Gebietes sind Schnittpunkt zweier Gleichungen.
Sie werden also dadurch bestimmt, dass zwei Variablen 
Null gesetzt werden, die verbleibenden drei Variablen
lassen sich dann mit Hilfe des Gauss-Algorithmus bestimmen.
Die oben graphisch gefundene L"osung entspricht der
Wahl $s_2=0$ und $s_3=0$.

Die Zielfunktion $Z$ ist eine lineare Funktion von $x$ und $y$.
Da drei Variablen eindeutig bestimmt sind, sobald zwei der
Variablen festgelegt sind, muss sich auch $Z$ durch diese beiden
Variablen ausdr"ucken lassen.
Um diesen Ausdruck zu finden,
kann man den Gauss-Algorithmus
verwenden, wenn man $Z$ als zus"atzliche erste Variable betrachtet.
Das zugeh"orige Gaustableau ist dann
\[
\begin{tabular}{|>{$}c<{$}|>{$}c<{$}>{$}c<{$}|>{$}c<{$}>{$}c<{$}>{$}c<{$}|>{$}c<{$}|}
\hline
Z&x&y&s_1&s_2&s_3& \\
\hline
1&-2&-3& 0& 0& 0& 0\\
\hline
0&-1& 3& 1& 0& 0&12\\
0& 1& 3& 0& 1& 0&18\\
0& 4& 1& 0& 0& 1&28\\
\hline
\end{tabular}
\]
Die erste Zeile ist die Zielfunktion, die unteren drei Zeilen codieren die
Ungleichungen.

F"uhrt man jetzt den Gaussalgorithmus durch, erh"alt man
\begin{align*}
&\rightarrow
\begin{tabular}{|>{$}c<{$}|>{$}c<{$}>{$}c<{$}|>{$}c<{$}>{$}c<{$}>{$}c<{$}|>{$}c<{$}|}
\hline
 1&-2&-3& 0& 0& 0&  0\\
\hline
 0& 1&-3&-1& 0& 0&-12\\
 0& 0& 6& 1& 1& 0& 30\\
 0& 0&13& 4& 0& 1& 76\\
\hline
\end{tabular}
&&\rightarrow
\begin{tabular}{|>{$}c<{$}|>{$}c<{$}>{$}c<{$}|>{$}c<{$}>{$}c<{$}>{$}c<{$}|>{$}c<{$}|}
\hline
 1&-2&-3&      0    &      0    & 0&  0\\
\hline
 0& 1&-3&     -1    &      0    & 0&-12\\
 0& 0& 1&\frac16    &\frac16    & 0&  5\\
 0& 0& 0&\frac{11}6 &-\frac{13}6& 1& 11\\
\hline
\end{tabular}
\\
&\rightarrow
\begin{tabular}{|>{$}c<{$}|>{$}c<{$}>{$}c<{$}|>{$}c<{$}>{$}c<{$}>{$}c<{$}|>{$}c<{$}|}
\hline
 1&-2&-3&          0&             0&            0&  0\\
\hline
 0& 1&-3&          0&-\frac{13}{11}& \frac{6}{11}& -6\\
 0& 0& 1&          0&  \frac{4}{11}&-\frac{1}{11}&  4\\
 0& 0& 0&          1&-\frac{13}{11}& \frac{6}{11}&  6\\
\hline
\end{tabular}
&&\rightarrow
\begin{tabular}{|>{$}c<{$}|>{$}c<{$}>{$}c<{$}|>{$}c<{$}>{$}c<{$}>{$}c<{$}|>{$}c<{$}|}
\hline
 1&-2& 0&          0& \frac{12}{11}&-\frac{3}{11}& 12\\
\hline
 0& 1& 0&          0& -\frac{1}{11}& \frac{3}{11}&  6\\
 0& 0& 1&          0&  \frac{4}{11}&-\frac{1}{11}&  4\\
 0& 0& 0&          1&-\frac{13}{11}& \frac{6}{11}&  6\\
\hline
\end{tabular}
\\
&\rightarrow
\begin{tabular}{|>{$}c<{$}|>{$}c<{$}>{$}c<{$}|>{$}c<{$}>{$}c<{$}>{$}c<{$}|>{$}c<{$}|}
\hline
 1& 0& 0&          0& \frac{10}{11}& \frac{3}{11}& 24\\
\hline
 0& 1& 0&          0& -\frac{1}{11}& \frac{3}{11}&  6\\
 0& 0& 1&          0&  \frac{4}{11}&-\frac{1}{11}&  4\\
 0& 0& 0&          1&-\frac{13}{11}& \frac{6}{11}&  6\\
\hline
\end{tabular}
\end{align*}
Aus dem letzten Tableau kann man die vollst"andige L"osung
des Optimierungsproblems ablesen. Das Optimum wird angenommen
f"ur $x=6$ und $y=4$, die erste der drei Ungleichungen ist
nicht scharf erf"ullt, und der optimale Wert der Zielfunktion
\[
Z=24-\frac{10}{11}s_2-\frac{3}{11}s_3
\]
ist $24$.

Ebenso kann man ablesen, dass sich der Wert der Zielfunktion
nicht mehr verbessern l"asst.
Dazu m"usste man n"amlich notwendigerweise $s_2$ oder $s_3$
vergr"ossern, und beides w"urde den Wert der Zielfunktion verringern.

Im allgemeinen Fall eines linearen Optimierungsproblems mit $n$
Unbekannten und $m$ Ungleichungen ist also ein Gleichungssystem
zu l"osen mit $n+m+1$ Variablen ($n$ Variablen, $m$ Schlupfvariablen und
$Z$) und $m+1$ Gleichungen zu l"osen.
Dabei sind $(n+m+1)-(m+1)=n$ Variablen frei w"ahlbar.
Sobald man weiss, welche $n$ Variablen frei gew"ahlt werden k"onnen,
kann das Problem mit dem Gauss-Algorithmus gel"ost werden.
Ob man tats"achlich eine L"osung gefunden hat, kann man daran
erkennen, ob alle Koeffizienten in der $Z$-Zeile des Gauss-Tableau
positiv sind.

\section{Simplex-Algorithmus\label{lp:section:simplex}}
\rhead{Simplex-Algorithmus}
\subsection{L"osungsprinzip}
Der Simplex-Algorithmus l"ost das Problem, die $n$ Variablen
zu bestimmen, die auf $0$ gesetzt werden k"onnen, um die L"osung
des Optimierungsproblems zu finden.
Kennt man bereits einen noch nicht optimalen Eckpunkt des zul"assigen
Gebietes, dann gibt es mindestens einen benachbarten Eckpunkt,
der die Funktion $Z$ vergr"ossert.
Man kann also von Ecke zu Ecke den Kanten des zul"assigen
Gebietes folgen, bis sich die Zielfunktion nicht mehr vergr"ossern
l"asst. Dies ist das Prinzip des Simplex-Algorithmus.

Die benachbarte Ecke erf"ullt bis auf eine Gleichung alle
Gleichungen der aktuellen Ecke.
Ausserdem erf"ullt sie eine Gleichung, die aktuelle Ecke nicht
erf"ullt.
Aus einer Menge der Variablen, die f"ur den urspr"unglichen
Eckpunkt auf $0$ gesetzt werden m"ussen, muss eine Variable
entfernt werden, und daf"ur eine neue Variable hinzugef"ugt
werden, die bisher $>0$ war.
Der Simplex-Algorithmus muss also in jedem Schritt entscheiden,
welche der frei w"ahlbaren Variablen durch welche andere 
Variable ersetzt werden soll.

\subsection{Beispiel}
\begin{figure}
\begin{center}
\includegraphics{images/lp-7.pdf}
\end{center}
\caption{Austauschschritte im Simplex-Algorithmus
\label{lp:simplex-steps}}
\end{figure}
Wir f"uhren dies am Beispiel (\ref{lp-beispiel}) und 
(\ref{lp-beispiel-zielfunktion}) durch, die einzelnen Schritte
sind in Abbildung~\ref{lp:simplex-steps} graphisch dargestellt.
Die Ecke $(0,0)$ ist im zul"assigen Gebiet, sie entspricht
der Menge $\{x,y\}$ der frei w"ahlbaren Variablen. Das zugh"orige
Gauss-Tableau ist
\[
\begin{tabular}{|>{$}c<{$}|>{$}c<{$}>{$}c<{$}|>{$}c<{$}>{$}c<{$}>{$}c<{$}|>{$}c<{$}|}
\hline
Z&x&y&s_1&s_2&s_3&\\
\hline
1&-2&-3& 0& 0& 0& 0\\
\hline
0&-1& 3& 1& 0& 0&12\\
0& 1& 3& 0& 1& 0&18\\
0& 4& 1& 0& 0& 1&28\\
\hline
 & *& *&  &  &  &  \\
\hline
\end{tabular}
\]

Zur Verbesserung des Resultats muss jetzt eine der Variablen $x,y$
durch eine der Variablen $s_1,s_2, s_3$ ersetzt werden. Beide
Koeffizienten von $x$ und $y$ in der Zielfunktion sind negativ, 
beide Variablen kommen also daf"ur in Frage, ersetzt zu werden.
Der Koeffizient von $y$ ist jedoch kleiner, so dass die Wahl von $y$
eine gr"ossere Verbesserung der Zielfunktion verspricht, als die
Wahl von $x$ als auszutauschender Variablen.

Ersetzt man $y$ durch die Variable $s_i$, dann zeigt Gleichung
$i$ an, wie gross $s_i$ gemacht werden kann, um zur n"achsten Ecke
zu gelangen:
\begin{align*}
s_1&=0&\Rightarrow&& y&= \frac{12}3=4
\\
s_2&=0&\Rightarrow&& y&= \frac{18}3=6
\\
s_3&=0&\Rightarrow&& y&= 28
\\
\end{align*}
Die n"achste Ecke wird also erreicht, wenn man f"ur $x$ den
Wert $7$ nimmt, was geschieht, wenn man $s_1=0$ setzt.
Es muss also $y$ gegen $s_1$ ausgetauscht werden.

Neu soll jetzt also $s_1$ eine frei w"ahlbare Variable werden, man
erreicht dies mit Hilfe eines Gauss-Schrittes, der das Element $-1$
in der zweiten Zeile und der zweiten Spalte zu $1$ macht und die "ubrigen
Elemente dieser Spalte zu $0$:
\[
\begin{tabular}{|>{$}c<{$}|>{$}c<{$}>{$}c<{$}|>{$}c<{$}>{$}c<{$}>{$}c<{$}|>{$}c<{$}|}
\hline
1&        -3& 0&       1& 0& 0& 12\\
\hline
0&  -\frac13& 1& \frac13& 0& 0&  4\\
0&         2& 0&      -1& 1& 0&  6\\
0&\frac{13}3& 0&-\frac13& 0& 1& 24\\
\hline
 & *&  & *&  &  &  \\
\hline
\end{tabular}
\]
Weil nicht alle Koeffizienten in der ersten Zeile positiv sind,
ist dies noch nicht die optimale L"osung.

Im n"achsten Schritt muss die Variable $x$ ersetzt werden, zur 
Auswahl stehen $y$, $s_2$ und $s_3$.
\begin{align*}
  y&=0&\Rightarrow&& -x&= 12
\\
s_2&=0&\Rightarrow&&  x&= \frac{6}{2} = 3
\\
s_3&=0&\Rightarrow&&  x&= \frac{72}{13}
\end{align*}
Damit die Ecke weiterhin im zul"assigen Bereich bleibt, ist $s_2=0$
die verbleibende Ecke, also muss $y$ gegen $s_2$ ausgetauscht werden.
\[
\begin{tabular}{|>{$}c<{$}|>{$}c<{$}>{$}c<{$}|>{$}c<{$}>{$}c<{$}>{$}c<{$}|>{$}c<{$}|}
\hline
1&         0& 0&   -\frac12&    \frac32&      0& 21\\
\hline
0&         0& 1&    \frac16&    \frac16&      0&  5\\
0&         1& 0&   -\frac12&    \frac12&      0&  3\\
0&         0& 0& \frac{11}6&-\frac{13}6&      1& 11\\
\hline
 &  &  & *& *&  &  \\
\hline
\end{tabular}
\]
Jetzt zeigt sich, dass $s_1$ ausgetauscht werden muss. 
\begin{align*}
  y&=0&\Rightarrow&& s_1&= 30
\\
  x&=0&\Rightarrow&& -s_1&= 6
\\
s_3&=0&\Rightarrow&& s_1&= 6
\end{align*}
Limitierend ist diesmal die dritte Gleichung. Also muss $s_1$ gegen
$s_3$ ausgetauscht werden.
\[
\begin{tabular}{|>{$}c<{$}|>{$}c<{$}>{$}c<{$}|>{$}c<{$}>{$}c<{$}>{$}c<{$}|>{$}c<{$}|}
\hline
1&         0& 0&          0& \frac{10}{11}& \frac{3}{11}& 24\\
\hline
0&         0& 1&          0&  \frac{4}{11}&-\frac{1}{11}&  4\\
0&         1& 0&          0& -\frac{1}{11}& \frac{3}{11}&  6\\
0&         0& 0&          1&-\frac{13}{11}& \frac{6}{11}&  6\\
\hline
 &          &  &           &             *&            *&  \\
\hline
\end{tabular}
\Leftrightarrow
\begin{tabular}{|>{$}c<{$}|>{$}c<{$}>{$}c<{$}|>{$}c<{$}>{$}c<{$}>{$}c<{$}|>{$}c<{$}|}
\hline
1&         0& 0&          0& \frac{10}{11}& \frac{3}{11}& 24\\
\hline
0&         1& 0&          0& -\frac{1}{11}& \frac{3}{11}&  6\\
0&         0& 1&          0&  \frac{4}{11}&-\frac{1}{11}&  4\\
0&         0& 0&          1&-\frac{13}{11}& \frac{6}{11}&  6\\
\hline
 &          &  &           &             *&            *&  \\
\hline
\end{tabular}
\]
Dieses Verfahren hat die gleiche L"osung gefunden wie das
graphische Verfahren.

\subsection{Austauschschritte\label{lp:subsection:austausch}}
\index{Austauschschritt}
Das lineare Programm
\[
\max\{ c^tx\,|\, Ax\le b, x\ge 0\}
\]
entspricht dem Tableau
\begin{equation}
\begin{tabular}{|
>{$}c<{$}|
>{$}c<{$}
>{$}c<{$}
>{$}c<{$}|
>{$}c<{$}
>{$}c<{$}
>{$}c<{$}|
>{$}c<{$}|}
\hline
Z&x_1&\dots&x_n&s_1&\dots&s_m&\\
\hline
1&-c_1&\dots&-c_2&0&\dots&0&0\\
\hline
0&a_{11}&\dots&a_{1n}&1&\dots&0&b_1\\
\vdots&\vdots&\ddots&\vdots&\vdots&\ddots&\vdots&\vdots\\
0&a_{m1}&\dots&a_{mn}&0&\dots&1&b_m\\
\hline
% &*&\dots&*& && & \\
%\hline
\end{tabular}
\end{equation}
Wir nehmen zun"achst an, dass der Punkt $x_i=0$, $i=1,\dots,n$ 
im zul"assigen Gebiet liegt.
Die Variablen $x_1,\dots,x_n$ sind also die frei w"ahlbar, die
Schlupfvariablen $s_1,\dots,s_m$ sind durch die Wahl der
$x_i$ bestimmt.
Wir symbolisieren dies durch Markierung der Variablen $x_i$ mit
Sternen in der untersten Zeile:
\begin{equation}
\begin{tabular}{|
>{$}c<{$}|
>{$}c<{$}
>{$}c<{$}
>{$}c<{$}|
>{$}c<{$}
>{$}c<{$}
>{$}c<{$}|
>{$}c<{$}|}
\hline
Z&x_1&\dots&x_n&s_1&\dots&s_m&\\
\hline
1&-c_1&\dots&-c_2&0&\dots&0&0\\
\hline
0&a_{11}&\dots&a_{1n}&1&\dots&0&b_1\\
\vdots&\vdots&\ddots&\vdots&\vdots&\ddots&\vdots&\vdots\\
0&a_{m1}&\dots&a_{mn}&0&\dots&1&b_m\\
\hline
 &*&\dots&*& && & \\
\hline
\end{tabular}
\end{equation}

Der Simplex-Algorithmus muss in jedem Schritt entscheiden,
welche der ausgew"ahlten Variablen durch welche andere Variable
ersetzt werden soll.
Danach muss ein Gauss-Schritt mit dieser Variablen durchgef"uhrt werden,
so dass alle abh"angigen Variablen durch neu als frei w"ahlbaren Variablen
festgelegten ausgedr"uckt werdne. Nach einigen Schritten wird man also
ein Tableau der Form
\begin{equation}
\begin{tabular}{|>{$}c<{$}|>{$}c<{$}>{$}c<{$}>{$}c<{$}>{$}c<{$}>{$}c<{$}|>{$}c<{$}|}
\hline
Z&y_1&y_2&y_3&\dots&y_n& \\
\hline
1&\tilde c_1&\tilde c_2&\tilde c_3&\dots&\tilde c_n&\tilde Z\\
\hline
&\tilde a_{11}&\tilde a_{12}&\tilde a_{13}&\dots&\tilde a_{1n}&\tilde b_1\\
&\tilde a_{21}&\tilde a_{22}&\tilde a_{23}&\dots&\tilde a_{2n}&\tilde b_2\\
&\vdots&\vdots&\vdots&\ddots&\vdots&\vdots\\
&\tilde a_{m1}&\tilde a_{m2}&\tilde a_{m3}&\dots&\tilde a_{mn}&\tilde b_m\\
\hline
& &*& &\dots&*&\\
\hline
\end{tabular}
\end{equation}
Wir haben die Bezeichnung der Variablen vereinheitlicht, und unterschieden
nicht mehr zwischen $x_i$ und $s_j$.
Mit einem Stern bezeichnete Spalten sind frei w"ahlbar. Setzt man diese
Variablen auf $0$, findet man die Koordinaten der zugeh"origen Ecke
des zul"assigen Gebietes. Die Spalten, die nicht mit $*$ bezeichnet sind,
enthalten nur $0$ und genau eine $1$.

Auf Grund dieses Tableau muss jetzt entschieden
werden, zu welcher Ecke im n"achsten Schritt weitergewandert werden soll,
d.\,h.~welche der Variablen im n"achsten Schritt neu frei w"ahlbar
sein sollen, und welche nicht mehr. 

Jede Variable, f"ur die in der Zeile der Zielfunktion ein negativer
Koeffizient vorliegt, ist ein Austauschkandidat. Welche Variable
man w"ahlt ist eigentlich gleichg"ultig, und es gibt eine breite Literatur
dar"uber, durch welche Wahl die Laufzeit des Algorithmus
optimiert werden kann.
Eine einfache Regel ist, diejenige Variable zu w"ahlen, zu der der
betragsm"assig gr"osste Koeffizient in der Zeile der Zielfunktion
steht. Sie $j$ die Spalte, die neu frei w"ahlbar sein soll.

Jetzt muss eine der bisher abh"angigen Variablen gew"ahlt werden.
Dazu pr"ufen wir jeden m"oglichen Kandidaten $y_k$. Es gibt genau
eine Gleichung $l$, f"ur welche $\tilde a_{lj}=1$ gilt.
\[
\begin{tabular}{|>{$}c<{$}|>{$}c<{$}>{$}c<{$}>{$}c<{$}>{$}c<{$}>{$}c<{$}|>{$}c<{$}|}
\hline
Z&\dots &y_j           &\dots&y_k   &\dots  & \\
\hline
&\dots &\tilde a_{1j} &\dots& 0    & \dots &\tilde b_1\\
&      &\vdots        &     &\vdots&       &\vdots    \\
&\dots &\tilde a_{lj} &\dots& 1    & \dots &\tilde b_l\\
&      &\vdots        &     &\vdots&       &\vdots    \\
&\dots &\tilde a_{mj} &\dots& 0    & \dots &\tilde b_m\\
\hline
&      &*             &     &      &       &          \\
\hline
\end{tabular}
\]
Diese Gleichung legt fest, wie gross $y_j$ werden kann. Wenn $y_k$
neue $=0$ sein soll, dann gilt offenbar
\[
y_j =\frac{\tilde b_{l}}{\tilde a_{lj}}.
\]
Es sind also alle Quotienten $\frac{\tilde b_l}{\tilde a_{lj}}$ auszurechnen
und derjenige Auszuw"ahlen, der den kleinsten positiven Wert f"ur $y_j$ ergibt.
Dann ist jene Variable $y_k$ zu w"ahlen, f"ur die $\tilde a_{lk}=1$.
Dann wird ein Austauschschritt mit dieser Variablen vollzogen:
D.\,h.~es wird die Zeile $l$ durch $\tilde a_{lj}$ geteilt:
\[
\begin{tabular}{|>{$}c<{$}|>{$}c<{$}>{$}c<{$}>{$}c<{$}>{$}c<{$}>{$}c<{$}|>{$}c<{$}|}
\hline
Z&\dots &y_j           &\dots&y_k   &\dots           & \\
\hline
&\dots &\tilde a_{1j} &\dots& 0    & \dots          &\tilde b_1\\
&      &\vdots        &     &\vdots&                &\vdots    \\
&\dots &1             &\dots&\frac{1}{\tilde a_{lj}}& \dots &\frac{\tilde b_l}{\tilde a_{lj}}\\
&      &\vdots        &     &\vdots&                &\vdots    \\
&\dots &\tilde a_{mj} &\dots& 0    & \dots          &\tilde b_m\\
\hline
\end{tabular}
\]
Dann wird in jeder anderen Zeile das $\tilde a_{kj}$-fache der $l$-ten
Zeile subtrahiert, analog auch in der Zeile der Zielfunktion:
\[
\begin{tabular}{|>{$}c<{$}|>{$}c<{$}>{$}c<{$}>{$}c<{$}>{$}c<{$}>{$}c<{$}|>{$}c<{$}|}
\hline
Z&\dots &y_j           &\dots&y_k   &\dots           & \\
\hline
&\dots &0             &\dots&\tilde c_k-\frac{\tilde c_j}{\tilde a_{lj}}&\dots    &\tilde Z-\frac{\tilde c_j}{\tilde a_{lj}}\tilde b_l\\
\hline
&\dots &0             &\dots&-\frac{\tilde a_{1j}}{\tilde a_{lj}}& \dots          &\tilde b_1-\frac{\tilde a_{1j}}{\tilde a_{lj}}\tilde b_l\\
&      &\vdots        &     &\vdots&                            &\vdots    \\
&\dots &1             &\dots&\frac{1}{\tilde a_{lj}}            & \dots &\frac{\tilde b_l}{\tilde a_{lj}}\\
&      &\vdots        &     &\vdots&                            &\vdots    \\
&\dots &0             &\dots&-\frac{\tilde a_{mj}}{\tilde a_{lj}}& \dots          &\tilde b_m-\frac{\tilde a_{mj}}{\tilde a_{lj}}\tilde b_l\\
\hline
&      &              &     &       *                            &                &\\
\hline
\end{tabular}
\]
Damit ist jetzt die Variable $y_k$ eine frei w"ahlbare Variable geworden.

\section{Anfangsecke\label{lp:section:anfang}}
\index{Anfangsecke}
Der Algorithmus im letzten Abschnitt hat nur funktioniert, weil wir
davon ausgehen konnten, dass wir zu Beginn die Variablen $x_1=\dots=x_n=0$
setzen k"onnen, ohne das zul"assige Gebiet zu verlassen. Wir befinden
uns also zu Beginn mit dieser Wahl bereits auf einer Ecke.

\begin{figure}
\begin{center}
\includegraphics{images/lp-4.pdf}
\caption{Zul"assiges Gebiet des Beispiels
(\ref{lp:gebiet2quadrant-ungl}), es git keine L"osung mit $x\ge0$ und $y\ge 0$,
und die Ecke $(0,0)$ ist nicht zul"assig.
\label{lp:gebiet2quadrant}}
\end{center}
\end{figure}
Im allgemeinen wird dies nicht so sein. Das Gebiet beschrieben durch
die Ungleichungen
\begin{equation}
\begin{linsys}{2}
-x_1&-&x_2&\le&1\\
 x_1& &   &\le&-1\\
    & &x_2&\le&1
\end{linsys}
\label{lp:gebiet2quadrant-ungl}
\end{equation}
enth"alt zum Beispiel den Punkt $(0,0)$ nicht, siehe
Abbildung~\ref{lp:gebiet2quadrant}. Ausserdem ist es vollst"andig
im zweiten Quadranten enthalten, es gibt also gar keine zul"assigen 
Punkte mit $x_1\ge 0$ und $x_2\ge 0$. 

Das Bespiel illustriert, dass wir mit folgenden Problemen fertig werden
m"ussen:
\begin{compactenum}
\item Das Gebiet ist nicht in $x_i\ge 0$ enthalten.
\item Der Nullpunkt ist kein zul"assiger Punkt.
\end{compactenum}

\subsection{Negative Variablen}
Unsere Beschreibung des Simplex-Algorithmus ging davon aus, dass die
Variablen $x_i\ge 0$ sind.
Im Allgemeinen k"onnen die Variablen aber jedes beliebige Vorzeichen
haben.
W"ussten wir bereits, dass zum Beispiel die Variable $x_3$ negativ ist,
k"onnten wir sie durch $-x_3$ ersetzen.
Da wir dies aber noch nicht wissen, verdoppeln wir die Anzahl der
Variablen, und lassen sie je in einer positiven und negativen
Variante zu. Die Variablen sind jetzt also
$(y_1,\dots,y_n,z_1,\dots,z_n)$, wobei $y_i\ge 0$ und $z_i\ge 0$.
Die urspr"unglichen Variablen erh"alt man dann als Differenz
$x=y-z$ zur"uck.
Die Zielfunktion ist dann $c^tx=c^t(y-z)$ und die Ungleichungen
werden $A(y-z)=Ay-Az\le b$. Wir erhalten also ein neues lineares Programm
\begin{equation}
\max
\left\{
\begin{pmatrix}c^t&-c^t\end{pmatrix}
\begin{pmatrix}y\\z\end{pmatrix}\,\left|\,
\begin{pmatrix}A&-A\end{pmatrix}\begin{pmatrix}y\\z\end{pmatrix}\le 0, y\ge 0, z\ge 0
\right.
\right\}.
\label{lp:nonegative}
\end{equation}
Dieses lineare Programm f"uhrt auf die gleichen Extrema wie das urspr"ungliche
lineare Programm. Indem wir notfalls die Transformation 
(\ref{lp:nonegative}) durchf"uhren d"urfen wir also annehmen, dass
das zu l"osende lineare Programm eine L"osung mit $x_i\ge 0\forall i$ hat.

\begin{beispiel}
F"ur das Beispiel (\ref{lp:gebiet2quadrant-ungl}) ergeben sich neu die 
Ungleichungen
\begin{equation}
\begin{linsys}{4}
-y_1&-&y_2&+&z_1&+&z_2&\le&1\\
 y_1& &   &-&z_1& &   &\le&-1\\
    & &y_2& &   &-&z_2&\le&1
\end{linsys}
\qquad
\Leftrightarrow
\qquad
\begin{pmatrix}
-1&-1&1&1\\
1&0&-1&0\\
0&1&0&-1
\end{pmatrix}
\begin{pmatrix}y_1\\y_2\\z_1\\z_2\end{pmatrix}
\le 
\begin{pmatrix}1\\-1\\1\end{pmatrix}
\notag
\end{equation}
\end{beispiel}

\begin{figure}
\begin{center}
\includegraphics{images/lp-5.pdf}
\end{center}
\caption{Transformation eines Gebietes mit negativen Koordinaten
in ein Gebiet doppelter Dimension mit positiven Koordinaten.
Aus dem eindimesionalen Gebiet $G_1$ wird in einem ersten
Schritt das zweidimensionale Gebiet $G_2$.
\label{lp:transform-negative1}}
\end{figure}
\begin{figure}
\begin{center}
\includegraphics{images/lp-6.pdf}
\end{center}
\caption{Transformation eines Gebietes mit negativen Koordinaten:
Einschr"ankung auf den Bereich $x_1\ge 0$ und $x_2\ge 0$.
Rot eingezeichnet die Geraden der auf zwei Dimensionen erweiterten
Zielfunktion f"ur die F"alle $c_1>0$ und $c_1<0$.
\label{lp:transform-negative2}}
\end{figure}
Um besser verstehen zu k"onnen, was diese Transformation
geometrisch bedeutet, betrachten wir das System mit nur einer
Variablen
\begin{equation}
\begin{linsys}{1}
 x_1&\le&-1\\
-x_1&\le&2
\end{linsys}
\end{equation}
Zul"assig ist nur das Intervall $G_1=[-2,-1]$, in
Abbildung~\ref{lp:transform-negative1} rot eingezeichnet.
Nach der Transformation haben wir ein System mit zwei Variablen:
\begin{equation}
\begin{linsys}{2}
 x_1&-&x_2&\le&-1\\
-x_1&+&x_2&\le&2
\end{linsys}
\end{equation}
mit einem Gebiet $G_2$, welches auch einen Teil im ersten Quadranten
umfasst. Allerdings ist das Gebiet jetzt unbeschr"ankt, und hat keine
Ecken, also auch keine L"osungen eines linearen Programmes mit welcher
Zielfunktion auch immer.

Wenn jetzt zus"atzlich gefordert wird, dass $x_i\ge 0$ sein muss,
erhalten wir ein Gebiet, welches nicht nur im ersten Quadranten enthalten
ist, sondern auch Ecken und damit m"oglicherweise L"osungen eines hat
(Abbildung~\ref{lp:transform-negative2}).

Die Zielfunktion in diesem eindimensionalen Problem hat nur einen
Koeffizienten, der positiv oder negativ sein kann. Ein positiver 
Koeffizient $c_1>0$ bedeutet, dass das Optimum f"ur $x_1=-1$ angenommen
wird, bei einem negativen Koeffizienten wird es f"ur $x_2=-2$ angenommen.

Nach der Transformation hat die Zielfunktion die Koeffizienten
$(c_1,-c_1)$. F"ur $c_1>0$ zeigt die Normale der Zielfunktionsgeraden
nach rechts unten, also sind alle Punkte der Form $(x_1,x_1+1)$ L"osungen
des Optimierungsproblems. Da die gesuchte L"osung die Differenz der L"osungen
im erweiterten System ist, gibt es genau eine L"osung bei $-1$.
Ist umgekehrt $c_1 < 0$, dann zeigt die Normale der
Zielfunktionsgeraden in Richtung $(-1,1)$, das Optimum wird in den
Punkten $(x_1,x_1+2)$ angenommen, also ist die L"osung des urspr"unglichen
Problems bei $-2$ zu finden.

\subsection{Nullpunkt nicht zul"assig\label{lp:zero-not-admissible}}
Die Ungleichungen
\begin{equation}
\begin{linsys}{2}
 x_1& &   &\le 1\\
    & &x_2&\le 1\\
-x_1&-&x_2&\le -1
\end{linsys}
\label{lp:zero-not-admissible-example}
\end{equation}
lassen den Punkt $(0,0)$ nicht zu. Ursache daf"ur ist offenbar die letzte
Gleichung. Setzt man $(0,0)$ ein, entsteht auf der linken Seite immer $0$,
eine Ungleichung mit einer negativen rechten Seite kann also niemals
erf"ullt sein.

Wir k"onnen die Ungleichungen $Ax\le b$ in zwei Teilmengen aufteilen.
Die ersten $m'$ Ungleichungen haben rechte Seiten $\ge 0$. Die 
verbleibenden $m''$ Ungleichungen haben rechte Seiten $<0$. Wir k"onnen
das so schreiben:
\begin{equation}
Ax\le b \qquad\Leftrightarrow\qquad
\left\{
\begin{aligned}
 A'x&\le b',  &b'&\ge 0,&&\text{$m'$ Ungleichungen}\\
A''x&\le b'', &b''& < 0,&&\text{$m''$ Ungleichungen}
\end{aligned}
\right.
\end{equation}
Das Ziel ist, eine zul"assige Ecke f"ur das Gebiet $Ax\le b$ zu finden.
Wie bereits erw"ahnt, m"ussen dazu auch einige der Variablen von $0$
verschieden sein, damit die Ungleichungen $A''x\le b''$ ebenfalls erf"ullt
sind.

Die Differenz $y=b''-A''x$ wird bei einem zul"assigen Punkt positiv sein,
wir suchen also eine L"osung des Ungleichungssystems
\begin{equation}
\begin{aligned}
A'x&\le b',&x&\ge 0\\
A''x+y&\ge b'',&y&\ge 0
\end{aligned}
\label{lp:first-corner-extended}
\end{equation}
Dadurch wurde zwar die Anzahl der Unbekannten von $n$ auf $n+m''$ erh"oht,
daf"ur ist jetzt der Nullpunkt ein zul"assiger Punkt.

Nicht jeder zul"assige Punkt von
(\ref{lp:first-corner-extended}) 
liefert auch einen zul"assigen Punkt von $Ax\le b$, der Nullpunkt
liefert ein Gegenbeispiel.
Die Menge
\[
\{ x\, |\, \text{$(x,y)$ ist zul"assig in 
(\ref{lp:first-corner-extended}) }\}
\]
ist ein Schnitt durch das
Gebiet (\ref{lp:first-corner-extended}).
Ecken dieses Gebietes kommen also im Allgemeinen nur von Kanten
von (\ref{lp:first-corner-extended}) her, wir m"ussen also eine
ganz spezielle Ecke
des Gebietes (\ref{lp:first-corner-extended}) ausw"ahlen.

Wir bestimmen mit dem Simplex-Algorithmus eine Ecke von
(\ref{lp:first-corner-extended}), die 
\begin{equation}
e^t (A''x + y) \ge e^t b''
\label{lp:first-corner-objective}
\end{equation}
minimiert, wobei $e$ ein Vektor aus $m''$ Einsen ist.
Die rechte Seite ist also einfach die Summe der Elemente von $b''$.
Da also $e^t(A''x+y)$ nach unten beschr"ankt ist, kann es nicht
beliebig klein werden, es muss ein Minimum geben.
Da der Nullpunkt
ein zul"assiger Punkt von (\ref{lp:first-corner-extended}) ist,
k"onnen wir den Simplex-Algorithmus in der bereits
dargestellten Form verwenden.

Sei jetzt also  $(x,y)$ ein Minimum von
(\ref{lp:first-corner-objective})
unter den Bedingungen (\ref{lp:first-corner-extended}).
Ein solche L"osung muss ein Ecke des Gebietes sein.
Ob eine Ecke vorliegt erkennt man daran, wieviele Ungleichungen
exakt erf"ullt sind. 
von den $m+n+m''$ Ungleichungen in $n+m''$ Variablen m"ussen $n+m''$
exakt erf"ullt sein.

Es muss aber auch $A''x+y=b''$ gelten.
W"are n"amlich eine der Ungleichungen, zum Beispiel die mit der Nummer $i$,
nicht exakt erf"ullt, dann k"onnte man das entsprechende $y_i$ noch
etwas kleiner machen, die Ungleichungen w"aren immer noch erf"ullt.
Dabei w"urde auch die Zielfunktion
(\ref{lp:first-corner-objective})
nochmals kleiner, wir h"atten also noch gar nicht das Minimum gehabt.
Also sind alle $m''$ Ungleichungen $A''x+y\ge b''$ als Gleichungen
erf"ullt.

Von den $n+m+m''$ Ungleichungen 
(\ref{lp:first-corner-extended})
wissen wir jetzt also, dass die $m''$ Ungleichungen $A''x+y\ge b''$
als Gleichungen erf"ullt sind.
Also m"ussen von den "ubrigen Ungleichungen
deren $n+m$ als Gleichungen erf"ullt sein.

Es bleiben also jetzt noch die Gleichungen
\begin{align*}
A'x&\le b' & m'&\;\text{Ungleichungen}\\
x  &\ge 0  &  n&\;\text{Ungleichungen}\\
y  &\ge 0  &m''&\;\text{Ungleichungen}
\end{align*}
Ist eine der Ungleichungen f"ur $y$ als Gleichung erf"ullt, dann
ist auch die entsprechende Ungleichung aus $A''x\le b''$ als
Gleichung erf"ullt. Denn wegen $y=b''-A''x$ folgt aus $y_i=0$
sofort, dass die $i$-te Zeile in $A''x\le b''$ als Gleichung
erf"ullt ist.

Sind $k$ der Ungleichungen $y\ge 0$ als Gleichungen erf"ullt,
dann auch $k$ der Ungleichungen $A''x\le b$. Ausserdem
m"ussen dann noch $n-k$ Ungleichungen von
\begin{align*}
A'x&\le b' & m'&\;\text{Ungleichungen}\\
x  &\ge 0  &  n&\;\text{Ungleichungen}
\end{align*}
als Gleichungen erf"ullt sein, oder $n=n-k+k$ von
\begin{align*}
A'x&\le b' & m'&\;\text{Ungleichungen}\\
A''x\le b''&m''&\;\text{Ungleichungen}\\
x  &\ge 0  &  n&\;\text{Ungleichungen}
\end{align*}
Es sind also genau $n$ der Ungleichungen des urspr"unglichen Systems
als Gleichungen erf"ullt, was eine Ecke des urspr"unglichen Gebietes
charakterisiert.

\begin{beispiel}
Man finde das Maximum von $Z=2x_1-x_2$ unter den Bedingungen
\begin{equation}
\begin{linsys}{2}
x_1 & &    &\le& 2\\
    & & x_2&\le& 1\\
-x_1&-&2x_2&\le&-2
\end{linsys}
\end{equation}
und $x_i\ge 0$.
Wegen der letzten Gleichung ist $(0,0)$ kein zul"assiger Punkt, 
wir m"ussen also den in diesem Abschnitt beschriebenen Algorithmus
anwenden, um einen zul"assigen Punkt zu finden.

Die Matrizen $A'$ und $A''$ und die Vektoren $b'$ und $b''$  sind 
\begin{align*}
A'&=\begin{pmatrix}
1&0\\
0&1
\end{pmatrix}
&
b'&=\begin{pmatrix}
2\\1
\end{pmatrix}
\\
A''&=\begin{pmatrix}
-1&-2
\end{pmatrix}
&
b''&=
\begin{pmatrix}-2\end{pmatrix}
\end{align*}
Das modifizierte Problem hat also zus"atzlich eine Variable $y$.
Die Zielfunktion des modifizierten Systems ist
\[
e^t A''\begin{pmatrix}x_1\\x_2\end{pmatrix}
+
e^t y
=-x_1-2x_2+y
\]
Sie ist zu minimieren oder es ist
\[
x_1+2x_2-y
\]
zu maximieren unter den Bedingungen
\[
\begin{linsys}{3}
 x_1& &    & &   &\le& 2\\
    & & x_2& &   &\le& 1\\
-x_1&-&2x_2&+&  y&\ge& -2
\end{linsys}
\]
Durch Umkehren des Vorzeichens in der letzten Gleichung erreichen
wir die Standardform
\begin{equation}
\begin{linsys}{3}
 x_1& &    & &   &\le 2\\
    & & x_2& &   &\le 1\\
 x_1&+&2x_2&-&  y&\le 2
\end{linsys}
\end{equation}
Ausserdem gelten die Bedingungen $x_i\ge 0$ und $y\ge 0$. 

Auf dieses Problem wenden wir jetzt den Simplex-Algorithmus an.
Dazu brauchen wir drei Schlupfvariablen $s_1$, $s_2$ und $s_3$, an
denen wir sp"ater erkennen k"onnen, welche Ungleichungen als
Gleichungen erf"ullt sind.
Das Ausgangstableau ist also
\[
\begin{tabular}{|>{$}c<{$}|>{$}c<{$}>{$}c<{$}>{$}c<{$}|>{$}c<{$}>{$}c<{$}>{$}c<{$}|>{$}c<{$}|}
\hline
Z&x_1&x_2& y&s_1&s_2&s_3& b\\
\hline
1& -1& -2& 1&  0&  0&  0& 0\\
\hline
0&  1&  0& 0&  1&  0&  0& 2\\
0&  0&  1& 0&  0&  1&  0& 1\\
0&  1&  2&-1&  0&  0&  1& 2\\
\hline
 &  *&  *& *&   &   &   &  \\
\hline
\end{tabular}
\]
Im ersten Schritt wird $x_2$ gegen $s_2$ ausgetauscht:
\[
\begin{tabular}{|>{$}c<{$}|>{$}c<{$}>{$}c<{$}>{$}c<{$}|>{$}c<{$}>{$}c<{$}>{$}c<{$}|>{$}c<{$}|}
\hline
Z&x_1&x_2& y&s_1&s_2&s_3& b\\
\hline
1& -1&  0& 1&  0&  2&  0& 2\\
\hline
0&  1&  0& 0&  1&  0&  0& 2\\
0&  0&  1& 0&  0&  1&  0& 1\\
0&  1&  0&-1&  0& -2&  1& 0\\
\hline
 &  *&   & *&   &  *&   &  \\
\hline
\end{tabular}
\]
Im zweiten Schritt wird $x_1$ gegen $s_1$ ausgetauscht:
\[
\begin{tabular}{|>{$}c<{$}|>{$}c<{$}>{$}c<{$}>{$}c<{$}|>{$}c<{$}>{$}c<{$}>{$}c<{$}|>{$}c<{$}|}
\hline
Z&x_1&x_2& y&s_1&s_2&s_3& b\\
\hline
1&  0&  0& 1&  1&  2&  0& 4\\
\hline
0&  1&  0& 0&  1&  0&  0& 2\\
0&  0&  1& 0&  0&  1&  0& 1\\
0&  0&  0&-1& -1& -2&  1&-2\\
\hline
 &   &   & *&  *&  *&   &  \\
\hline
\end{tabular}
\]
Daraus lesen wir ab, dass der Punkt $(2,1)$ ein zul"assiger Punkt ist, der
als Anfangspunkt dienen kann.
Dieser Punkt wird beschrieben durch $s_1=0$ und $s_2=0$.

Damit k"onnen wir jetzt das Simplex-Tableau f"ur das urspr"ungliche
Problem aufstellen
\[
\begin{tabular}{|>{$}c<{$}|>{$}c<{$}>{$}c<{$}|>{$}c<{$}>{$}c<{$}>{$}c<{$}|>{$}c<{$}|}
\hline
Z&x_1&x_2&s_1&s_2&s_3& b\\
\hline
1& -2&  1&  0&  0&  0& 0\\
\hline
0&  1&  0&  1&  0&  0& 2\\
0&  0&  1&  0&  1&  0& 1\\
0& -1& -2&  0&  0&  1&-2\\
\hline
\end{tabular}
\]
Wir wollen allerdings $s_1=0$ und $s_2=0$  setzen, d.~h. wir m"ussen
die Gleichungen umformen, so dass die Variablen $x_1$, $x_2$ und $s_3$ 
durch diese bestimmt sind:
\[
\begin{tabular}{|>{$}c<{$}|>{$}c<{$}>{$}c<{$}|>{$}c<{$}>{$}c<{$}>{$}c<{$}|>{$}c<{$}|}
\hline
Z&x_1&x_2&s_1&s_2&s_3& b\\
\hline
1&  0&  0&  2& -1&  0& 3\\
\hline
0&  1&  0&  1&  0&  0& 2\\
0&  0&  1&  0&  1&  0& 1\\
0&  0&  0&  1&  2&  1& 2\\
\hline
 &   &   &  *&  *&   &  \\
\hline
\end{tabular}
\]
Offenbar kann das Resultat noch verbessert werden, indem man $s_2$
gegen eine andere Variable austauscht. Das "ubliche Kriterium f"uhrt
auf $x_2$ als auszutauschende Variable, der Austauschschritt ergibt
\[
\begin{tabular}{|>{$}c<{$}|>{$}c<{$}>{$}c<{$}|>{$}c<{$}>{$}c<{$}>{$}c<{$}|>{$}c<{$}|}
\hline
Z&x_1&x_2&s_1&s_2&s_3& b\\
\hline
1&  0&  1&  2&  0&  0& 4\\
\hline
0&  1&  0&  1&  0&  0& 2\\
0&  0&  1&  0&  1&  0& 1\\
0&  0& -2&  1&  0&  1& 0\\
\hline
 &   &  *&  *&   &   &  \\
\hline
\end{tabular}
\]
Damit ist das Optimum im Punkt $(2, 0)$ mit maximalem Wert $4$ der Zielfunktion
gefunden.
\end{beispiel}

\section{Das duale Problem}
Das duale Problem hat bereits die Form, die wir f"ur die Durchf"uhrung des
Simplex-Algorithmus brauchen: eine Menge von Gleichungen, und die
einzigen Ungleichungen sind $\mu_i\ge 0$.
Wir brauchen also nur einen Startpunkt, dann kann der Simplex-Algorithmus
genau gleich wie beim urspr"unglichen Problem durchgef"uhrt werden.

Setzt man im dualen Problem alle $\mu_i=0$, entsteht auf der linken Seite
der Gleichungen $0$.
Der Nullpunkt ist also im allgemeinen kein zul"assiger Punkt.
Da es sich um Gleichungen, und nicht um Ungleichungen handelt, ist
der in Abschnitt~\ref{lp:zero-not-admissible} nicht direkt anwendbar.

Der Gleichung $A^t\mu=c$ entsprechen aber zwei Ungleichungen
$Ax\le b$ und $A^t\mu \ge c$, letztere wird zur Standardform wenn wir
das Vorzeichen kehren: $-A^t\mu\le -c$.
Das duale Problem k"onnen wir also wie folgt l"osen:
\begin{enumerate}
\item Wende den Algorithmus von Abschnitt~\ref{lp:zero-not-admissible} auf
das System
\begin{align*}
A^t \mu&\le c\\
-A^t\mu&\le -c
\end{align*}
an, um einen zul"assigen Punkt zu finden.
\item Wende den Simplexalgorithmus auf das duale Problem an, wobei
der im ersten Schritt gefundene zul"assige Punkt als Startpunkt verwendet
werden muss.
\end{enumerate}
Wir zeigen das Vorgehen am Beispiel des bereits mehrfach untersuchten
Einf"uhrungsbeispiels.

\begin{beispiel} Man l"ose das duale lineare Programm 
(\ref{dual-lp-zielfunktion})
und
(\ref{lp-dual-gleichungen}).

Zun"achst m"ussen wir also einen zul"assigen Punkt f"ur die Gleichungen
(\ref{lp-dual-gleichungen}) finden.
Das Ungleichungssystem dazu ist
\begin{equation}
\begin{linsys}{5}
- \mu_1&+& \mu_2&+&4\mu_3&-&\mu_4& &      &\le&2\\
 3\mu_1&+&3\mu_2&+& \mu_3& &     &-& \mu_5&\le&3\\
  \mu_1&-& \mu_2&-&4\mu_3&+&\mu_4& &      &\le&-2\\
-3\mu_1&-&3\mu_2&-& \mu_3& &     &+& \mu_5&\le&-3\\
\end{linsys}
\end{equation}
Problematisch sind nur die beiden letzten Gleichungen, wir haben
\begin{align*}
A'&=\begin{pmatrix}
-1& 1& 4&-1& 0\\
 3& 3& 1& 0&-1
\end{pmatrix}
&
b'&=\begin{pmatrix}2\\3\end{pmatrix}
\\
A''&=\begin{pmatrix}
 1&-1&-4& 1& 0\\
-3&-3&-1& 0& 1
\end{pmatrix}
&
b'&=\begin{pmatrix}-2\\-3\end{pmatrix}
\end{align*}
Das modifizierte Problem besteht also darin, die Zielfunktion
\[
e^tA''x+e^ty=
-2\mu_1-4\mu_2-5\mu_3+\mu_4+\mu_5
+y_1+y_2
\]
zu minimieren unter den Bedingungen
\begin{equation}
\begin{linsys}{7}
- \mu_1&+& \mu_2&+&4\mu_3&-&\mu_4& &      & &   & &   &\le&2\\
 3\mu_1&+&3\mu_2&+& \mu_3& &     &-& \mu_5& &   & &   &\le&3\\
- \mu_1&+& \mu_2&+&4\mu_3&-&\mu_4& &      &-&y_1& &   &\le&2\\
 3\mu_1&+&3\mu_2&+& \mu_3& &     &-& \mu_5& &   &-&y_2&\le&3\\
\end{linsys}
\end{equation}
Das zugeh"orige Simplex-Tableau ist
\[
\begin{tabular}{|>{$}c<{$}|>{$}c<{$}>{$}c<{$}>{$}c<{$}>{$}c<{$}>{$}c<{$}>{$}c<{$}>{$}c<{$}|>{$}c<{$}>{$}c<{$}>{$}c<{$}>{$}c<{$}|>{$}c<{$}|}
\hline
Z&\mu_1&\mu_2&\mu_3&\mu_4&\mu_5&y_1&y_2&s_1&s_2&s_3&s_4&b\\
\hline
-1&  -2&   -4&   -5&    1&    1&  1&  1&  0&  0&  0&  0&0\\
\hline
 0&  -1&    1&    4&   -1&    0&  0&  0&  1&  0&  0&  0&2\\
 0&   3&    3&    1&    0&   -1&  0&  0&  0&  1&  0&  0&3\\
 0&  -1&    1&    4&   -1&    0& -1&  0&  0&  0&  1&  0&2\\
 0&   3&    3&    1&    0&   -1&  0& -1&  0&  0&  0&  1&3\\
\hline
  &   *&    *&    *&    *&    *&  *&  *&   &   &   &   & \\
\hline
\end{tabular}
\]
Im ersten Schritt tauschen wir $\mu_3$ gegen $s_1$ aus.
\[
\begin{tabular}{|>{$}c<{$}|>{$}c<{$}>{$}c<{$}>{$}c<{$}>{$}c<{$}>{$}c<{$}>{$}c<{$}>{$}c<{$}|>{$}c<{$}>{$}c<{$}>{$}c<{$}>{$}c<{$}|>{$}c<{$}|}
\hline
Z&       \mu_1&      \mu_2&\mu_3&   \mu_4&\mu_5&y_1&y_2&     s_1&s_2&s_3&s_4&      b\\
\hline
-1&-\frac{13}4&-\frac{11}4&    0&-\frac14&    1&  1&  1& \frac54&  0&  0&  0&\frac52\\
\hline
 0&   -\frac14&    \frac14&    1&-\frac14&    0&  0&  0& \frac14&  0&  0&  0&\frac12\\
 0& \frac{13}4& \frac{11}4&    0& \frac14&   -1&  0&  0&-\frac14&  1&  0&  0&\frac52\\
 0&          0&          0&    0&       0&    0& -1&  0&      -1&  0&  1&  0&      0\\
 0& \frac{13}4& \frac{11}4&    0& \frac14&   -1&  0& -1&-\frac14&  0&  0&  1&\frac52\\
\hline
  &          *&          *&     &       *&    *&  *&  *&       *&   &   &   &       \\
\hline
\end{tabular}
\]
Im zweiten Schritt tauschen wir $\mu_1$ gegen $s_2$ aus:
\[
\begin{tabular}{|>{$}c<{$}|>{$}c<{$}>{$}c<{$}>{$}c<{$}>{$}c<{$}>{$}c<{$}>{$}c<{$}>{$}c<{$}|>{$}c<{$}>{$}c<{$}>{$}c<{$}>{$}c<{$}|>{$}c<{$}|}
\hline
Z&         \mu_1&        \mu_2&\mu_3&       \mu_4&        \mu_5&y_1&y_2&          s_1&         s_2&s_3&s_4&      b\\
\hline
-1&           0&             0&    0&           0&            0&  1&  1&            1&           1&  0&  0&      5\\
\hline
 0&           0& \frac{6}{13}&    1&-\frac{3}{13}&-\frac{1}{13}&  0&  0& \frac{3}{13}&\frac{1}{13}&  0&  0&\frac{9}{13}\\
 0&           1&\frac{11}{13}&    0& \frac{1}{13}&-\frac{4}{13}&  0&  0&-\frac{1}{13}&\frac{4}{13}&  0&  0&\frac{10}{13}\\
 0&           0&            0&    0&            0&            0& -1&  0&           -1&           0&  1&  0&      0\\
 0&           0&            0&    0&            0&            0&  0& -1&            0&          -1&  0&  1&      0\\
\hline
  &            &             *&     &           *&            *&  *&  *&            *&           *&   &   &       \\
\hline
\end{tabular}
\]
Damit ist eine Ecke gefunden, man kann also das urspr"ungliche duale Problem ausgehend
von einer Ecke l"osen, die durch $\mu_2=\mu_4=\mu_5=0$ gegeben wird. 

Da das duale Problem keine 
Ungleichungen ausser den $\mu_i\ge 0$ hat, braucht es auch keine Schlupfvariablen.
Das Starttableau ist 
\[
\begin{tabular}{|>{$}c<{$}|>{$}c<{$}>{$}c<{$}>{$}c<{$}>{$}c<{$}>{$}c<{$}|>{$}c<{$}|}
\hline
 Z&\mu_1&\mu_2&\mu_3&\mu_4&\mu_5&b\\
\hline
-1&   12&   18&   28&    0&    0&0\\
\hline
  &   -1&    1&    4&   -4&    0&2\\
  &    3&    3&    1&    0&   -1&3\\
\hline
\end{tabular}
\]
aber es ist noch so umzuformen, dass die Variablen $\mu_2$, $\mu_4$ und $\mu_5$ frei
w"ahlbar sind:
\[
\begin{tabular}{|>{$}c<{$}|>{$}c<{$}>{$}c<{$}>{$}c<{$}>{$}c<{$}>{$}c<{$}|>{$}c<{$}|}
\hline
 Z&\mu_1&\mu_2&\mu_3&\mu_4&\mu_5& b\\
\hline
-1&    0&   30&   76&  -48&    0&24\\
\hline
  &    1&   -1&   -4&    4&    0&-2\\
  &    0&    6&   13&  -12&   -1& 9\\
\hline
  &     &     &     &     &     &  \\
\hline
\end{tabular}
\rightarrow
\begin{tabular}{|>{$}c<{$}|>{$}c<{$}>{$}c<{$}>{$}c<{$}>{$}c<{$}>{$}c<{$}|>{$}c<{$}|}
\hline
 Z&\mu_1&         \mu_2&\mu_3&          \mu_4&        \mu_5&           b\\
\hline
-1&    0&-\frac{66}{13}&    0&\frac{288}{13}&\frac{76}{13}&-\frac{372}{13}\\
\hline
  &    1& \frac{11}{13}&    0&  \frac{4}{13}&-\frac{4}{13}&  \frac{10}{13}\\
  &    0&  \frac{6}{13}&    1&-\frac{12}{13}&-\frac{1}{13}&   \frac{9}{13}\\
\hline
  &    &              *&     &             *&            *&               \\
\hline
\end{tabular}
\]
Offenbar kann man das Resultat verbessern, indem man $\mu_2$ gegen $\mu_1$ austauscht:
\[
\begin{tabular}{|>{$}c<{$}|>{$}c<{$}>{$}c<{$}>{$}c<{$}>{$}c<{$}>{$}c<{$}|>{$}c<{$}|}
\hline
 Z&        \mu_1&         \mu_2&\mu_3&          \mu_4&       \mu_5&              b\\
\hline
-1&            6&             0&    0&            24&            4&            -24\\
\hline
  &\frac{13}{11}&             1&    0&  \frac{4}{11}&-\frac{4}{11}&  \frac{10}{11}\\
  &-\frac{6}{11}&             0&    1&-\frac{12}{11}&-\frac{1}{11}&   \frac{3}{11}\\
\hline
  &            *&              &     &             *&            *&               \\
\hline
\end{tabular}
\]
Damit ist das Optimum gefunden.
Es wird f"ur $\mu_2=\frac{10}{11}$ und $\mu_3=\frac{3}{11}$ erreicht,
der minimale Wert ist $24$. 
\end{beispiel}

\section{Zusammenfassung: Das Wichtigste in K"urze}
\rhead{Zusammenfassung}
\begin{itemize}
\item Ein lineares Programm ist ein Optimierungsproblem mit linearer
Zielfunktion
und einem zul"assigen Gebiet beschrieben durch lineare Gleichungen und/oder 
Ungleichungen (Abschnitt~\ref{lp:section:problem}).
\item Die graphische L"osung eines zweidimensionalen linearen Programms
suggeriert, dass L"osungen Ecken des zul"assigen Gebietes sind
(Abschnitt~\ref{lp:section:graphisch})
\item Zu jedem linearen Programm gibt es ein duales lineares Programm.
Eine L"osung des dualen linearen Programms f"uhrt in wenigen Schritten auch
zur L"osung des urspr"unglichen Problems.
Von $0$ verschiedene Variablen der L"osung des dualen Problems geben an,
welche Ungleichungen des urspr"unglichen Problems als Gleichungen erf"ullt sind
(Abschnitt~\ref{lp:section:dual}).
\item Schlupfvariablen erlauben, ein System von Ungleichungen in ein System von
Gleichungen umzuwandeln (Abschnitt~\ref{lp:section:ungleichungen}).
\item Variablen, die Null gesetzt sind, entsprechen Ungleichungen, die
als Gleichungen erf"ullt sind. Sind gen"ugend Variablen Null gesetzt,
lassen sich die Gleichungen eindeutig l"osen und beschreiben daher eine
Ecke des zul"assigen Gebiets (Abschnitt~\ref{lp:section:ungleichungen}).
\item Gauss-Eliminations-Schritte erlauben, die Menge der Variablen zu
ver"andern, die man frei w"ahlen kann, und insbesondere auch Nullsetzen
m"ochte (Abschnitt~\ref{lp:subsection:austausch}).
\item Der Simplex-Algorithmus liefert ausgehend von einer Ausgangsecke des
zul"assigen Gebietes die L"osung eines linearen Programms, indem er
eine Regel angibt, wie man von einer Ecke zu einer im Sinne der Zielfunktion
besseren Nachbarecke kommen kann (Abschnitt~\ref{lp:section:simplex}).
\item Zu jedem linearen Programm kann man ein modifiziertes Programm
samt Anfangsecke finden, welches als L"osung, die zum Beispiel mit dem
Simplex-Algorithmus gefunden werden kann, einen Anfangsecke des 
urspr"unglichen Problems hat (Abschnitt~\ref{lp:section:simplex}).
\end{itemize}
