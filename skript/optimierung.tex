%
% optimierung.tex -- Skript ueber Optimierung
%
% (c) 2012 Prof. Dr. Andreas Mueller, HSR
% $Id: ws-skript.tex,v 1.34 2008/11/02 22:46:16 afm Exp $
%
%\documentclass[a4paper,12pt]{book}
\documentclass[a4paper]{book}
\usepackage{geometry}
\geometry{papersize={170mm,220mm},total={130mm,180mm}}
\usepackage{ngerman}
\usepackage{times}
\usepackage{amsmath}
\usepackage{amssymb}
\usepackage{amsfonts}
\usepackage{amsthm}
\usepackage{graphicx}
\usepackage{fancyhdr}
\usepackage{textcomp}
\usepackage[all]{xy}
\usepackage{txfonts}
\usepackage{alltt}
\usepackage{verbatim}
\usepackage{paralist}
\usepackage{makeidx}
\usepackage{array}
\usepackage{hyperref}
\usepackage{tikz}
\usetikzlibrary{arrows,decorations.pathmorphing,positioning,fit,petri}
\usetikzlibrary{calc,intersections,through,backgrounds,graphs}
\usetikzlibrary{patterns,decorations.pathreplacing}
\usetikzlibrary{shapes,snakes,trees}
\usetikzlibrary{decorations.pathreplacing}
\usetikzlibrary{patterns}
\usepackage{listings}
\lstdefinestyle{Matlab}{
  numbers=left,
  belowcaptionskip=1\baselineskip,
  breaklines=true,
  frame=L,
  xleftmargin=\parindent,
  language=Matlab,
  showstringspaces=false,
  basicstyle=\footnotesize\ttfamily,
  keywordstyle=\bfseries\color{green!40!black},
  commentstyle=\itshape\color{purple!40!black},
  identifierstyle=\color{blue},
  stringstyle=\color{orange},
  numberstyle=\ttfamily\tiny
}
\usepackage{caption}
\usepackage{subcaption}
%\usepackage{cite}
\usepackage{standalone}
\usepackage[sorting=none,backend=bibtex]{biblatex}
\addbibresource{partikelschwarm/literature.bib}
\addbibresource{downhill/bib.bib}
\addbibresource{innerepunkte/bib.bib}
\addbibresource{gps/bib.bib}
\addbibresource{descent/bib.bib}
\addbibresource{licht/licht.bib}
\AtEndDocument{\clearpage\ifodd\value{page}\else\null\clearpage\fi}
\makeindex
\begin{document}
\pagestyle{fancy}
%\lhead{Optimierung}
\frontmatter
\newcommand\HRule{\noindent\rule{\linewidth}{1.5pt}}
\begin{titlepage}
\vspace*{\stretch{1}}
\HRule
\vspace*{5pt}
\begin{flushright}
{
\LARGE
Mathematisches Seminar\\
\vspace*{20pt}
\Huge
Optimierung}
\end{flushright}
\HRule
\begin{flushright}
\vspace{60pt}
\Large
Leitung: Andreas M"uller\\
\vspace{40pt}
\Large
Dorian Amiet,
Hannes Badertscher,
Gregor Dengler,\\
Roman Gassmann,
Lukas Loser,
Selina Malacarne,\\
Tabea M\'endez,
Raphael Nestler,
Philip Riedel,\\
Christian Schmid,
Armin Stocklin,
Dario Wikart
\end{flushright}
\vspace*{\stretch{2}}
\begin{center}
Hochschule f"ur Technik, Rapperswil, 2013
\end{center}
\end{titlepage}
\hypersetup{
    colorlinks=true,
    linktoc=all,
    linkcolor=blue
}
\newcounter{beispiel}
\newenvironment{beispiele}{
\bgroup\smallskip\parindent0pt\bf Beispiele\egroup

\begin{list}{\arabic{beispiel}.}
  {\usecounter{beispiel}
  \setlength{\labelsep}{5mm}
  \setlength{\rightmargin}{0pt}
}}{\end{list}}

\newenvironment{teilaufgaben}{
\begin{enumerate}
\renewcommand{\labelenumi}{\alph{enumi})}
}{\end{enumerate}}
% Loesung
\def\swallow#1{
%nothing
}
\newenvironment{loesung}{%
\begin{proof}[L"osung]%
\renewcommand{\qedsymbol}{$\bigcirc$}
}{\end{proof}}
\def\keineloesungen{%
\renewenvironment{loesung}{\swallow\begingroup}{\endgroup}%
}

\newenvironment{beispiel}{%
\begin{proof}[Beispiel]%
\renewcommand{\qedsymbol}{$\bigcirc$}
}{\end{proof}}

\tableofcontents
\newtheorem{satz}{Satz}[chapter]
\newtheorem{hilfssatz}{Hilfssatz}[chapter]
\newtheorem{definition}{Definition}[chapter]
\newtheorem{annahme}{Annahme}[chapter]
\mainmatter
\input vorwort.tex
\part{Grundlegende klassische Optimierungsverfahren}
\input linsys.tex
\input uebersicht.tex
\input lp.tex
\input nlp.tex
\input variation.tex
\part{Weiterf"uhrende Themen zur Optimierung}
\lhead{Weiterf"uhrende Themen}
\def\chapterauthor#1{{\large #1}\bigskip\bigskip}
\input add/simpleximpl.tex
\input innerepunkte/innerepunkte.tex
\input innerepunkte.tex
\input gps/gps.tex
\input partikelschwarm/partikelschwarm.tex
\input sa/sa.tex
\input descent/descent.tex
\input downhill/downhill.tex
\input licht/licht.tex
\lhead{Index}
\rhead{}
%\input genetic/genetic.tex
\input optimierung.ind

\end{document}
