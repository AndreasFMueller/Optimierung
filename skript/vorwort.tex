\chapter*{Vorwort}
Dieses Skript entstand im Rahmen des Mathematischen Seminars
im Fr"uhrjahrssemester 2013 an der Hochschule f"ur Technik Rapperswil.
Die Teilnehmer, Studierende der Abteilung f"ur Elektrotechnik der
HSR, erarbeiteten nach einer Einf"uhrung in das Themengebiet jeweils
einzelne Aspekte des Gebietes in Form einer Seminararbeit, "uber
deren Resultate sie auch in einem Vortrag informierten. 

Im Fr"uhjahr 2013 war das Thema des Seminars ``Optimierung''.
Die Einf"uhrung bestand in einigen Vorlesungsstunden, deren
Inhalt im ersten Teil dieses Skripts zusammengefasst ist.
Die generelle Aufgabenstellung der mathematischen Optimierung
wurde in Kapitel~1 beschrieben.
Im Kapitel~2 wurde der Simplex-Algorithmus zur linearen Optimierung
vorgestellt, wobei auch Gewicht auf das Dualit"atsprinzip gelegt
wurde.
Im Kapitel~3 wurden die Methode der Lagrange-Multiplikatoren
und ihre Erweiterung, die Karush-Kuhn-Tucker-Bedingung hergeleitet
und gezeigt, wie man mit Hilfe des Newton-Verfahrens eine L"osung
derselben finden kann. Im Kapitel~4 wurde schliesslich eine
Einf"uhrung in die Variationsrechnung gegeben.

Im zweiten Teil dieses Skripts kommen dann die Teilnehmer
selbst zu Wort. Ihre Arbeiten wurden jeweils als einzelne
Kapitel mit meist nur typographischen "Anderungen "ubernommen.
Diese weiterf"uhrenden Kapitel sind sehr verschiedenartig.
Sie beleuchten Implementationsaspekte (Kapitel~5, Implementation
des Simplex-Algorithmus), Alternative Optimierungsalgorithmen
(Kapitel~6: Das Innere-Punkte-Verfahren,
Kapitel~9: Partikelschwarm-Optimierung,
Kapitel~10: Simulated Annealing,
Kapitel~11: Simplex Downhill,
Kapitel~12: Genetische Algorithmen) oder Anwendungen
der Algorithmus
(Kapitel~8: GPS als Optimierungsproblem, Kapitel~13: Lichtausbreitung
als Minimalproblem).

In den meisten Arbeiten wurde auch Code zur Demonstration der 
besprochenen Algorithmen und Experimente geschrieben, soweit
m"oglich und sinnvoll wurde dieser Code im Github-Repository
dieses Kurses \url{https://github.com/AndreasFMueller/Optimierung.git}
abgelegt, in anderen F"allen verweisen die Artikel selbst auf
das zugeh"orige Code-Repository.

Im genannten Repository findet sich auch der Source-Code dieses
Skriptes, es wird hier unter einer Creative Commons Lizenz
zur Verf"ugung gestellt.
