%
% gps.tex
%
% (c) Tabea Méndez und Christian Schmid, HSR
%

\chapter{Global Positioning System (GPS) als Optimierungsproblem}
	Das Global Positioning System ist heute weit herum bekannt. Es erm"oglicht die satellitengest"utzte Positionsbestimmung und erreicht aktuell eine garantierte Genauigkeit von 7.8 Metern in 95\% der Messungen \footnote{\url{http://de.wikipedia.org/wiki/Global_Positioning_System}, abgerufen am 24.03.2013}. Dies erlaubt unglaublich vielf"altige Anwendungen, angefangen bei der Navigation, "uber Landesvermessung bis hin zur Analysierung des Zugverhaltens von V"ogeln, welche mit einem kleinen Sender ausger"ustet wurden.\\[0.15cm]	
	Diese Arbeit befasst sich mit der Optimierung der Positionsberechnung anhand der GPS-Daten welche von den Satelliten empfangen werden. Dazu wird zuerst die grundlegende Funktionsweise sowie die technischen und mathematischen Herausforderungen des Systems beschrieben.
	
	\subsection{Funktionsweise des GPS-Systems}
		Das GPS-System basiert auf mehreren Satelliten, welche die Erde auf genau festgelegten Bahnen umkreisen. Diese Bahnen sind so angelegt, dass zu jeder Zeit an jedem Punkt der Erde gen"ugend Satelliten f"ur eine Positionsbestimmung empfangen werden k"onnen. Die Satelliten sind mit (meist mehreren) genauen Atom-Uhren ausgestattet und senden ihre Position innerhalb eines gemeinsamen Koordinatensystems sowie ihre genaue lokale Zeit aus. Aufgrund der bekannten Position und der exakten Systemzeit zum Sendezeitpunkt kann der Empf"anger seine Distanz zu jedem Satelliten bestimmen. Dieser Abstand beschreibt f"ur jeden Satelliten eine Kugel im Koordinatensystem, wobei der Sender im Kugelzentrum liegt. Sind nun drei solche Abst"ande bzw. Kugeln bekannt, so kann sich der Empf"anger nur an den zwei Schnittpunkten der drei Kugeln befinden. Meist kann einer dieser Punkte vernachl"assigt werden, da er entweder unterhalb der Erdoberfl"ache oder weit ausserhalb der Atmosph"are liegt. Die Position des Empf"angers liesse sich also theoretisch mit den Daten von drei Satelliten bestimmen. \\[0.2cm]
		F"ur die Berechnung der Signallaufzeit (und damit des Abstandes vom Empf"anger zum Sender) muss nebst dem Sende- auch der Empfangszeitpunkt exakt bekannt sein. Da die meisten GPS-Empf"anger jedoch nicht "uber eine gen"ugend genaue Zeitbasis verf"ugen, muss neben den drei Raumkoordinaten des Empfangsortes auch die genaue Zeit bestimmt werden. Um diese vier Unbekannten zu berechnen sind also die Signale von mindestens vier Satelliten notwendig. \\[0.2cm]
		Das GPS-System ist so ausgelegt, dass die Positionsbestimmung m"oglichst zu jeder Zeit an jedem Ort m"oglich ist, auch wenn sich der Empf"anger in einem Tal befindet und keine freie Sicht zum Horizont besteht. Deshalb sind immer mindestens 24 Satelliten gleichzeitig aktiv, "ublicherweise mehr. Damit lassen sich meist mehr als nur vier Satelliten f"ur die Positionsbestimmung nutzen. Da aber nur vier Unbekannte zu bestimmen sind wird das Gleichungssystem "uberbestimmt. Mit diesem Überfluss an Informationen l"asst sich die Positionsbestimmung aber markant verbessern, indem m"ogliche Fehler in den einzelnen empfangenen Signalen weniger ins Gewicht fallen. 

	\subsection{Fehlereinfl"usse auf die Positionsberechnung}
		Die Technik des GPS-Systems bietet mehrere Fehlerquellen, welche die exakte Positionsbestimmung behindern k"onnen. Gegenstand dieser Arbeit soll nicht die Verhinderung oder Kompensation dieser Fehler sein, sondern die Optimierung der Berechnung der Position. Dabei ist es unerheblich, woher die Fehlereinfl"usse stammen, sie sollen das Endresultat nur m"oglichst wenig beeinflussen. Sofern die Qualit"at jedes empfangenen Datenpakets bekannt ist kann es bei einem "uberbestimmten System entsprechend mehr oder weniger stark gewichtet werden. Die GPS-Daten welche die Satelliten aussenden enthalten eine solche Qualit"atsangabe, da die Satelliten selbst in der Lage sind, zu erkennen, wie viel sie beispielsweise von ihrer Soll-Position abweichen oder ob ihre Zeitbasis noch korrekt ist. Weiter kann auch der Empf"anger eine Qualit"atsanalyse vornehmen. So l"asst die Signalst"arke eines Signals beispielsweise R"uckschl"usse darauf zu, ob das Signal direkt empfangen oder von einer Oberfl"ache reflektiert wurde. Wird die Empfangsst"arke eines Signals "uber eine Zeit lang beobachtet, so erlaubt auch dies eine Beurteilung des Signalwegs. Da die Position des Senders im Signal mitgeteilt wird k"onnen auch Satelliten, welche tief "uber dem Horizont liegen weniger stark gewichtet werden, da ihr Signal einen weiteren Weg durch die Atmosph"are zur"uckgelegt hat und deshalb die Signallaufzeit weniger genau stimmen d"urfte, als wenn das Signal senkrecht oberhalb des Empf"angers ausgesendet wurde. 

\section{Grundlegende Positionsberechnung}
	F"ur jeden empfangenen Satelliten kann eine Gleichung aufgestellt werden, in welcher seine Position in Relation zum Abstand zum Empf"anger gestellt wird.
	\begin{center}
		\fbox{$(x_i - x)^2 + (y_i - y)^2 + (z_i - z)^2 - (t - t_i)^2\cdot c^2 = 0$}
	\end{center}
	$x_i, y_i, z_i$ stehen dabei f"ur die Raumkoordinaten des Senders, $x, y, z$ f"ur diejenigen des Empf"angers. Die ersten drei Quadrate ergeben aufsummiert den Abstand zwischen Sender und Empf"anger. Das vierte Quadrat wiederum errechnet den Abstand aus der Zeitdifferenz, multipliziert mit der Geschwindigkeit mit welcher sich die Signale ausbreiten. Da diese beiden Distanzen gleich gross sein m"ussen werden sie voneinander subtrahiert und sollen 0 ergeben.\\[0.2cm]
	Sind nun die Daten von nur vier Satelliten bekannt, so kann ein Gleichungssystem mit 4 Gleichungen und 4 Unbekannten aufgestellt und eindeutig gel"ost werden.\begin{center}
	$\begin{array}{l}
	(x_1 - x)^2 + (y_1 - y)^2 + (z_1 - z)^2 - c^2\cdot(t - t_1)^2 ~=~ 0\\
	(x_2 - x)^2 + (y_2 - y)^2 + (z_2 - z)^2 - c^2\cdot(t - t_2)^2 ~=~ 0\\
	(x_3 - x)^2 + (y_3 - y)^2 + (z_3 - z)^2 - c^2\cdot(t - t_3)^2 ~=~ 0\\
	(x_4 - x)^2 + (y_4 - y)^2 + (z_4 - z)^2 - c^2\cdot(t - t_4)^2 ~=~ 0\\
	\end{array}$                                                                                                                              \end{center}
	Werden nun mehr als vier Satelliten empfangen, so ist das Gleichungssystem "uberbestimmt. Damit l"auft die Positionsbestimmung auf ein Optimierungsproblem heraus.\\

	\begin{figure}[ht!]\centering
		\includegraphics[scale = 0.3]{gps/gps.png}
		\caption{Satellitenkonstellation}
		\label{fig. satellitenkonstellation}
	\end{figure}

\section{Optimierungsproblem}
	Wenn die eigene Position mit Hilfe von GPS-Satelliten bestimmt werden soll, k"onnen in den meisten F"allen von mehr als vier Satelliten Daten empfangen werden. Dadurch entsteht ein "uberbestimmtes Gleichungssystem mit $4$ unbekannten Variablen $(x_0,y_0,z_0,t_0)$ und $n$ Gleichungen ($n = $ Anzahl empfangene Satellitendaten). Das Optimierungsproblem liegt dann darin, aus den empfangen Satellitendaten die Position zu finden, die am besten stimmt, bzw. den Ort an dem der quadratische Fehler am kleinsten ist (''method of least squares'').

	\subsection{Zielfunktion}
		Die Zielfunktion, die es dabei zu minimieren gilt, lautet folgendermassen:\\[0.2cm]
		\begin{minipage}{1\textwidth}
			\fbox{$f(x,y,z,t) = \text{\large$\sum\limits_{i=1}^{n}$}\;{\left[\underbrace{(x_i-x)^2 + (y_i-y)^2 + (z_i-z)^2}_{\text{Distanz aus Raumkoordinaten}} \;\, - \underbrace{c^2 (t-t_i)^2}_{\text{Distanz aus Zeit}}\;\right]^2}$}
		\end{minipage}\\[0.2cm]
		\begin{minipage}{1\textwidth}
			\begin{tabular}{ll}
				Anzahl Satelliten: $\quad$& $n$\\
				Satellitendaten: $\quad$& $x_i$, $y_i$, $z_i$ und $t_i$\\
				Lichtgeschwindigkeit: $\quad$& $c$\\
			\end{tabular}
		\end{minipage}\\[0.3cm]
		Bei dieser Zielfunktion werden die Fehler der einzelnen Satelliten quadriert und anschliessend aufsummiert. Durch das Quadrieren wird verhindert, dass sich die Fehler der Satelliten gegenseitig kompensieren k"onnen. Wird nun ein Punkt gefunden an dem die Zielfunktion und damit der quadrierte und aufsummierte Fehler der Satelliten minimal wird, so wird in diesem Punkt auch der Positionsfehler minimal. Diesen Punkt gilt es also zu finden.

	\subsection{Minimum der Zielfunktion finden}\label{Minimum der ZF}
		Um die Extremalstellen (Maxima oder Minima) der Zielfunktion $f$ zu finden, muss der Gradient von $f$ gleich Null gesetzt werden. Der Gradient ist ein Vektor, welcher die ersten partiellen Ableitungen der Zielfunktion beinhaltet und immer in die Richtung des steilsten Anstiegs zeigt. Ist nun der Gradient, und damit der steilste Anstieg in einem Punkt Null, so handelt es sich bei diesem Punkt um ein Maximum, ein Minimum oder einen Sattelpunkt.\\[0.2cm]
		\begin{minipage}{\textwidth}
		$ \nabla f = \begin{bmatrix} \frac{\partial f}{\partial x}\\[0.2cm] \frac{\partial f}{\partial y}\\[0.2cm] \frac{\partial f}{\partial z}\\[0.2cm] \frac{\partial f}{\partial t}\end{bmatrix} %= \begin{bmatrix} f_x \\ f_y \\ f_z \\ f_t \end{bmatrix}
		= \begin{bmatrix} \text{\large$\sum\limits_{i=1}^{n}$}\;{-4\,(x_i-x)\left[(x_i - x)^2 + (y_i - y)^2 + (z_i - z)^2 - c^2(t - t_i)^2 \right]} \\\text{\large$\sum\limits_{i=1}^{n}$}\;{-4\,(y_i-y)\left[(x_i - x)^2 + (y_i - y)^2 + (z_i - z)^2 - c^2(t - t_i)^2 \right]} \\ \text{\large$\sum\limits_{i=1}^{n}$}\;{-4\,(z_i-z)\left[(x_i - x)^2 + (y_i - y)^2 + (z_i - z)^2 - c^2(t - t_i)^2 \right]} \\ \text{\large$\sum\limits_{i=1}^{n}$}\;{-4 c^2(t-t_i)\left[(x_i - x)^2 + (y_i - y)^2 + (z_i - z)^2 - c^2(t - t_i)^2\right]}\end{bmatrix} = \begin{bmatrix} 0 \\ 0 \\ 0 \\ 0 \end{bmatrix}$\\
		\end{minipage}\\[0.2cm]
		Zum l"osen dieses nicht linearen Gleichungssystems ist das Newton-Verfahren geeignet.

	\subsection{Newton-Verfahren}
		Die Idee des Newton-Verfahrens ist, die Funktion fortlaufend durch ihre Linearisierung zu approximieren und dann das vereinfachte lineare Gleichungssystem zu l"osen.\\[0.1cm]
		Zur Illustration wird das Newton-Verfahren vorerst an einem 1-dimensionalen Beispiel gezeigt (Abbildung \ref{fig. 1dim-Bsp-Newton}).\\[0.2cm]
		Als erstes muss ein Startpunkt gew"ahlt werden, in welchem die Funktion $f(x)$ linearisiert wird. Im gezeigten Beispiel ist dies der Punkt $P_0$. Danach wird das linearisierte Gleichungssystem $|g_0(x)=0|$ gel"ost. Dies liefert einen neuen Punkt $P_1$, mit dem das Ganze wiederholt wird. Wiederum die Funktion $f(x)$ im Punkt $P_1$ linearisieren, dann das linearisierte Gleichungssystem $|g_1(x)=0|$ l"osen und wiederum im berechneten neuen Punkt $P_2$ linearisieren. Mit jedem Iterationsschritt wird die L"osung verbessert, wobei der Verbesserungsschritt dabei immer kleiner wird. Die Iteration kann abgebrochen werden, sobald der Verbesserungsschritt gen"ugend klein und damit die L"osung gen"ugend genau ist.\\ 
		\begin{figure}[ht!]\centering
 			\begin{tikzpicture}[>=latex',scale=1.4]
				\draw[->] (0.3,0) -- (4.9,0)node[right] {$x$};
				\draw[->] (0.5,-0.2) -- (0.5,3) node[above] {$y$};
				\draw[smooth,samples=100,domain=0.3:4.8, black, line width=0.75 ] plot (\x,{(0.4*\x-0.1)*(0.4*\x-0.1)-0.3})node[above]{\footnotesize $f(x)$}; 
				\draw[smooth,samples=100,domain=2:4.8, gray, line width=0.5 ] plot (\x,{1.36*(\x-4.5)+2.59})node at(3.35,0.5){\footnotesize $g_0(x)$};
				\filldraw[red!70!black] (4.5,0)circle(1.5pt)nodeat(4.5,-0.3){\footnotesize $P_0$};
				\draw[->, line width=0.5, red!70!black, dashed] (4.5,0)--(4.5,2.59);
				\filldraw[red!70!black] (4.5,2.59)circle(1.5pt);
				\draw[smooth,samples=100,domain=1:4, gray, line width=0.5 ] plot (\x,{0.750588*(\x-2.59559)+0.580285})node at(1,-0.8){\footnotesize $g_1(x)$};
				\filldraw[red!70!black] (2.59559,0)circle(1.5pt)nodeat(2.7,-0.3){\footnotesize $P_1$};
				\draw[->, line width=0.5, red!70!black, dashed] (2.59559,0)--(2.59559,0.580285);
				\filldraw[red!70!black] (2.59559,0.580285)circle(1.5pt);
				\draw[smooth,samples=100,domain=0.6:2.5, gray, line width=0.5 ] plot (\x,{0.503194*(\x-1.82248)+0.095631})node at(0.3,-0.6){\footnotesize $g_2(x)$};
				\filldraw[red!70!black] (1.82248,0)circle(1.5pt)node at(1.9,-0.3){\footnotesize $P_2$};
				\filldraw[red!70!black] (1.82248,0.095631)circle(1.5pt);
				\filldraw[green!70!black!] (1.63243,0)circle(1.5pt) node at(1.1,0.3) {\footnotesize L"osung};
 			\end{tikzpicture}
 			\caption{Newton-Verfahren}
			\label{fig. 1dim-Bsp-Newton}
 		\end{figure}
 
		\subsubsection{Schleifen beim Newton-Verfahren}
			Das Newton-Verfahren f"uhrt leider nicht in jedem Fall zu einer L"osung. Je nach dem wie die Funktion aussieht, ist es beispielsweise m"oglich, dass der Newton-Algorithmus in einer Schliefe h"angen bleibt, wie das in der Grafik gezeigt ist. Dieses Problem ist bei der L"osungsfindung zu ber"ucksichtigen.\\
 			\begin{figure}[ht!]\centering
				\begin{tikzpicture}[>=latex', scale=0.8]
					\draw[->] (-2.3,0) -- (2.5,0)node[right] {$x$};
					\draw[->] (0,-1.5) -- (0,1.5) node[above] {$y$};
					\draw[smooth,samples=100,domain=-2.3:2.3, gray, line width=0.75 ] plot (\x,{0.5*\x+1});			
					\draw[smooth,samples=100,domain=-2.3:2.4, gray, line width=0.75 ] plot (\x,{0.5*\x-1});			
					\draw[smooth,samples=100,domain=-2.5:2.5, black, line width=1 ] plot (\x,{2.15*sin((\x)*105/pi)});
					\draw[->,red!70!black, line width=0.75,dashed] (2,0) -- (2,2);
					\draw[->,red!70!black, line width=0.75,dashed] (-2,0) -- (-2,-2);
					\filldraw[red!70!black] (-2,0)circle(2pt);
					\filldraw[red!70!black] (2,0)circle(2pt);
					\filldraw[red!70!black] (-2,-2)circle(2pt);
					\filldraw[red!70!black] (2,2)circle(2pt);
% 					\draw[white](-1,-4)--(1,-4);
				\end{tikzpicture}
 				\caption{Schleifen beim Newton-Verfahren}
				\label{fig. Schleife}
 			\end{figure}

		\subsubsection{Anwendung des Newton-Verfahren auf das GPS-Optimierungsproblem}
			Wird nun das Newton-Verfahren auf das GPS-Optimierungsproblem angewendet, funktioniert das folgendermassen:
			\begin{enumerate}
				\item Geeigneter Startpunkt $p_0 = (x_0,\,y_0,\,z_0,\,t_0)$ w"ahlen.
				\item Gleichungssystem in diesem Punkt linearisieren.\\[0.2cm]
					$\nabla f (x_j,y_j,z_j,t_j) = \nabla f (p_j) =  
					\begin{bmatrix} f_x(p_j) \\ f_y(p_j) \\ f_z(p_j) \\ f_t(p_j) \end{bmatrix} \approx \begin{bmatrix} g_x(p_j) \\ g_y(p_j) \\ g_z(p_j) \\ g_t(p_j) \end{bmatrix}$\\[0.2cm]
					$= \begin{bmatrix} f_{xx}(p_j)\, (x-x_j) + f_{xy}(p_j)\, (y-y_j) + f_{xz}(p_j)\, (z-z_j) + f_{xt}(p_j)\, (t-t_j) + f_x(p_j) \\
					f_{xy}(p_j)\, (x-x_j) + f_{yy}(p_j)\, (y-y_j) + f_{yz}(p_j)\, (z-z_j) + f_{yt}(p_j)\, (t-t_j) + f_y(p_j) \\
					f_{xz}(p_j)\, (x-x_j) + f_{yz}(p_j)\, (y-y_j) + f_{zz}(p_j)\, (z-z_j) + f_{zt}(p_j)\, (t-t_j) + f_z(p_j) \\
					f_{xt}(p_j)\, (x-x_j) + f_{yt}(p_j)\, (y-y_j) + f_{zt}(p_j)\, (z-z_j) + f_{tt}(p_j)\, (t-t_j) + f_t(p_j) 
					\end{bmatrix} $\\[0.2cm]
					$= \begin{bmatrix} 
					f_{xx}(p_j) &  f_{xy}(p_j) & f_{xz}(p_j) & f_{xt}(p_j)\\
					f_{xy}(p_j) & f_{yy}(p_j) & f_{yz}(p_j) & f_{yt}(p_j)\\
					f_{xz}(p_j) & f_{yz}(p_j) & f_{zz}(p_j) & f_{zt}(p_j)\\
					f_{xt}(p_j) & f_{yt}(p_j) & f_{zt}(p_j) & f_{tt}(p_j)
					\end{bmatrix} \cdot \begin{bmatrix} 
					\Delta x \\
					\Delta y \\
					\Delta z \\
					\Delta t 
					\end{bmatrix} + \begin{bmatrix} 
					f_x(p_j) \\
					f_y(p_j) \\
					f_z(p_j) \\
					f_t(p_j) 
					\end{bmatrix}$
				\item Linearisiertes Gleichungssystem l"osen.\\[0.2cm]
					$\begin{bmatrix} 
					f_{xx}(p_j) &  f_{xy}(p_j) & f_{xz}(p_j) & f_{xt}(p_j)\\
					f_{xy}(p_j) & f_{yy}(p_j) & f_{yz}(p_j) & f_{yt}(p_j)\\
					f_{xz}(p_j) & f_{yz}(p_j) & f_{zz}(p_j) & f_{zt}(p_j)\\
					f_{xt}(p_j) & f_{yt}(p_j) & f_{zt}(p_j) & f_{tt}(p_j)
					\end{bmatrix} \cdot \begin{bmatrix} 
					\Delta x \\
					\Delta y \\
					\Delta z \\
					\Delta t 
					\end{bmatrix} + \begin{bmatrix} 
					f_x(p_j) \\
					f_y(p_j) \\
					f_z(p_j) \\
					f_t(p_j) 
					\end{bmatrix} = \begin{bmatrix} 0 \\ 0 \\ 0 \\ 0 \end{bmatrix}$\\[0.2cm]
					$\begin{bmatrix} 
					f_{xx}(p_j) &  f_{xy}(p_j) & f_{xz}(p_j) & f_{xt}(p_j)\\
					f_{xy}(p_j) & f_{yy}(p_j) & f_{yz}(p_j) & f_{yt}(p_j)\\
					f_{xz}(p_j) & f_{yz}(p_j) & f_{zz}(p_j) & f_{zt}(p_j)\\
					f_{xt}(p_j) & f_{yt}(p_j) & f_{zt}(p_j) & f_{tt}(p_j)
					\end{bmatrix} \cdot \begin{bmatrix} 
					\Delta x \\
					\Delta y \\
					\Delta z \\
					\Delta t 
					\end{bmatrix} = \begin{bmatrix} 
					-f_x(p_j) \\
					-f_y(p_j) \\
					-f_z(p_j) \\
					-f_t(p_j) 
					\end{bmatrix}$\\
				\item Neuer Punkt berechnen.\\[0.2cm]
					$p_{j+1} = (x_j + \Delta x,\,y_j + \Delta y,\,z_j + \Delta z,\,t_j + \Delta t)$
				\item Wieder bei Punkt 2. beginnen.
				\item Solange wiederholen, bis $\Delta x$, $\Delta y$, $\Delta z$ und $\Delta t$ gen"ugend klein sind. 
			\end{enumerate}

	\subsection{Arten von station"aren Punkten}\label{stat_Punkte}
		Wie im Abschnitt \ref{Minimum der ZF} bereits erw"ahnt, muss zum finden der Extremalstellen der Gradient von $f$ gleich Null gesetzt werden. Dabei k"onnen als L"osungen dieses Gleichungssystems $|\nabla f = \vec 0 |$ drei Arten von station"aren Punkten vorkommen. Zum einen Maxima oder Minima, die f"ur Optimierungsprobleme im Allgemeinen von Bedeutung sind, und zum anderen Sattelpunkte. Sattelpunkte sind Stellen an denen es in gewisse Richtungen ansteigt und in andere abf"allt. Da es drei Arten von L"osungen gibt, jedoch f"ur das GPS-Optimierungsproblem nur die Minima von Bedeutung sind, m"ussen die anderen L"osungen ausgesondert werden. Um bestimmen zu k"onnen, um welche L"osungsart es sich bei einem bestimmten Punkt handelt, m"ussen die Eigenwerte $\lambda_i$ der Hesseschen Matrix berechnet werden. Die Hessesche Matrix sieht folgendermassen aus:\\[0.2cm]
		$H = \begin{bmatrix} 
		f_{xx}(p_L) & f_{xy}(p_L) & f_{xz}(p_L) & f_{xt}(p_L)\\
		f_{xy}(p_L) & f_{yy}(p_L) & f_{yz}(p_L) & f_{yt}(p_L)\\
		f_{xz}(p_L) & f_{yz}(p_L) & f_{zz}(p_L) & f_{zt}(p_L)\\
		f_{xt}(p_L) & f_{yt}(p_L) & f_{zt}(p_L) & f_{tt}(p_L)
		\end{bmatrix} \qquad\begin{array}{l} p_L\text{: L"osungspunkt}\end{array}$\\[0.2cm]
		Anhand der berechneten Eigenwerte kann nun die Art des station"aren Punktes bestimmt werden.\\[0.2cm]
		\begin{tabular}{lcllcl}
			& $\bullet$ & Minimum: & alle Eigenwerte $\lambda_i$ sind positiv: & $\Rightarrow$ & $\lambda_i > 0\quad,\forall i$\\
			& $\bullet$ & Maximum: & alle Eigenwerte $\lambda_i$ sind negativ: & $\Rightarrow$ & $\lambda_i < 0\quad,\forall i$\\
			& $\bullet$ & Sattelpunkt: & alle Eigenwerte $\lambda_i$ sind positiv oder negativ: & $\Rightarrow$ & $\lambda_i > 0 \;\lor\; \lambda_i < 0\quad,\forall i$\\
		\end{tabular}

	\subsection{Ber"ucksichtigung der Sattelpunkte beim Newton-Verfahren}
		Mit dem Wissen, dass das Gleichungssystem $|\nabla f = \vec 0 |$ (Abschnitt \ref{Minimum der ZF}) als L"osungen auch Sattelpunkte liefert, kann nun das Newton-Verfahren noch verbessert werden. Angenommen das Newton-Verfahren liefert als L"osung kein Minimum sondern ein Sattelpunkt, so gibt es mindestens eine Richtung, in die die Funktion abf"allt. Wird nun der Newton-Algorithmus gezielt in diese Richtung gelenkt, so ist es m"oglich, dass doch noch ein Minimum gefunden wird. Dabei ist die erfolgversprechendste Richtung nat"urlich die, in die es am steilsten abf"allt. Um diese Richtung zu finden, muss der Eigenvektor $\vec v_-$ zu einem negativen Eigenwert $\lambda_-$ gefunden werden. Der gefunde Eigenvektor $\vec v_-$ zeigt dann in die Richtung des steilsten Abstiegs. \\[0.25cm]
		\begin{minipage}{\textwidth}
%			\arrayrulecolor{red!70!black}
			\begin{tabular}{|l|}
				\hline
				\\[-0.2cm]
				$ H\cdot \vec v_- = \lambda_- \cdot \vec v_-$\\[0.2cm]
				$\left[H-\lambda_-\cdot E\right] \cdot \vec v_- = \vec 0$\\[0.2cm]
				\hline
			\end{tabular}
%			\arrayrulecolor{black}
		\end{minipage}\\[0.25cm]
		\begin{minipage}{\textwidth}
			$\begin{array}{ll} H: & \text{Hessesche Matrix des Sattelpunktes (vgl. Kapitel \ref{stat_Punkte})}\\ \lambda_-: & \text{Negativer Eigenwert der Hesseschen Matrix (vgl. Kapitel \ref{stat_Punkte})}\\ \vec v_-:& \text{Eigenvektor zum Eigenwert }\lambda_- \text{ der in die Richtung des steilsten Abstiegs zeigt.}\end{array}$
		\end{minipage}

		\begin{figure}[ht!]\centering
			\includegraphics[scale = 0.6]{gps/sattel.pdf}
			\caption{Sattelpunkte}
			\label{fig. Sattelpunkte}
		\end{figure}

	\subsection{Gewichtung der Satelliten}
		Um das Resultat der Positionsbestimmung noch zu verbessern, k"onnen die einzelnen Satelliten, bzw. die Fehler der Satellitendaten zus"atzlich mit einem Faktor gewichtet werden. Die Gewichtung wird zum einen aus der Empfangsqualit"at und zum anderen aus der Genauigkeit der Daten selber, die vom Satelliten gesendet wird, bestimmt. Die Zielfunktion mit Gewichtungsfaktoren sieht dann wie folgt aus:\\[0.2cm]
		\fbox{$f(x,y,z,t) = \text{\large$\sum\limits_{i=1}^{n}$}\;{\left[\dfrac{1}{\sigma_i}\cdot \left((x_i-x)^2 + (y_i-y)^2 + (z_i-z)^2\;\, - c^2 (t-t_i)^2\;\right)\right]^2}$}\\[0.2cm]
		Ist dabei die Standardabweichung $\sigma_i$ eines Satelliten gross, so wird sein Gewichtungsfaktor klein und dadurch wird er bei der Berechnung der Position weniger gewichtet.
\newpage
	\subsection{Zielfunktion und deren Ableitungen}
		\subsubsection{Ohne Gewichtung der Satelliten}
			\begin{tabular}{l}
				\textbf{Zielfunktion:}\\
				$f(x,y,z,t) = \text{\large$\sum\limits_{i=1}^{n}$}\;{\left[(x_i-x)^2 + (y_i-y)^2 + (z_i-z)^2\;\, - c^2 (t-t_i)^2\;\right]^2}$\\[0.4cm]
				\hline
				\\[-0.3cm]
				\textbf{Erste partielle Ableitungen in $x$, $y$, $z$ und $t$:}\\
				$f_x(x,y,z,t) = -4\;\text{\large$\sum\limits_{i=1}^{n}$}\;{\;(x_i-x)\left[(x_i - x)^2 + (y_i - y)^2 + (z_i - z)^2 - c^2(t - t_i)^2 \right]}$\\
				$f_y(x,y,z,t) = -4\;\text{\large$\sum\limits_{i=1}^{n}$}\;{\;(y_i-y)\left[(x_i - x)^2 + (y_i - y)^2 + (z_i - z)^2 - c^2(t - t_i)^2 \right]}$\\
				$f_z(x,y,z,t) = -4\;\text{\large$\sum\limits_{i=1}^{n}$}\;{\;(z_i-z)\left[(x_i - x)^2 + (y_i - y)^2 + (z_i - z)^2 - c^2(t - t_i)^2 \right]}$\\
				$f_t(x,y,z,t) = -4 \,c^2\;\text{\large$\sum\limits_{i=1}^{n}$}\;{\;(t-t_i)\left[(x_i - x)^2 + (y_i - y)^2 + (z_i - z)^2 - c^2(t - t_i)^2\right]}$\\[0.4cm]
				\hline
				\\[-0.3cm]
				\textbf{Zweite partielle Ableitungen in $x$, $y$, $z$ und $t$:}\\
				$f_{xx}(x,y,z,t) = 4\;\text{\large$\sum\limits_{i=1}^{n}$}\;{\;\left[3\cdot(x_i - x)^2 + (y_i - y)^2 + (z_i - z)^2 - c^2(t - t_i)^2\right]}$\\
				$f_{xy}(x,y,z,t) = 8\;\text{\large$\sum\limits_{i=1}^{n}$}\;{\;(x_i-x)(y_i-y)}$\\
				$f_{xz}(x,y,z,t) = 8\;\text{\large$\sum\limits_{i=1}^{n}$}\;{\;(x_i-x)(z_i-z)}$\\
				$f_{xt}(x,y,z,t) = 8\,c^2\;\text{\large$\sum\limits_{i=1}^{n}$}\;{\;(x_i-x)(t-t_i)}$\\[0.4cm]
%				\arrayrulecolor{gray}
				\hline
				\\[-0.3cm]
				$f_{yy}(x,y,z,t) = 4\;\text{\large$\sum\limits_{i=1}^{n}$}\;{\;\left[(x_i - x)^2 + 3\cdot(y_i - y)^2 + (z_i - z)^2 - c^2(t - t_i)^2\right]}$\\
				$f_{yz}(x,y,z,t) = 8\;\text{\large$\sum\limits_{i=1}^{n}$}\;{\;(y_i-y)(z_i-z)}$\\
				$f_{yt}(x,y,z,t) = 8\,c^2\;\text{\large$\sum\limits_{i=1}^{n}$}\;{\;(y_i-y)(t-t_i)}$\\[0.4cm]
				\hline
				\\[-0.3cm]
				$f_{zz}(x,y,z,t) = 4\;\text{\large$\sum\limits_{i=1}^{n}$}\;{\;\left[(x_i - x)^2 + (y_i - y)^2 + 3 \cdot (z_i - z)^2 - c^2(t - t_i)^2\right]}$\\
				$f_{zt}(x,y,z,t) = 8\,c^2\;\text{\large$\sum\limits_{i=1}^{n}$}\;{\;(z_i-z)(t-t_i)}$\\[0.5cm]
				\hline
				\\[-0.3cm]
				$f_{tt}(x,y,z,t) = -4\,c^2\;\text{\large$\sum\limits_{i=1}^{n}$}\;{\;\left[(x_i - x)^2 + (y_i - y)^2 + (z_i - z)^2 - 3\,c^2\cdot(t - t_i)^2\right]}$\\[0.4cm]
%				\arrayrulecolor{black}
				\hline
			\end{tabular}
\newpage
		\subsubsection{Mit Gewichtung der Satelliten}
			\begin{tabular}{l}
				\textbf{Zielfunktion:}\\
				$f(x,y,z,t) = \text{\large$\sum\limits_{i=1}^{n}$}\;{\left[\dfrac{1}{\sigma_i}\cdot \left((x_i-x)^2 + (y_i-y)^2 + (z_i-z)^2\;\, - c^2 (t-t_i)^2\;\right)\right]^2}$\\[0.4cm]
				\hline
				\\[-0.3cm]
				\textbf{Erste partielle Ableitungen in $x$, $y$, $z$ und $t$:}\\[0.1cm]
				$f_x(x,y,z,t) = -4\;\text{\large$\sum\limits_{i=1}^{n}$}\;{\;\dfrac{1}{{\sigma_i}^2}\,(x_i-x)\left[(x_i - x)^2 + (y_i - y)^2 + (z_i - z)^2 - c^2(t - t_i)^2 \right]}$\\[0.4cm]
				$f_y(x,y,z,t) = -4\;\text{\large$\sum\limits_{i=1}^{n}$}\;{\;\dfrac{1}{{\sigma_i}^2}\,(y_i-y)\left[(x_i - x)^2 + (y_i - y)^2 + (z_i - z)^2 - c^2(t - t_i)^2 \right]}$\\[0.4cm]
				$f_z(x,y,z,t) = -4\;\text{\large$\sum\limits_{i=1}^{n}$}\;{\;\dfrac{1}{{\sigma_i}^2}\,(z_i-z)\left[(x_i - x)^2 + (y_i - y)^2 + (z_i - z)^2 - c^2(t - t_i)^2 \right]}$\\[0.4cm]
				$f_t(x,y,z,t) = -4 \,c^2\;\text{\large$\sum\limits_{i=1}^{n}$}\;{\;\dfrac{1}{{\sigma_i}^2}\,(t-t_i)\left[(x_i - x)^2 + (y_i - y)^2 + (z_i - z)^2 - c^2(t - t_i)^2\right]}$\\[0.4cm]
				\hline
				\\[-0.3cm]
				\textbf{Zweite partielle Ableitungen in $x$, $y$, $z$ und $t$:}\\[0.1cm]
				$f_{xx}(x,y,z,t) = 4\;\text{\large$\sum\limits_{i=1}^{n}$}\;{\;\dfrac{1}{{\sigma_i}^2}\,\left[3\cdot(x_i - x)^2 + (y_i - y)^2 + (z_i - z)^2 - c^2(t - t_i)^2\right]}$\\[0.4cm]
				$f_{xy}(x,y,z,t) = 8\;\text{\large$\sum\limits_{i=1}^{n}$}\;{\;\dfrac{1}{{\sigma_i}^2}\,(x_i-x)(y_i-y)}$\\[0.4cm]
				$f_{xz}(x,y,z,t) = 8\;\text{\large$\sum\limits_{i=1}^{n}$}\;{\;\dfrac{1}{{\sigma_i}^2}\,(x_i-x)(z_i-z)}$\\[0.4cm]
				$f_{xt}(x,y,z,t) = 8\,c^2\;\text{\large$\sum\limits_{i=1}^{n}$}\;{\;\dfrac{1}{{\sigma_i}^2}\,(x_i-x)(t-t_i)}$\\[0.4cm]
%				\arrayrulecolor{gray}
				\hline
				\\[-0.3cm]
				$f_{yy}(x,y,z,t) = 4\;\text{\large$\sum\limits_{i=1}^{n}$}\;{\;\dfrac{1}{{\sigma_i}^2}\,\left[(x_i - x)^2 + 3\cdot(y_i - y)^2 + (z_i - z)^2 - c^2(t - t_i)^2\right]}$\\[0.4cm]
				$f_{yz}(x,y,z,t) = 8\;\text{\large$\sum\limits_{i=1}^{n}$}\;{\;\dfrac{1}{{\sigma_i}^2}\,(y_i-y)(z_i-z)}$\\[0.4cm]
				$f_{yt}(x,y,z,t) = 8\,c^2\;\text{\large$\sum\limits_{i=1}^{n}$}\;{\;\dfrac{1}{{\sigma_i}^2}\,(y_i-y)(t-t_i)}$\\[0.4cm]
				\hline
				\\[-0.3cm]
				$f_{zz}(x,y,z,t) = 4\;\text{\large$\sum\limits_{i=1}^{n}$}\;{\;\dfrac{1}{{\sigma_i}^2}\,\left[(x_i - x)^2 + (y_i - y)^2 + 3 \cdot (z_i - z)^2 - c^2(t - t_i)^2\right]}$\\[0.4cm]
				$f_{zt}(x,y,z,t) = 8\,c^2\;\text{\large$\sum\limits_{i=1}^{n}$}\;{\;\dfrac{1}{{\sigma_i}^2}\,(z_i-z)(t-t_i)}$\\[0.5cm]
				\hline
				\\[-0.3cm]
				$f_{tt}(x,y,z,t) = -4\,c^2\;\text{\large$\sum\limits_{i=1}^{n}$}\;{\;\dfrac{1}{{\sigma_i}^2}\,\left[(x_i - x)^2 + (y_i - y)^2 + (z_i - z)^2 - 3\,c^2\cdot(t - t_i)^2\right]}$\\[0.4cm]
%				\arrayrulecolor{black}
				\hline
			\end{tabular}

