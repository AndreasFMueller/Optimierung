\chapter{Lineare Optimierung\label{chapter-lineare-optimierung}}

\section{Problemstellung}
In einem linearen Optimierungsproblem ist die Kostenfunktion linear
und die Ungleichungen, die das zul"assige Gebiet definieren, sind
ebenfalls linear. 
Die Kostenfunktion ist also von der Form
\[
Z=c^tx,\quad c,x\in\mathbb R^n.
\]
Da eine lineare Funktion unbeschr"ankt ist, kann ein lineares
Optimierungsproblem nur dann eine L"osung haben, wenn das zul"assige
Gebiet beschr"ankt ist, mindestens in Richtung gr"osster Zunahme der
Kostenfunktion. Es ist daher kaum eine Einschr"ankung, anzunehmen, dass 
die Variablen $x\in\mathbb R^n$ alle positiv sind. Notfalls kann man 
das durch addieren einer Konstante oder Vorzeichenwechsel erzwingen.

Die Ungleichungen sind von der Form
\[
Ax\le b,
\]
wobei $A$ eine $m\times n$-Matrix und $b\in\mathbb R^m$ ist.
Dazu kommen noch die Bedingungen $x\ge 0$.

Eine lineare Gleichung beschreibt eine Ebene, eine
lineare Ungleichung demzufolge einen Halbraum. Das zul"assige Gebiet
eines linearen Optimierungsproblem ist also immer die Schnittmenge
mehrere Halbr"aume.

\section{Graphische L"osung}
F"ur nur zwei Variablen l"asst sich ein lineares Optimierungsproblem
sehr gut visualisieren. Die Ungleichungen
\[
Ax\le b\qquad\text{und}\qquad x\ge 0
\]
definieren ein polygonales Gebiet in der Ebene, das zul"assige Gebiet.
Wir illustrieren das an dem Beispielproblem
\begin{equation}
\begin{linsys}{2}
-x&+&3y&\le&12\\
x&+&3x&\le&18\\
4x&+&y&\le&28
\end{linsys}
\label{lp-beispiel-ungleichungen}
\end{equation}
Sie definieren das Gebiet $G$ in Abbildung \ref{lp-beispiel}.
\begin{figure}
\begin{center}
\includegraphics[width=\hsize]{images/lp-1}
\end{center}
\caption{Zul"assiges Gebiet f"ur die Ungleichungen
(\ref{lp-beispiel-ungleichungen})\label{lp-beispiel}}
\end{figure}

Die Zielfunktion ist ebenfalls linear, f"ur das Beispielgebiet
(\ref{lp-beispiel-ungleichungen}) verwenden wir die Zielfunktion
\begin{equation}
Z=2x+3y
\label{lp-beispiel-zielfunktion}
\end{equation}
\begin{figure}
\begin{center}
\includegraphics[width=\hsize]{images/lp-2}
\end{center}
\caption{Zul"assiges Gebiet mit Graph der Zielfunktion f"ur verschiedene
Zielwerte. Zielwert $Z=24$ ist optimal, er wird angenommen im
Punkt $(6,4)$.
\label{lp-beispiel-zielfunktion}}
\end{figure}%
In Abbildung \ref{lp-beispiel-zielfunktion} sind die Geraden $Z=\operatorname{const}$
f"ur einige Werte der Zielfunktion eingezeichnet.
Man kann ablesen, dass das Optimum im Schnittpunkt der Geraden
beschrieben der beiden Gleichungen
\begin{equation}
\begin{linsys}{2}
x&+&3x&\le&18\\
4x&+&y&\le&28
\end{linsys}
\notag
\end{equation}
angenommen wird.
Die L"osung des Gleichungssystems liefert die optimalen Koordinaten
$(6,3)$.

\section{Von Ungleichungen zu Gleichungen}
Die graphische L"osung hat gezeigt, dass der L"osungsvektor eines
linearen Optimierungsproblems immer einige der Ungleichungen exakt
erf"ullt. Die L"osung kann also in zwei Schritten gefunden werden:
\begin{compactenum}
\item Herausfinden, welche Ungleichungen exakt erf"ullt sein m"ussen.
\item Mit diesen Gleichungen ein lineares Gleichungssystem l"osen.
\end{compactenum}
Die L"osung des Gleichungssystems l"asst sich mit dem Gauss-Algorithmus
sehr effizient finden.

Die Ungleichungen (\ref{lp-beispiel-ungleichungen}) k"onnen durch
Einf"uhrung zus"atzlicher Variablen $s_1,\dots,s_3$ in die Gleichungen
\begin{equation}
\begin{linsys}{5}
-x&+&3y&+&s_1& &   & &   &=&12\\
 x&+&3y& &   &+&s_2& &   &=&18\\
4x&+& y& &   & &   &+&s_3&=&28
\end{linsys}
\label{lp-ungleichungen-schlupf}
\end{equation}
umgewandelt werden.
Die Variable $s_i$ gibt wieder, wie weit entfernt von der Gleichheit
die Ungleichung Nummer $i$ ist.
Ist die Variable $s_i=0$, dann ist die Gleichung scharf erf"ullt.
Schritt~1 l"auft also darauf hinaus, dass festgestellt werden muss,
welche der Variablen $s_i$ verschwinden.

Die L"osung des Optimierungsproblems ist auf dem Rand des
zul"assigen Gebietes zu finden.
Die Ecken des Gebietes sind Schnittpunkt zweier Gleichungen.
Sie werden also dadurch bestimmt, dass zwei Variablen 
Null gesetzt werden, die verbleibenden drei Variablen
lassen sich dann mit Hilfe des Gauss-Algorithmus bestimmen.
Die oben graphisch gefundene L"osung entspricht der
Wahl $s_2=0$ und $s_3=0$.

Die Zielfunktion $Z$ ist eine lineare Funktion von $x$ und $y$.
Da drei Variablen eindeutig bestimmt sind, sobald zwei der
Variablen festgelegt sind, muss sich auch $Z$ durch diese beiden
Variablen ausdr"ucken l"asst. Dazu kann man auch den Gauss-Algorithmus
verwenden, wenn man $Z$ als zus"atzliche Variable betrachtet.
Das zugeh"orige Gaustableau ist dann
\[
\begin{tabular}{|>{$}c<{$}|>{$}c<{$}>{$}c<{$}|>{$}c<{$}>{$}c<{$}>{$}c<{$}|>{$}c<{$}|}
\hline
1&-2&-3& 0& 0& 0& 0\\
\hline
0&-1& 3& 1& 0& 0&12\\
0& 1& 3& 0& 1& 0&18\\
0& 4& 1& 0& 0& 1&28\\
\hline
\end{tabular}
\]
Die erste Zeile ist die Zielfunktion, die unteren drei Zeilen codieren die
Ungleichungen.

F"uhrt man jetzt den Gaussalgorithmus durch, erh"alt man
\begin{align*}
&\rightarrow
\begin{tabular}{|>{$}c<{$}|>{$}c<{$}>{$}c<{$}|>{$}c<{$}>{$}c<{$}>{$}c<{$}|>{$}c<{$}|}
\hline
 1&-2&-3& 0& 0& 0&  0\\
\hline
 0& 1&-3&-1& 0& 0&-12\\
 0& 0& 6& 1& 1& 0& 30\\
 0& 0&13& 4& 0& 1& 76\\
\hline
\end{tabular}
&&\rightarrow
\begin{tabular}{|>{$}c<{$}|>{$}c<{$}>{$}c<{$}|>{$}c<{$}>{$}c<{$}>{$}c<{$}|>{$}c<{$}|}
\hline
 1&-2&-3&      0    &      0    & 0&  0\\
\hline
 0& 1&-3&     -1    &      0    & 0&-12\\
 0& 0& 1&\frac16    &\frac16    & 0&  5\\
 0& 0& 0&\frac{11}6 &-\frac{13}6& 1& 11\\
\hline
\end{tabular}
\\
&\rightarrow
\begin{tabular}{|>{$}c<{$}|>{$}c<{$}>{$}c<{$}|>{$}c<{$}>{$}c<{$}>{$}c<{$}|>{$}c<{$}|}
\hline
 1&-2&-3&          0&             0&            0&  0\\
\hline
 0& 1&-3&          0&-\frac{13}{11}& \frac{6}{11}& -6\\
 0& 0& 1&          0&  \frac{4}{11}&-\frac{1}{11}&  4\\
 0& 0& 0&          1&-\frac{13}{11}& \frac{6}{11}&  6\\
\hline
\end{tabular}
&&\rightarrow
\begin{tabular}{|>{$}c<{$}|>{$}c<{$}>{$}c<{$}|>{$}c<{$}>{$}c<{$}>{$}c<{$}|>{$}c<{$}|}
\hline
 1&-2& 0&          0& \frac{12}{11}&-\frac{3}{11}& 12\\
\hline
 0& 1& 0&          0& -\frac{1}{11}& \frac{3}{11}&  6\\
 0& 0& 1&          0&  \frac{4}{11}&-\frac{1}{11}&  4\\
 0& 0& 0&          1&-\frac{13}{11}& \frac{6}{11}&  6\\
\hline
\end{tabular}
\\
&\rightarrow
\begin{tabular}{|>{$}c<{$}|>{$}c<{$}>{$}c<{$}|>{$}c<{$}>{$}c<{$}>{$}c<{$}|>{$}c<{$}|}
\hline
 1& 0& 0&          0& \frac{10}{11}& \frac{3}{11}& 24\\
\hline
 0& 1& 0&          0& -\frac{1}{11}& \frac{3}{11}&  6\\
 0& 0& 1&          0&  \frac{4}{11}&-\frac{1}{11}&  4\\
 0& 0& 0&          1&-\frac{13}{11}& \frac{6}{11}&  6\\
\hline
\end{tabular}
\end{align*}
Aus dem letzten Tableau kann man die vollst"andige L"osung
des Optimierungsproblems ablesen. Das Optimum wird angenommen
f"ur $x=6$ und $y=4$, die erste der drei Ungleichungen ist
nicht scharf erf"ullt, und der optimale Wert der Zielfunktion
\[
Z=24-\frac{10}{11}s_2-\frac{3}{11}s_3
\]
ist $24$.

Ebenso kann man ablesen, dass sich der Wert der Zielfunktion
nicht mehr verbessern l"asst.
Dazu m"usste man n"amlich notwendigerweise $s_2$ oder $s_3$
vergr"ossern, und beides w"urde den Wert der Zielfunktion verringern.

Im allgemeinen Fall eines linearen Optimierungsproblems mit $n$
Unbekannten und $m$ Ungleichungen ist also ein Gleichungssystem
zu l"osen mit $n+m+1$ Variablen ($n$ Variablen, $m$ Schlupfvariablen und
$Z$) und $m+1$ Gleichungen zu l"osen.
Dabei sind $(n+m+1)-(m+1)=n$ Variablen frei w"ahlbar.
Sobald man weiss, welche $n$ Variablen frei gew"ahlt werden k"onnen,
kann das Problem mit dem Gauss-Algorithmus gel"ost werden.
Ob man tats"achlich eine L"osung gefunden hat, kann man daran
erkennen, ob alle Koeffizienten in der $Z$-Zeile des Gauss-Tableau
positiv sind.

\section{Simplex-Algorithmus}
\subsection{L"osungsprinzip}
Der Simplex-Algorithmus l"ost das Problem, die $n$ Variablen
zu bestimmen, die auf $0$ gesetzt werden k"onnen, um die L"osung
des Optimierungsproblems zu finden.
Kennt man bereits einen noch nicht optimalen Eckpunkt des zul"assigen
Gebietes, dann gibt es mindestens einen benachbarten Eckpunkt,
der die Funktion $Z$ vergr"ossert.
Man kann also von Ecke zu Ecke den Kanten des zul"assigen
Gebietes folgen, bis sich die Zielfunktion nicht mehr vergr"ossern
l"asst. Dies ist das Prinzip des Simplex-Algorithmus.

Die benachbarte Ecke erf"ullt bis auf eine Gleichung alle
gleichungen der aktuellen Ecke.
Ausserdem erf"ullt sie eine Gleichung, die aktuelle Ecke nicht
erf"ullt.
Aus einer Menge der Variablen, die f"ur den urspr"unglichen
Eckpunkt auf $0$ gesetzt werden m"ussen, muss eine Variable
entfernt werden, und daf"ur eine neue Variable hinzugef"ugt
werden, die bisher $>0$ war.
Der Simplex-Algorithmus muss also in jedem Schritt entscheiden,
welche der frei w"ahlbaren Variablen durch welche andere 
Variable ersetzt werden soll.

\subsection{Beispiel}
Wir f"uhren dies am Beispiel (\ref{lp-beispiel}) und 
(\ref{lp-beispiel-zielfunktion}) durch.
Die Ecke $(0,0)$ ist im zul"assigen Gebiet, sie entspricht
der Menge $\{x,y\}$ der frei w"ahlbaren Variablen. Das zugh"orige
Gauss-Tableau ist
\[
\begin{tabular}{|>{$}c<{$}|>{$}c<{$}>{$}c<{$}|>{$}c<{$}>{$}c<{$}>{$}c<{$}|>{$}c<{$}|}
\hline
1&-2&-3& 0& 0& 0& 0\\
\hline
0&-1& 3& 1& 0& 0&12\\
0& 1& 3& 0& 1& 0&18\\
0& 4& 1& 0& 0& 1&28\\
\hline
 & *& *&  &  &  &  \\
\hline
\end{tabular}
\]

Zur Verbesserung des Resultats muss jetzt eine der Variablen $x,y$
durch eine der Variablen $s_1,s_2, s_3$ ersetzt werden. Beide
Koeffizienten von $x$ und $y$ in der Zielfunktion sind negativ, 
beide Variablen kommen also daf"ur in Frage, ersetzt zu werden.
Der Koeffizient von $y$ ist jedoch kleiner, so dass die Wahl von $y$
eine gr"ossere Verbesserung der Zielfunktion verspricht, als die
Wahl von $x$ als auszutauschen Variable.

Ersetzt man $y$ durch die Variable $s_i$, dann zeigt Ungleichung
$i$ an, wie gross $s_i$ gemacht werden kann:
\begin{align*}
s_1&=0&\Rightarrow&\quad y\le \frac{12}3=4
\\
s_2&=0&\Rightarrow&\quad y\le \frac{18}3=6
\\
s_3&=0&\Rightarrow&\quad y\le 28
\\
\end{align*}
Da alle Ungleichungen erf"ullt bleiben m"ussen, darf $x$ den Wert $7$
nicht "ubersteigen, und $y$ darf den Wert $4$ nicht "uberschreiten.
Die Wahl f"allt also auf $s_1$.
Es muss also $y$ gegen $s_1$ ausgetauscht werden.

Neu soll jetzt also $s_1$ eine frei w"ahlbare Variable werden, man
erreicht dies mit Hilfe eines Gauss-Schrittes, der das Element $-1$
in der zweiten Zeile und der zweiten Spalte zu $1$ macht und die "ubrigen
Elemente dieser Spalte zu $0$:
\[
\begin{tabular}{|>{$}c<{$}|>{$}c<{$}>{$}c<{$}|>{$}c<{$}>{$}c<{$}>{$}c<{$}|>{$}c<{$}|}
\hline
1&        -3& 0&       1& 0& 0& 12\\
\hline
0&  -\frac13& 1& \frac13& 0& 0&  4\\
0&         2& 0&      -1& 1& 0&  6\\
0&\frac{13}3& 0&-\frac13& 0& 1& 24\\
\hline
 & *&  & *&  &  &  \\
\hline
\end{tabular}
\]
Weil nicht alle Koeffizienten in der ersten Zeile positiv sind,
ist dies noch nicht die optimale L"osung.

Im n"achsten Schritt muss die Variable $x$ ersetzt werden, zur 
Auswahl stehen $y$, $s_2$ und $s_3$.
\begin{align*}
  y&=0&\Rightarrow&\quad -x\le 12
\\
s_2&=0&\Rightarrow&\quad x\le \frac{6}{2} = 3
\\
s_3&=0&\Rightarrow&\quad x\le \frac{72}{13}
\end{align*}
Damit die Ecke weiterhin im zul"assigen Bereich bleibt, ist $s_2=0$
die verbleibende Ecke, also muss $y$ gegen $s_2$ ausgetauscht werden.
\[
\begin{tabular}{|>{$}c<{$}|>{$}c<{$}>{$}c<{$}|>{$}c<{$}>{$}c<{$}>{$}c<{$}|>{$}c<{$}|}
\hline
1&         0& 0&   -\frac12&    \frac32&      0& 21\\
\hline
0&         0& 1&    \frac16&    \frac16&      0&  5\\
0&         1& 0&   -\frac12&    \frac12&      0&  3\\
0&         0& 0& \frac{11}6&-\frac{13}6&      1& 11\\
\hline
 &  &  & *& *&  &  \\
\hline
\end{tabular}
\]
Jetzt zeigt sich, dass $s_1$ ausgetauscht werden muss. 
\begin{align*}
  y&=0&\Rightarrow&\quad s_1\le 30
\\
  x&=0&\Rightarrow&\quad -s_1\le 6
\\
s_3&=0&\Rightarrow&\quad s_1\le 6
\end{align*}
Limitierend ist also die dritte Ungleichung. Also muss $s_1$ gegen
$s_3$ ausgetauscht werden.
\[
\begin{tabular}{|>{$}c<{$}|>{$}c<{$}>{$}c<{$}|>{$}c<{$}>{$}c<{$}>{$}c<{$}|>{$}c<{$}|}
\hline
1&         0& 0&          0& \frac{10}{11}& \frac{3}{11}& 24\\
\hline
0&         0& 1&          0&  \frac{4}{11}&-\frac{1}{11}&  4\\
0&         1& 0&          0& -\frac{1}{11}& \frac{3}{11}&  6\\
0&         0& 0&          1&-\frac{13}{11}& \frac{6}{11}&  6\\
\hline
 &          &  &           &             *&            *&  \\
\hline
\end{tabular}
\Leftrightarrow
\begin{tabular}{|>{$}c<{$}|>{$}c<{$}>{$}c<{$}|>{$}c<{$}>{$}c<{$}>{$}c<{$}|>{$}c<{$}|}
\hline
1&         0& 0&          0& \frac{10}{11}& \frac{3}{11}& 24\\
\hline
0&         1& 0&          0& -\frac{1}{11}& \frac{3}{11}&  6\\
0&         0& 1&          0&  \frac{4}{11}&-\frac{1}{11}&  4\\
0&         0& 0&          1&-\frac{13}{11}& \frac{6}{11}&  6\\
\hline
 &          &  &           &             *&            *&  \\
\hline
\end{tabular}
\]
Dieses Verfahren hat also die gleiche L"osung gefunden wie das
graphische Verfahren auch.

\subsection{Austauschschritte}
Das lineare Optimierungsproblem entspricht dem Tableau
\begin{equation}
\begin{tabular}{|
>{$}c<{$}|
>{$}c<{$}
>{$}c<{$}
>{$}c<{$}|
>{$}c<{$}
>{$}c<{$}
>{$}c<{$}|
>{$}c<{$}|}
\hline
Z&x_1&\dots&x_n&s_1&\dots&s_m&\\
\hline
1&-c_1&\dots&-c_2&0&\dots&0&0\\
\hline
0&a_{11}&\dots&a_{1n}&1&\dots&0&b_1\\
\vdots&\vdots&\ddots&\vdots&\vdots&\ddots&\vdots&\vdots\\
0&a_{m1}&\dots&a_{mn}&0&\dots&1&b_m\\
\hline
% &*&\dots&*& && & \\
%\hline
\end{tabular}
\end{equation}
Wir nehmen zun"achst an, dass der Punkt $x_i=0$, $i=1,\dots,n$ 
im zul"assigen Gebiet liegt.
Die Variablen $x_1,\dots,x_n$ sind also die frei w"ahlbar, die
Schlupfvariablen $s_1,\dots,s_m$ sind durch die Wahl der
$x_i$ bestimmt.
Wir symbolisieren dies durch Markierung der Variablen $x_i$ mit
Sternen in der untersten Zeile:
\begin{equation}
\begin{tabular}{|
>{$}c<{$}|
>{$}c<{$}
>{$}c<{$}
>{$}c<{$}|
>{$}c<{$}
>{$}c<{$}
>{$}c<{$}|
>{$}c<{$}|}
\hline
Z&x_1&\dots&x_n&s_1&\dots&s_m&\\
\hline
1&-c_1&\dots&-c_2&0&\dots&0&0\\
\hline
0&a_{11}&\dots&a_{1n}&1&\dots&0&b_1\\
\vdots&\vdots&\ddots&\vdots&\vdots&\ddots&\vdots&\vdots\\
0&a_{m1}&\dots&a_{mn}&0&\dots&1&b_m\\
\hline
 &*&\dots&*& && & \\
\hline
\end{tabular}
\end{equation}

Der Simplex-Algorithmus muss in jedem Schritt entscheiden,
welche der ausgew"ahlten Variablen durch welche andere Variable
ersetzt werden soll.
Danach muss ein Gauss-Schritt mit dieser Variablen durchgef"uhrt werden.

Jede Variable, f"ur die in der Zeile der Zielfunktion ein negativer
Koeffizient vorliegt, ist ein Austauschkandidat. 

\subsection{Startpunkt}

\section{Das duale Problem}

