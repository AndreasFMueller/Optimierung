\chapter{Lineare Optimierung\label{chapter-lineare-optimierung}}

\section{Problemstellung}
In einem linearen Optimierungsproblem ist die Kostenfunktion linear
und die Ungleichungen, die das zul"assige Gebiet definieren, sind
ebenfalls linear. 
Die Kostenfunktion ist also von der Form
\[
Z=c^tx,\quad c,x\in\mathbb R^n.
\]
Da eine lineare Funktion unbeschr"ankt ist, kann ein lineares
Optimierungsproblem nur dann eine L"osung haben, wenn das zul"assige
Gebiet beschr"ankt ist, mindestens in Richtung gr"osster Zunahme der
Kostenfunktion. Es ist daher kaum eine Einschr"ankung, anzunehmen, dass 
die Variablen $x\in\mathbb R^n$ alle positiv sind. Notfalls kann man 
das durch addieren einer Konstante oder Vorzeichenwechsel erzwingen.

Die Ungleichungen sind von der Form
\[
Ax\le b,
\]
wobei $A$ eine $m\times n$-Matrix und $b\in\mathbb R^m$ ist.
Dazu kommen noch die Bedingungen $x\ge 0$.

Eine lineare Gleichung beschreibt eine Ebene, eine
lineare Ungleichung demzufolge einen Halbraum. Das zul"assige Gebiet
eines linearen Optimierungsproblem ist also immer die Schnittmenge
mehrere Halbr"aume.

\section{Graphische L"osung}

\section{Von Ungleichungen zu Gleichungen}
Die graphische L"osung hat gezeigt, dass der L"osungsvektor eines
linearen Optimierungsproblems immer einige der Ungleichungen exakt
erf"ullt. Die L"osung kann also in zwei Schritten gefunden werden:
\begin{compactenum}
\item Herausfinden, welche Ungleichungen exakt erf"ullt sein m"ussen.
\item Mit diesen Gleichungen ein lineares Gleichungssystem l"osen.
\end{compactenum}
Die L"osung des Gleichungssystems l"asst sich mit dem Guass-Algorithmus
sehr effizient finden.

\section{Simplex-Algorithmus}

\section{Das duale Problem}
