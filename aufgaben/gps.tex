\begin{aufgabe}
Positionsbestimmung mit GPS als Minimalproblem.
\end{aufgabe}

{\parindent 0pt Das Global Positioning System GPS}
erm"oglicht, aus den von Satelliten "ubermittelten Positionen
$(x_i, y_i, z_i, t_i)$ durch Vergleich der Zeiten die eigene
Position $(x,y,z,t)$
zu bestimmen. Sobald man mehr als 4 Satelliten empfangen kann,
k"onnen die Daten nicht mehr exakt einen Punkt bestimmen, das
Problem ist "uberbestimmt. Da man nicht weiss, wie ``falsch''
das Signal eines Satelliten ist, sucht man eine L"osung, die am
``wenigsten falsch'', man l"ost also ein Optimierungsproblem.
Konzipieren Sie einen Algorithmus, der die optimale Position findet.

Man kann nat"urlich auch aus der Lage der Satelliten absch"atzen, welche
nicht gut geeignet sind f"ur eine Positionsbestimmung. Das Signal
eines Satelliten unmittelbar "uber dem Horizont wird zum Beispiel st"arker
von der H"ohe der Tropopause beeinflusst. Auch die emfangene Feldst"arke
liefert Hinweise darauf, wie gut ein Satellit geeignet ist. Erweitern
Sie ihren Algorithmus, dass er auch mit einer solchen Gewichtung der
Satelliten funktioniert.

