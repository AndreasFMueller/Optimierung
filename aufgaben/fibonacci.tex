\begin{aufgabe}
Wie findet die Fibonacci-Methode ein Minimum einer unstetigen Funktion?
\end{aufgabe}

{\parindent 0pt Extremwerte einer differenzierbaren Funktion finden geh"ort
zu den n"utzlichen Leistungen der Analysis.} Doch wie findet man das
Minimum einer Funtion, die nur stetig oder sogar nicht stetig ist?
Die Fibonacci-Methode oder die damit verwandte Methode des
goldenen Schnittes findet ein Optimum in solchen F"allen.

Zeigen Sie an Beispielen, wie diese Methode funktioniert. 
Wie schnell und wie genau ist die Methode?
Unter welchen Voraussetzung ist sie einsetzbar, wann
funktioniert sie nicht?
