\begin{aufgabe}
Wie findet der Simplex-Algorithmus die optimale L"osung?
\end{aufgabe}

{\parindent 0pt Der Simplex-Algorithmus}
findet zuverl"assig eine optimale L"osung f"ur lineare Optimierungsprobleme.
Trotzdem weiss man, dass seine Worst-case-Laufzeit exponentiell mit der
Problemgr"osse anw"achst. Dieser scheinbare Widerspruch l"ost sich auf,
wenn man besser verstehen kann, wie der Simplex-Algorithmus sich der
L"osung n"ahert.  Sein Vorgehen ist "ahnlich einem Abstiegs-Verfahren.

Um dies zu verstehen, wird folgendes Experiment vorgeschlagen. Man
versucht, eine lineares Optimierungsproblem mit nur sehr wenigen Variablen
$x_1,\dots,x_n$, z.~B.~$n=3$, aber sehr vielen Ungleichungen zu l"osen.
Dann visualisiert man den Pfad der vom Simplex-Algorithmus in jedem
Schritt gefundenen Ecke, und verfolgt, wie sich die Ecke dem Optimum
n"ahert.

Die grosse Zahl von Ungleichungen erzeugt man zuf"allig: man w"ahlt
zun"achst $N$ (z.~B.~N=10000) zuf"allige Einheitsvektoren $\vec n_i$ im ersten
Oktanten aus ($x\ge 0$, $y\ge 0$, $z\ge 0$). Dann nimmt man als
Ungleichungen
\[
\vec n_i\cdot \vec x\le 1
\]
Die Ebenen $\vec n_i\cdot \vec x$ sind alle Tangentialgebenen
der Einheitskugel. Dann sucht man mit dem Simplex-Algorithmus des
Maximum der Zielfunktion 
\[
Z=\sum_{i=1}^nx_i,
\]
also $Z=x+y+z$ im dreidimensionalen Fall. Man erwartet als Maximum
einen Punkt in der N"ahe von $\frac1{\sqrt{3}}(1,1,1)$.

\begin{compactenum}
\item Wieviele Schritte braucht der Simplex-Algorithmus im Schnitt,
um das Optimum zu finden?
\item Wie lange ist der Weg von der Ausgangsecke bis zum Optimum?
\end{compactenum}

