\begin{aufgabe}
Wie optimiert eine Ameisen-Kolonie?
\end{aufgabe}

{\parindent 0pt Ameisen-Kolonien} finden gemeinsam einen optimalen Weg von einem
Futter-Fundort zum Bau des Ameisenstaates. Sie markieren den Weg
mit Pheromonen, die sich aber auch wieder verfl"uchtigen k"onnen.
Findet eine Ameise eine Abk"urzung werden viele Ameisen ihre Pheromonspur 
folgen.
Da die Abk"urzung schneller und damit auch h"aufiger begangen werden 
kann, wird sich diese Pheromon-Spur weniger schnell verfl"uchtigen,
so dass immer mehr Ameisen den k"urzeren Weg nehmen werden.

Zeigen Sie an Beispielen, wie sich diese Idee aus der Natur auf
mathematische Optimierungsprobleme "ubertragen l"asst. Welche
Art von Optimierungsproblemen sind der L"osung durch diese Methode
zug"anglich?
