\begin{aufgabe}
Das Licht nimmt immer den k"urzesten Weg.
Die Lichtgeschwindigkeit
h"angt von der optischen Dichte des Materials ab.
Wie wird ein Lichtstrahl gekr"ummt, wenn die optische Dichte
nicht konstant ist?
\end{aufgabe}

{\parindent 0pt Die Natur regelt die Ausbreitung des Lichtes mit einem
Minimumprinzip.} Zeigen Sie, wie man aus diesem Minimumprinzip
f"ur die Ausbreitung von Lichtstrahlen zum Beispiel in einer Fata Morgana
eine Differentialgleichung ableiten kann. Und auch das
Brechungsgesetz von Snellius folgt aus diesem Optimumprinzip.
