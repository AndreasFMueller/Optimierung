%
% aufgaben.tex -- Kurzbeschreibungen der Aufgabenstellungen fuer die
%                 Seminar-Arbeiten
%
% (c) 2013 Prof Dr Andreas Mueller, Hochschule Rapperswil
% $Id$
%
\documentclass[a4paper,12pt]{article}
\usepackage{german}
\usepackage{amsmath}
\usepackage{amssymb}
\usepackage{amsfonts}
\usepackage{amsthm}
\usepackage{graphicx}
\usepackage{fancyhdr}
\usepackage{textcomp}
\usepackage[all]{xy}
\usepackage{txfonts}
\usepackage{alltt}
\usepackage{verbatim}
\usepackage{paralist}
\usepackage{makeidx}
\usepackage{array}
\begin{document}
\title{Aufgabenstellungen f"ur das Seminar ``Optimierung''}
\author{Andreas M"uller\footnote{
University of Applied Sciences, Oberseestrasse 10, CH-8640 Rapperswil,
Switzerland, Email: {\tt andreas.mueller@hsr.ch}}}
\date{}
\maketitle
\section{Ziele des Seminars}
\begin{itemize}
\item
Teilnehmer verstehen, wie mathematische Optimierungsprobleme gestellt
werden, und k"onnen sie grob klassifizieren.
\item
Lineare Optimierung: Teilnehmer verstehen das Konzept des dualen
Optimierungsproblems und k"onnen ein lineares Optimierungsproblem
mit dem Simplex-Algorithmus l"osen.
\item
Analytische Behandlung nichtlineare Optimierungsprobleme:  Teilnehmer
kennen die wichtigsten notwendigen und hinreichenden Bedingungen f"ur
Extrema nichtlinearer Funktionen mehrere Variablen ohne und mit
Nebenbedingungen und/oder Einschraenkungen, insbesondere das Verfahren
der Lagrange-Multiplikatoren und die Karush-Kuhn-Tucker-Bedinungen.
\item
Numerische Verfahren zur Bestimmung eines Optimums ohne Nebenbedingungen:
\begin{itemize}
\item Simplex-Methode
\item Abstieg
\item GPS
\end{itemize}
\item
Algorithmen f"ur Optimierung mit Nebenbedingungen:
\begin{itemize}
\item Penalty Functions
\end{itemize}

\item
Algorithmen f"ur ganzzahlige Optimierungsprobleme:
\begin{itemize}
\item Branch and Bound
\end{itemize}

\item
Teilnehmer kennen eine Auswahl von modernen (nicht analytischen) Verfahren
zur L"osung von Optimierungsproblemen, zum Beispiel
\begin{itemize}
\item genetische Algorithmen
\item Simulated Annealing
\item Teilchenschwarm-Optimierung
\item Ameisen-Kolonie-Optimierung
\end{itemize}

\item
Teilnehmer verstehen, was ein Variationsproblem ist und k"onnen mit
Hilfe der Euler-Gleichung ein Variationsproblem in eine
Differentialgleichung umwandeln.
\end{itemize}

\section{Auftrag}
Jeder Seminarteilnehmer bearbeitet ein Thema aus dem Bereich der
mathematischen Optimierung, stellt seine Resultate in Form eines
kurzen Papers (wenige Seiten) zusammen und stellt sie ausserdem
in Form einer Pr"asentation im Klassenrahmen vor.

Im Idealfall l"asst sich Ihr Paper fast unver"andert als Abschnitt
in das Script aufnehmen, so dass den Teilnehmern am Ende des Seminars
ein kleiner ``Leitfaden der Optimierungstheorie'' zur Verf"ugung steht,
der als Einstieg in die Fachliteratur dienen kann.
Das Skript im Source Code wird als Creative Commons Projekt auf
Github gefuhrt, unter dem URL
{\tt https://github.com/AndreasFMueller/Optimierung.git}.

Die Pr"asentation sollte nicht nur einfach den Stoff vorf"uhren,
sondern sollte gen"ugend Zahlenbeispiele, Aufgaben oder von Hand
durchf"uhrbare Schritte einhalten, dass die Zuh"orer sich durch
die Probleml"osung durcharbeiten und so ihr Verst"andnis des
Stoffes vertiefen k"onnen. Die Pr"asentation soll sich auch darum
bem"uhen, den Zusammenhang mit den anderen im Seminar behandelten
Themen herzustellen, also zum Beispiel darlegen, warum eine bestimmte
Methode in gewissen F"allen einer anderen, von jemand anderem behandelten
Methode vorzuziehen ist.

Die Darstellung soll vor allem verst"andlich und anschaulich
sein, aber nat"urlich auch so exakt wie m"oglich.
Sie muss alllerdings nicht die strengen Anforderungen an einen
vollst"andigen Beweis erf"ullen. Es ist besser, sich auf die
wesentlichsten F"alle zu beschr"anken, oder die Methode mit
einem niedrigdimensionalen Beispiel zu illustrieren, welche
die Methode verst"andlich macht, als den Zuh"orer mit stundenlangen
Falldiskussionen zu langweilen.

\section{Aufgaben}
Die nachstehend skizzierten Aufgaben werden in dieser Reihenfolge im
Seminar behandelt. Die genauen Termine werden sp"ater bekannt
gegeben. Als Faustregel dient, dass Aufgabe Nummer $n$ in der Woche
$n + 2$ behandelt wird.

\newtheorem{aufgabe}{Aufgabe}

\subsection{Lineare Optimierung}
\input blas.tex
\input simplex.tex
\input karmarkar.tex

\subsection{Nichtlineare Optimierung}
\input fibonacci.tex
\input descent.tex
\input nlsimplex.tex
\input gps.tex

\subsection{Ganzzahlige Optimierung}
\input branchandbound.tex

\subsection{Moderne Verfahren}
\input genalg.tex
\input simann.tex
\input particles.tex
\input antcolony.tex

\subsection{Variationsrechnung}
\input brechung.tex

\section{Voraussetzungen}
Die folgende Tabelle zeigt, welche Voraussetzungen man vor allem
zur Bearbeitung einer der genannten Themen braucht.
\begin{center}
\begin{tabular}{|c|c|c|c|c|}
\hline
Aufgabe&LinAlg&Analysis&FuVar&Programmierung\\
\hline
 1&$\checkmark$&            &            &$\checkmark$\\
 2&$\checkmark$&            &            &$\checkmark$\\
 3&$\checkmark$&$\checkmark$&            &            \\
 4&            &$\checkmark$&            &            \\
 5&$\checkmark$&$\checkmark$&$\checkmark$&            \\
 6&$\checkmark$&            &            &            \\
 7&$\checkmark$&$\checkmark$&$\checkmark$&$\checkmark$\\
 8&$\checkmark$&            &            &            \\
 9&            &$\checkmark$&            &$\checkmark$\\
10&            &$\checkmark$&            &$\checkmark$\\
11&            &$\checkmark$&            &$\checkmark$\\
12&            &$\checkmark$&            &$\checkmark$\\
13&$\checkmark$&$\checkmark$&$\checkmark$&            \\
\hline
\end{tabular}
\end{center}

\section{Bewertung}
\subsection{Note}
Die Vortr"age und Papers werden benotet und geben die Modulnote.
Die Note setzt sich zusammen aus den Resultaten eines Fragebogens,
den die Zuh"orer ausf"ullen, und einer Benotung durch den Dozenten.

\subsection{Kurztests}
Ausserdem wird zu den drei grossen Themen {\em Lineare Optimierung},
{\em Nichtlineare Optimierung} und {\em Variationsrechnung} je
ein Kurztest bestehend aus einer Standardaufgabe aus den jeweiligen
Themengebieten verlangt. In diesem Kurztest d"urfen beliebige Hilfsmittel
verwendet werden, auch Computer und das Internet, und es gibt keine
Zeitlimite (ausser der Schliessung des Ge"audes). Mit dem Kurztest
soll sichergestellt werden, dass jeder Teilnehmer mit den drei
Basismethoden mindestens einmal eine L"osung erarbeitet hat.
Die Kurztests geben keine Note, m"ussen aber alle erf"ullt sein,
damit das Modul als bestanden gilt.

\end{document}
 
